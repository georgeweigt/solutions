\beginsection{28.15}

Prove Leibniz' rule
$$(fg)^{(n)}=\sum_{k=0}^n\left({n\atop k}\right)f^{(k)}(a)g^{(n-k)}(a)$$
{\it Hint:} Use mathematical induction.
For $n=1$, apply Theorem 28.3(iii).

\medskip
The following equalities will be used.
$$\left({n\atop k}\right)+\left({n\atop k-1}\right)=\left({n+1\atop k}\right)
\eqno\hbox{(A)}$$
$$\left({n\atop0}\right)=1\eqno\hbox{(B)}$$
$$\left({n\atop n}\right)=1\eqno\hbox{(C)}$$
Theorem 28.3(iii) is not really needed.
Just using the case $n=0$ for induction step 1.
For induction step 2, show that Leibniz' rule is true for $n+1$ whenever it is
true for $n$.
$$\eqalign{
(fg)^{(n+1)}&=((fg)^{(n)})^\prime\cr
&=\left(\sum_{k=0}^n\left({n\atop k}\right)
f^{(k)}g^{(n-k)}\right)^\prime\cr
&=\sum_{k=0}^n\left({n\atop k}\right)
\left[f^{(k)}\left(g^{(n-k)}\right)^\prime
+\left(f^{(k)}\right)^\prime g^{(n-k)}\right]\cr
&=\sum_{k=0}^n\left({n\atop k}\right)
\left[f^{(k)}g^{(n+1-k)}+f^{(k+1)}g^{(n-k)}\right]\cr
&=\sum_{k=0}^n\left({n\atop k}\right)f^{(k)}g^{(n+1-k)}
+\sum_{k=0}^n\left({n\atop k}\right)f^{(k+1)}g^{(n-k)}
}$$
Use (B) to extract the $k=0$ term of the sum on the left and use (C)
to extract the $k=n$ term of the sum on the right.
$$\eqalign{
(fg)^{(n+1)}&=
f^{(0)}g^{(n+1)}
+\sum_{k=1}^n\left({n\atop k}\right)f^{(k)}g^{(n+1-k)}
+\sum_{k=0}^{n-1}\left({n\atop k}\right)f^{(k+1)}g^{(n-k)}
+f^{(n+1)}g^{(0)}
}$$
Here is the main trick.
The index $k$ is just a dummy variable that takes on a range of values.
We can shift the range of $k$ if we balance the math where $k$ is used.
For the sum on the right, change $k$ so that it runs from
1 to $n$.
$$\eqalign{
(fg)^{(n+1)}&=
f^{(0)}g^{(n+1)}
+\sum_{k=1}^n\left({n\atop k}\right)f^{(k)}g^{(n+1-k)}
+\sum_{k=1}^n\left({n\atop k-1}\right)f^{(k)}g^{(n+1-k)}
+f^{(n+1)}g^{(0)}
}$$
Note that the $n$ in the binomial factor does not change because $n$ does
not depend on $k$.
Next use (A) to combine the two sums.
$$\eqalign{
(fg)^{(n+1)}&=
f^{(0)}g^{(n+1)}
+\sum_{k=1}^n\left(n+1\atop k\right)f^{(k)}g^{(n+1-k)}
+f^{(n+1)}g^{(0)}
}$$
Finally, use (B) and (C) to combine all the terms.
$$(fg)^{(n+1)}=\sum_{k=0}^{n+1}
\left({n+1\atop k}\right)f^{(k)}g^{(n+1-k)}$$
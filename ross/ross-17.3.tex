\beginsection 17.3

Accept on faith that the following familiar functions are continuous on
their domains: $\sin x$, $\cos x$, $e^x$, $2^x$, $\log_e x$ for
$x>0$, $x^p$ for $x>0$ [$p$ any real number].
Use these facts and theorems in this section to prove that the following
functions are continuous.

\medskip
\noindent
(a) $\log_e(1+\cos^4 x)$

\medskip
\itemitem{}
The function $\cos^4 x$ is continuous by Theorem 17.4 (ii),
product of continuous functions.
The function $1+\cos^4 x$ is continuous by Theorem 17.4 (i),
sum of continuous functions.
The function $\log_e(1+\cos^4 x)$ is continuous by Theorem 17.5,
composition of continuous functions.

\medskip
\noindent
(b) $[\sin^2x+\cos^6x]^\pi$

\medskip
\itemitem{}
This is equivalent to $\exp[\pi\log_e(\sin^2x+\cos^6x)]$ which is continuous by
Theorems 17.4 and 17.5.
Now show that $\sin^2x+\cos^6x>0$ to ensure the domain requirement of $\log_e$.
We have $\sin^2x\ge0$, $\cos^6x\ge0$ by even exponents.
By $\sin^2x+\cos^2x=1$, $\sin^2x$ and $\cos^2x$ cannot both be zero for the
same $x$.
If $\cos^2x\ne0$ then $\cos^6x\ne0$.
Therefore $\sin^2x+\cos^6x>0$.

\medskip
\noindent
(c) $2^{x^2}$

\medskip
\itemitem{}
The function $x^2$ is continuous by $x^2=xx$ and
Theorem 17.4 (ii), product of continuous functions.
By Theorem 17.5, composition of continuous fuctions, $2^{x^2}$ is continuous.

\medskip
\noindent
(d) $8^x$

\medskip
\itemitem{}
This is equivalent to $\exp(x\log_e8)$ which is continuous by Theorems 17.4 and 17.5.

\medskip
\noindent
(e) $\tan x$ for $x\ne$ odd multiple of $\pi/2$.

\medskip
\itemitem{}
$\tan x=\sin x/\cos x$ is continuous by Theorem 17.4 (iii), ratio of
continuous functions.
The odd multiple restriction ensures $\cos x\ne0$.

\medskip
\noindent
(f) $x\sin(1/x)$ for $x\ne0$

\medskip
\itemitem{}
$1/x=\exp(-\log_ex)$ is continuous for $x>0$.
For $x<0$, $1/x=-1/|x|$ so $1/x$ is also continous for $x<0$.
Finally, $x\sin(1/x)$ is continuous by Theorems 17.4 and 17.5.

\medskip
\noindent
(g) $x^2\sin(1/x)$ for $x\ne0$

\medskip
\itemitem{}
$x^2\sin(1/x)=x[x\sin(1/x)]$, see (f).

\medskip
\noindent
(h) $(1/x)\sin(1/x^2)$ for $x\ne0$

\medskip
\itemitem{}
The only thing new here is $1/x^2$ which is continuous by $1/x^2=(1/x)(1/x)$
and Theorem 17.4 (ii). See (f) for continuity of $1/x$.

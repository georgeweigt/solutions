\documentclass[12pt]{article}
\usepackage{amsmath}
\parindent=0pt
\begin{document}

Exercise 6.8.
Do the same for the other two entangled triplet states,
\begin{align*}
|T_2\rangle&=\frac{1}{\sqrt2}\left(|uu\rangle+|dd\rangle\right)
\\[1ex]
|T_3\rangle&=\frac{1}{\sqrt2}\left(|uu\rangle-|dd\rangle\right)
\end{align*}
and interpret.

\bigskip
\hrule

\bigskip
See Eigenmath code.

\bigskip
For triplet state $T_2$ the expectation values are
\begin{align*}
\langle\sigma_z\tau_z\rangle&=\langle T_2|\sigma_z\tau_z|T_2\rangle=1
\\[1ex]
\langle\sigma_x\tau_x\rangle&=\langle T_2|\sigma_x\tau_x|T_2\rangle=1
\\[1ex]
\langle\sigma_y\tau_y\rangle&=\langle T_2|\sigma_y\tau_y|T_2\rangle=-1
\end{align*}

For triplet state $T_3$ the expectation values are
\begin{align*}
\langle\sigma_z\tau_z\rangle&=\langle T_3|\sigma_z\tau_z|T_3\rangle=1
\\[1ex]
\langle\sigma_x\tau_x\rangle&=\langle T_3|\sigma_x\tau_x|T_3\rangle=-1
\\[1ex]
\langle\sigma_y\tau_y\rangle&=\langle T_3|\sigma_y\tau_y|T_3\rangle=1
\end{align*}

\end{document}

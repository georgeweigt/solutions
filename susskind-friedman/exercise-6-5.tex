\documentclass[12pt]{article}
\usepackage{amsmath}
\parindent=0pt
\begin{document}

Exercise 6.5.
Prove the following theorem:

\bigskip
When any one of Alice's or Bob's spin operators acts on a
product state, the result is still a product state.

\bigskip
Show that in a product state, the expectation value of any
component of $\vec\sigma$ or $\vec\tau$ is exactly the same as it
would be in the individual single spin states.

\bigskip
\hrule

\bigskip
Let $\psi$ be the product state
\begin{equation*}
\psi=\psi_A\otimes\psi_B
\end{equation*}

For the spin operator $\vec\sigma$ we have the product state
\begin{equation*}
\vec\sigma\psi=\vec\sigma\psi_A\otimes\psi_B
\tag{1}
\end{equation*}

For the spin operator $\vec\tau$ we have the product state
\begin{equation*}
\vec\tau\psi=\psi_A\otimes\vec\tau\psi_B
\tag{2}
\end{equation*}

Hence by (1) and (2) the theorem is proved.

\bigskip
For the second part of the exercise we have
\begin{equation*}
\langle\vec\sigma\rangle
=\psi^*\vec\sigma\psi
=\psi^*(\vec\sigma\psi_A\otimes\psi_B)
=\psi^*\vec\sigma\psi_A\otimes\psi^*\psi_B
\end{equation*}

By orthogonality of $\psi_A$ and $\psi_B$ we have
\begin{equation*}
\langle\vec\sigma\rangle=\psi_A^*\vec\sigma\psi_A\otimes\psi_B^*\psi_B
\end{equation*}

By normalization we have $\psi_B^*\psi_B=1$ hence
\begin{equation*}
\langle\vec\sigma\rangle=\psi_A^*\vec\sigma\psi_A
\end{equation*}

Repeating the same argument for $\vec\tau$ we have
\begin{equation*}
\langle\vec\tau\rangle=\psi_B^*\vec\tau\psi_B
\end{equation*}

\end{document}

\documentclass[12pt]{article}
\usepackage{amsmath}
\parindent=0pt
\begin{document}

Exercise 6.5.
Prove the following theorem:

\bigskip
When any one of Alice's or Bob's spin operators acts on a
product state, the result is still a product state.

\bigskip
Show that in a product state, the expectation value of any
component of $\vec\sigma$ or $\vec\tau$ is exactly the same as it
would be in the individual single spin states.

\bigskip
\hrule

\bigskip
Let $\psi$ be the product state
\begin{equation*}
\psi=\psi_A\otimes\psi_B
\end{equation*}
where
\begin{equation*}
\psi_A=\alpha_u|u\}+\alpha_d|d\},
\qquad
\psi_B=\beta_u|u\rangle+\beta_d|d\rangle
\end{equation*}

For $\vec\sigma$ we have
\begin{equation*}
\vec\sigma\psi=\vec\sigma\psi_A\otimes\psi_B
\tag{1}
\end{equation*}

For $\vec\tau$ we have
\begin{equation*}
\vec\tau\psi=\psi_A\otimes\vec\tau\psi_B
\tag{2}
\end{equation*}

By the fact that (1) and (2) are product states, the theorem is proved.

\bigskip
For the second part of the exercise, let
\begin{equation*}
\vec\sigma\psi_A=\vec c_u|u\}+\vec c_d|d\}
\end{equation*}

It follows that
\begin{equation*}
\vec\sigma\psi=\vec c_u\beta_u|uu\rangle+\vec c_u\beta_d|ud\rangle+\vec c_d\beta_u|du\rangle+\vec c_d\beta_d|dd\rangle
\end{equation*}

Then the expectation $\langle\vec\sigma\rangle$ is
\begin{equation*}
\langle\vec\sigma\rangle=\psi^*\vec\sigma\psi
=\alpha_u^*\vec c_u\beta_u^*\beta_u+\alpha_u^*\vec c_u\beta_d^*\beta_d
+\alpha_d^*\vec c_d\beta_u^*\beta_u+\alpha_d^*\vec c_d\beta_d^*\beta_d
\end{equation*}

Rewrite as
\begin{equation*}
\langle\vec\sigma\rangle=\alpha_u^*\vec c_u(\beta_u^*\beta_u+\beta_d^*\beta_d)
+\alpha_d^*\vec c_d(\beta_u^*\beta_u+\beta_d^*\beta_d)
\end{equation*}

By normalization we have $\beta_u^*\beta_u+\beta_d^*\beta_d=1$ hence
\begin{equation*}
\langle\vec\sigma\rangle=\alpha_u^*\vec c_u+\alpha_d^*\vec c_d=\psi_A^*\vec\sigma\psi_A
\end{equation*}

Hence the expectation $\langle\vec\sigma\rangle$
is the same for the product state and the single spin state.
\begin{equation*}
\langle\vec\sigma\rangle=\psi^*\vec\sigma\psi=\psi_A^*\vec\sigma\psi_A
\end{equation*}

By a similar argument for $\vec\tau$ we have
\begin{equation*}
\langle\vec\tau\rangle=\psi^*\vec\tau\psi=\psi_B^*\vec\tau\psi_B
\end{equation*}

\end{document}

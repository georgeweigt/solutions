\documentclass[12pt]{article}
\usepackage{amsmath}
\parindent=0pt
\begin{document}

(4.5)
Take any unit 3-vector $\vec n$ and form the operator
\begin{equation*}
\mathbf H=\frac{\hbar\omega}{2}\sigma\cdot\vec n
\end{equation*}

Find the energy eigenvalues and eigenvectors by solving the
time-independent Schrodinger equation.
Recall that Eq.~3.23 gives $\sigma\cdot\vec n$ in component form.

\bigskip
\hrule

\bigskip
\begin{equation*}
\sigma_n=\begin{pmatrix}
n_z & n_x-in_y
\\
n_x+in_y & -n_z
\end{pmatrix}
\tag{3.23}
\end{equation*}

Then by hypothesis
\begin{equation*}
\mathbf H=\frac{\hbar\omega}{2}
\begin{pmatrix}
n_z & n_x-in_y
\\
n_x+in_y & -n_z
\end{pmatrix}
\end{equation*}

Equation (4.28) is the time-independent Schrodinger equation.
\begin{equation*}
\mathbf H|E_j\rangle=E_j|E_j\rangle
\tag{4.28}
\end{equation*}

The eigenvalues $E_j$ of $\mathbf H$ are solutions to $\det(\mathbf H-E_j\mathbf I)=0$.
Hence
\begin{align*}
0=\det(\mathbf H-E_j\mathbf I)
&=\left|
\begin{matrix}
\frac{\hbar\omega}{2}n_z-E_j & \frac{\hbar\omega}{2}(n_x-in_y)
\\[1ex]
\frac{\hbar\omega}{2}(n_x+in_y) & -\frac{\hbar\omega}{2}n_z-E_j
\end{matrix}
\right|
\\
&=E_j^2-\left(\frac{\hbar\omega}{2}\right)^2\left(n_x^2+n_y^2+n_z^2\right)
\end{align*}

By hypothesis $\vec n$ is a unit vector hence
\begin{equation*}
E_j^2-\left(\frac{\hbar\omega}{2}\right)^2=0
\end{equation*}

Therefore the eigenvalues are
\begin{equation*}
E_1=\frac{\hbar\omega}{2},\qquad E_2=-\frac{\hbar\omega}{2}
\end{equation*}

To find the eigenvectors we use the formula
\begin{equation*}
(\mathbf H-E_j\mathbf I)|E_j\rangle=\begin{pmatrix}0\\0\end{pmatrix}
\end{equation*}

Let
\begin{equation*}
|E_j\rangle=\begin{pmatrix}\cos\alpha\\\sin\alpha\end{pmatrix}
\end{equation*}

Then
\begin{equation*}
\begin{pmatrix}
\frac{\hbar\omega}{2}n_z-E_j & \frac{\hbar\omega}{2}(n_x-in_y)
\\[1ex]
\frac{\hbar\omega}{2}(n_x+in_y) & -\frac{\hbar}{\omega}{2}n_z-E_j
\end{pmatrix}
\begin{pmatrix}\cos\alpha\\\sin\alpha\end{pmatrix}=\begin{pmatrix}0\\0\end{pmatrix}
\end{equation*}

By multiplication of the first row times $|E_j\rangle$ we have
\begin{equation*}
\left(\frac{\hbar\omega}{2}n_z-E_j\right)\cos\alpha+\frac{\hbar\omega}{2}(n_x-in_y)\sin\alpha=0
\tag{1}
\end{equation*}

From exercise 3.4 let
\begin{equation*}
\begin{aligned}
n_x&=\sin\theta\cos\phi
\\
n_y&=\sin\theta\sin\phi
\\
n_z&=\cos\theta
\end{aligned}
\tag{2}
\end{equation*}

Substitute (2) into (1) to obtain
\begin{equation*}
\left(\frac{\hbar\omega}{2}\cos\theta-E_j\right)\cos\alpha
+\frac{\hbar\omega}{2}\sin\theta\cos\phi\sin\alpha
-i\frac{\hbar\omega}{2}\sin\theta\sin\phi\sin\alpha=0
\end{equation*}

The imaginary term must vanish hence we take $\phi=0$ leaving
\begin{equation*}
\left(\frac{\hbar\omega}{2}\cos\theta-E_j\right)\cos\alpha+\frac{\hbar\omega}{2}\sin\theta\sin\alpha=0
\end{equation*}

Rewrite as
\begin{equation*}
\frac{\hbar\omega}{2}(\cos\theta\cos\alpha+\sin\theta\sin\alpha)=E_j\cos\alpha
\end{equation*}

By angle difference identity
\begin{equation*}
\frac{\hbar\omega}{2}\cos(\theta-\alpha)=E_j\cos\alpha
\tag{3}
\end{equation*}

Substitute $E_j=E_1=\hbar\omega/2$ into (3) to obtain
\begin{equation*}
\cos(\theta-\alpha)=\cos\alpha
\end{equation*}

It follows that
\begin{equation*}
\alpha=\frac{\theta}{2}
\end{equation*}

Hence
\begin{equation*}
|E_1\rangle=\begin{pmatrix}\cos\frac{\theta}{2}\\[1ex]\sin\frac{\theta}{2}\end{pmatrix}
\end{equation*}

Substitute $E_j=E_2=-\hbar\omega/2$ into (3) to obtain
\begin{equation*}
\cos(\theta-\alpha)=-\cos\alpha
\end{equation*}

Rewrite as
\begin{equation*}
\sin(\theta-\alpha+\pi/2)=\sin(\alpha-\pi/2)
\end{equation*}

It follows that
\begin{equation*}
\alpha=\frac{\theta}{2}+\frac{\pi}{2}
\end{equation*}

Hence
\begin{equation*}
|E_2\rangle=\begin{pmatrix}\cos\left(\frac{\theta}{2}+\frac{\pi}{2}\right)
\\[1ex]
\sin\left(\frac{\theta}{2}+\frac{\pi}{2}\right)
\end{pmatrix}
=\begin{pmatrix}
-\sin\frac{\theta}{2}
\\[1ex]
\cos\frac{\theta}{2}
\end{pmatrix}
\end{equation*}

\end{document}

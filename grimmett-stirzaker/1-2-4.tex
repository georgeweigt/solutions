\section*{1.2.4}
Let $\mathcal F$ be a $\sigma$-field of subsets of $\Omega$ and suppose that $B\in\mathcal F$.
Show that ${\mathcal G}=\{A\cap B:A\in{\mathcal F}\}$ is a $\sigma$-field of subsets of $B$.

\bigskip
\noindent
Show that $\emptyset\in\mathcal G$.
\begin{enumerate}
\item Let $A=\emptyset$.
\item Then $A\in\mathcal F$ and $A\cap B=\emptyset$.
\item Hence $\emptyset\in\mathcal G$.
\end{enumerate}

\bigskip
\noindent
Show that if $B_1,B_2,\ldots\in\mathcal G$ then $\cup_{i=1}^\infty B_i\in\mathcal G$.
\begin{enumerate}
\item Let $B_1,B_2,\ldots\in\mathcal G$.
\item By the definition of $\mathcal G$ there are corresponding $A_i\in\mathcal F$
such that\break $A_i\cap B=B_i$.
\item Because $\mathcal F$ is a $\sigma$-field there is an $A$ in $\mathcal F$ such that\
$A=\cup_{i=1}^\infty A_i$.
\item By the definition of $\mathcal G$ we have $A\cap B\in\mathcal G$.
\item By the distributive law we have\hfill\break
$A\cap B=(\cup_{i=1}^\infty A_i)\cap B=\cup_{i=1}^\infty(A_i\cap B)
=\cup_{i=1}^\infty B_i$.
\item Hence $\cup_{i=1}^\infty B_i\in\mathcal G$.
\end{enumerate}

\bigskip
\noindent
Show that if $C\in\mathcal G$ then $C^c\in\mathcal G$.
\begin{enumerate}
\item Let $C\in\mathcal G$.
\item Then there is an $A\in\mathcal F$ such that $A\cap B=C$.
\item Since $\mathcal F$ is a field we have $A^c\in\mathcal F$.
\item Consequently $A^c\cap B\in\mathcal G$.
\item Note that $C^c=B-C=B\cap(A\cap B)^c=B\cap(A^c\cup B^c)=B\cap A^c$.
\item Therefore $C^c\in\mathcal G$.
\end{enumerate}

\bigskip
\noindent
Show that every $C\in\mathcal G$ is a subset of $B$.
\begin{enumerate}
\item Let $B\in\mathcal F$ and $C\in\mathcal G$.
\item By the definition of $\mathcal G$ we have $C=A\cap B$ for some
$A\in\mathcal F$.
\item It follows that for every $\omega\in C$ we have $\omega\in B$,
hence $C\subseteq B$.
\item Hence every $C\in\mathcal G$ is a subset of $B$.
\end{enumerate}

\bigskip
\noindent
Therefore $\mathcal G$ is a $\sigma$-field and
$A\subseteq B$ for every $A\in\mathcal G$.

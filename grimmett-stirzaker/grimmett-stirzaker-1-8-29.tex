\section*{1.8.29}
The event $A$ is said to be repelled by the event $B$ if
$P(A\mid B)<P(A)$, and to be attracted by $B$ if $P(A\mid B)>P(A)$.
Show that if $B$ attracts $A$, then $A$ attracts $B$, and $B^c$
repels $A$. If $A$ attracts $B$, and $B$ attracts $C$, does
$A$ attract $C$?

\bigskip
\noindent
Let $B$ attract $A$. Then
$$P(A\mid B)>P(A)$$
Hence
$${P(A\cap B)\over P(B)}>P(A)$$
Consequently
$${P(A\cap B)\over P(A)}>P(B)\eqno(1)$$
from which it follows
$$P(B\mid A)>P(B)$$
hence $A$ attracts $B$.
For $B^c$ we have
$$P(A\cap B^c)=P(A)-P(A\cap B)$$
hence
$$P(A\mid B^c)={P(A)-P(A\cap B)\over1-P(B)}$$
From (1) we obtain
$$P(A\cap B)>P(A)P(B)$$
hence
$$P(A\cap B^c)<{P(A)-P(A)P(B)\over1-P(B)}=P(A)$$
Therefore $B^c$ repels $A$.

\bigskip
\noindent
For the final question, the answer is no by the following
counterexample.
If $A$ and $C$ are disjoint then
$P(C\mid A)=0$ so $A$ does not attract $C$.


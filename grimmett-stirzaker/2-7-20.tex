\section*{2.7.20}
(a) If $U$ and $V$ are jointly continuous, show that $P(U=V)=0$.

\bigskip
\noindent
The challenge here is to write $P(U=V)$ in terms of the joint
distribution function $F(u,v)$.
If $U=V$ then the random vector $(U,V)$ falls on the line $u=v$.
The probability that $(U,V)$ falls on the line $u=v$ is
\begin{eqnarray*}
P(U=V)&=&\lim_{h\downarrow0}P(u-h<U\le u,u-h<V\le u) \\
&=&\lim_{h\downarrow0}\bigg[
F(u,u)-F(u-h,u)-F(u,u-h)+F(u-h,u-h)\bigg]\\
&=&0
\end{eqnarray*}

\bigskip
\noindent
(b) Let $X$ be uniformly distributed on $(0,1)$, and let
$Y=X$. Then $X$ and $Y$ are continuous, and $P(X=Y)=1$.
Is there a contradiction here?

\bigskip
\noindent
We have
$$P(X=Y)=P(\{\omega:X(\omega)=Y(\omega)\})$$
Since $X(\omega)=Y(\omega)$ for all $\omega$ we have
$$P(\{\omega:X(\omega)=Y(\omega)\})=P(\Omega)$$
hence
$$P(X=Y)=1$$
For part (a) we do not have $U(\omega)=V(\omega)$ for all $\omega$.


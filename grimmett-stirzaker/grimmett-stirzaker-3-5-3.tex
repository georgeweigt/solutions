\section*{3.5.3}
Let $X$ be Poisson distributed where 
$P(X=n)=p_n(\lambda)=\lambda^ne^{-\lambda}/n!$
for $n\ge0$. Show that
$P(X\le n)=1-\int_0^\lambda p_n(x)\,dx$.

\bigskip
\noindent
We have
$$P(X\le n)=\sum_{k=0}^np_k(\lambda)
=\sum_{k=0}^n{\lambda^ke^{-\lambda}\over k!}$$
Taking the derivative we have
\begin{eqnarray*}
{d\over d\lambda}\sum_{k=0}^n{\lambda^ke^{-\lambda}\over k!}
&=&\sum_{k=0}^n\left({k\lambda^{k-1}e^{-\lambda}\over k!}
-{\lambda^ke^{-\lambda}\over k!}\right)
\\
&=&e^{-\lambda}\left(
\sum_{k=0}^n{k\lambda^{k-1}\over k!}
-\sum_{k=0}^n{\lambda^k\over k!}
\right)
\\
&=&e^{-\lambda}\left(
\sum_{k=0}^{n-1}{\lambda^k\over k!}
-\sum_{k=0}^n{\lambda^k\over k!}
\right)
\\
&=&-{\lambda^ne^{-\lambda}\over n!}=-p_n(\lambda)
\end{eqnarray*}
By the above result and the fundamental theorem of calculus we have
\begin{eqnarray*}
\int_0^\lambda-p_n(x)\,dx
&=&\sum_{k=0}^n p_k(\lambda)
-\sum_{k=0}^n p_k(0)
\\
&=&\sum_{k=0}^n p_k(\lambda)-1
\end{eqnarray*}
Note that, in the previous calculation, $\sum p_k(0)=p_0(0)=1$
because $0^0=1$ and $0!=1$.
We can rewrite the above result as
$$1-\int_0^\lambda p_n(x)\,dx
=\sum_{k=0}^n p_k(\lambda)=P(X\le n)$$

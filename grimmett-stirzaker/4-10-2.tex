\section*{4.10.2}
Show that the mean of the $t(r)$ distribution is 0, and that the
mean of the $F(r,s)$ distribution is $s/(s-2)$ if $s>2$.
What happens if $s\le2$?

\bigskip
\noindent
The mean of $t(r)$ is
$$\int_{-\infty}^\infty t f_T(t)\,dt$$
Since $f_T(t)$ is an even function, $tf_T(t)$ is an odd function
hence the integral is zero.

\bigskip
\noindent
For $F(r,s)$ we are given the density function
$$f(x)={r\Gamma(r/2+s/2)\over s\Gamma(r/2)\Gamma(s/2)}
\cdot
{(rx/s)^{r/2-1}\over(1+rx/s)^{r/2+s/2}}
$$
Therefore the mean $\mu$ is
$$\mu={r\Gamma(r/2+s/2)\over s\Gamma(r/2)\Gamma(s/2)}
\int_0^\infty
{x(rx/s)^{r/2-1}\over(1+rx/s)^{r/2+s/2}}\,dx$$
We can pass $rx/s$ into the numerator of the integrand.
$$\mu={\Gamma(r/2+s/2)\over\Gamma(r/2)\Gamma(s/2)}
\int_0^\infty
{(rx/s)^{r/2}\over(1+rx/s)^{r/2+s/2}}\,dx$$
Now define $u$ and $v$ as follows.
\begin{eqnarray*}
u/2-1&=&r/2\cr
v/2+1&=&s/2
\end{eqnarray*}
Now we can write the integral as
$$\mu={\Gamma(r/2+s/2)\over\Gamma(r/2)\Gamma(s/2)}
\int_0^\infty
{(rx/s)^{u/2-1}\over(1+rx/s)^{u/2+v/2}}\,dx$$
Next, define
$$x=\left({v+2\over u-2}\right)uy/v
\qquad dx/dy=\left({v+2\over u-2}\right)\left({u\over v}\right)$$
Then
$$rx/s=uy/v$$
and we have
$$\mu={\Gamma(r/2+s/2)\over\Gamma(r/2)\Gamma(s/2)}
\left({v+2\over u-2}\right)\left({u\over v}\right)
\int_0^\infty
{(uy/v)^{u/2-1}\over(1+uy/v)^{u/2+v/2}}\,dy$$
The integral is another $F$ distribution so it integrates
to one over the normalization constant.
$$\mu={\Gamma(r/2+s/2)\over\Gamma(r/2)\Gamma(s/2)}
\left({v+2\over u-2}\right)\left({u\over v}\right)
\left({v\over u}\right)
{\Gamma(u/2)\Gamma(v/2)\over\Gamma(u/2+v/2)}$$
Now back-substitute $u=r+2$ and $v=s-2$.
$$\mu={\Gamma(r/2+s/2)\over\Gamma(r/2)\Gamma(s/2)}
\left({s\over r}\right)
{\Gamma(r/2+1)\Gamma(s/2-1)\over\Gamma(r/2+s/2)}$$
Cancel terms and apply the identity $\Gamma(z+1)=z\Gamma(z)$.
$$\mu={1\over\Gamma(r/2)(s/2-1)\Gamma(s/2-1)}
\left({s\over r}\right)
{(r/2)\Gamma(r/2)\Gamma(s/2-1)\over1}$$
Now we can cancel terms and obtain
$$\mu={s\over s-2}$$
If $s\le2$ then the mean does not exist.

\section*{1.8.24}
The probability that the first $k$ balls are blue and the $k+1$ ball is red is
$$p={b\over r+b}\times{b-1\over r+b-1}\times\cdots\times
{b-k+1\over r+b-k+1}\times{r\over r+b-k}$$
In terms of factorials we have
$$p={b!/(b-k)!\over(r+b)!/(r+b-k-1)!}\times r$$
Noting that $r=r!/(r-1)!$ we can write
$$p={(r+b-k-1)!\over(r-1)!\,(b-k)!}\times{b!\,r!\over(r+b)!}$$
We have
$$
{r+b-k-1\choose r-1}={(r+b-k-1)!\over (r-1)!\,(b-k)!}
\quad
\hbox{and}
\quad
{r+b\choose b}={(r+b)!\over b!\,r!}
$$
Therefore
$$p={r+b-k-1\choose r-1}\bigg/{r+b\choose b}$$

\bigskip
\noindent
(b) Find the probability that the last ball drawn is red.

\bigskip
\noindent
After the $k+1$ draw, the urn contains $r-1$ red balls out of
a total of $r+b-k-1$.
Therefore the probability that the last ball drawn is red is
$$p={r-1\over r+b-k-1}$$

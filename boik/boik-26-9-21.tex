\section{Section 25.9, problem 21}
Let ${\bf X}:n\times p$ be a rank $r$ matrix and define $\bf H$ as
${\bf H}={\bf X}({\bf X}'{\bf X})^{-{}}{\bf X}'$.
Prove the following results.

\bigskip
\noindent
(a) $\bf H$ is idempotent.

\bigskip
\noindent
By definition of a generalized inverse we have
$${\bf X}'{\bf X}({\bf X}'{\bf X})^{-{}}{\bf X}'{\bf X}
={\bf X}'{\bf X}$$
By Theorem 8.15 (b) on p. 60 we can cancel the leftmost ${\bf X}'$
on both sides of the above equation
regardless of the row rank of $\bf X$.
Therefore
$${\bf X}({\bf X}'{\bf X})^{-{}}{\bf X}'{\bf X}={\bf X}$$
Hence
\begin{eqnarray*}
{\bf HH}&=&
\bigg[{\bf X}({\bf X}'{\bf X})^{-{}}{\bf X}'\bigg]\cdot
\bigg[{\bf X}({\bf X}'{\bf X})^{-{}}{\bf X}'\bigg]\\
&=&\bigg[{\bf X}({\bf X}'{\bf X})^{-{}}
{\bf X}'{\bf X}\bigg]({\bf X}'{\bf X})^{-{}}{\bf X}'\\
&=&{\bf X}({\bf X}'{\bf X})^{-{}}{\bf X}'\\
&=&{\bf H}
\end{eqnarray*}
Therefore $\bf H$ is idempotent.

\bigskip
\noindent
(b) ${\bf HX}={\bf X}$.

\bigskip
\noindent
By part (a) above we have
$${\bf X}({\bf X}'{\bf X})^{-{}}{\bf X}'{\bf X}={\bf X}$$
Therefore
$${\bf HX}=\bigg[{\bf X}({\bf X}'{\bf X})^{-{}}{\bf X}'\bigg]
{\bf X}={\bf X}$$

\bigskip
\noindent
(c) ${\bf Ha}={\bf a}$ for any ${\bf a}\in{\cal R}({\bf X})$.

\bigskip
\noindent
Since ${\bf a}\in{\cal R}({\bf X})$ we have
${\bf a}={\bf Xb}$ for some $\bf b$.
Then, because ${\bf HX}={\bf X}$ in part (b) above,
we have ${\bf HXb}={\bf Xb}$.
Hence ${\bf Ha}={\bf a}$.

\bigskip
\noindent
(d) $\mathop{\rm rank}({\bf H})=\mathop{\rm rank}({\bf X})
=\mathop{\rm trace}({\bf H})$.

\bigskip
\noindent
Since ${\bf X}={\bf HX}$ and
${\bf H}={\bf X}({\bf X}'{\bf X})^{-{}}{\bf X}'$ we have
$$
\mathop{\rm rank}({\bf X})\le
\mathop{\rm rank}({\bf H})\quad\hbox{and}\quad
\mathop{\rm rank}({\bf H})\le\mathop{\rm rank}({\bf X})
$$
Hence $\mathop{\rm rank}({\bf H})=\mathop{\rm rank}({\bf X})$.
By Theorem 11.18 (b) on p.\ 92 we have
$\mathop{\rm trace}({\bf H})=\mathop{\rm rank}({\bf H})$.
%However, let us go the extra mile and prove it anyway.
%By Theorem 11.30 we have
%$${\bf H}={\bf C}({\bf C}'{\bf C})^{-1}{\bf C}'$$
%where ${\bf X}={\bf CR}$ is the full rank factorization of $\bf X$
%and ${\bf\Omega}^{-1}={\bf I}_r$.
%Then
%\begin{eqnarray*}
%\mathop{\rm trace}({\bf H})
%&=&\mathop{\rm trace}[{\bf C}({\bf C}'{\bf C})^{-1}{\bf C}']\\
%&=&\mathop{\rm trace}[{\bf C}'{\bf C}({\bf C}'{\bf C})^{-1}]\\
%&=&\mathop{\rm trace}({\bf I}_r)\\
%&=&r\\
%&=&\mathop{\rm rank}({\bf X})
%\end{eqnarray*}

\bigskip
\noindent
(e) $\bf H$ doesn't depend on the choice of generalized inverse.

\bigskip
\noindent
By Theorem 11.21, $\bf H$ is unique. Hence, choosing a different generalized inverse
must yield the same $\bf H$.
Therefore $\bf H$ doesn't depend on the choice of generalized inverse.

\bigskip
\noindent
(f) Prove that $\bf H$ is $\mathop{\rm ppo}({\bf X})$.

\bigskip
\noindent
By Theorem 11.22 on p.\ 95, $\mathbf H$ is a perpendicular projection
operator iff $\mathbf H=\mathbf H'$. Consequently,
we need to show that $\bf H$ is symmetric so let us compute the
transpose of $\bf H$.
\begin{eqnarray*}
\mathbf H'&=&[{\bf X}({\bf X}'{\bf X})^{-{}}{\bf X}']'\\
%&=&(\mathbf X')'[(\mathbf X'\mathbf X)^{-{}}]'\mathbf X'\\
&=&\mathbf X[(\mathbf X'\mathbf X)^{-{}}]'\mathbf X'
\end{eqnarray*}
Since $\mathbf X'\mathbf X$ is symmetric, we can use Theorem 8.7
on p.\ 54 to conclude that
$[(\mathbf X'\mathbf X)^{-{}}]'$ is a generalized inverse of
$\mathbf X'\mathbf X$.
By part (e) above, $\bf H$ doesn't depend on the choice of generalized inverse.
Therefore we must have
$$\mathbf H=\mathbf X[(\mathbf X'\mathbf X)^{-{}}]'\mathbf X'=
\mathbf H'$$
Hence $\mathbf H$ is $\mathop{\rm ppo}(\mathbf X)$.

\section{Section 25.8, problem 4}
Let $\bf A$ be a $p\times q$ matrix and let $\bf B$ be a $q\times s$
matrix.

\bigskip
\noindent
(a) Verify that ${\cal R}({\bf AB})\subseteq{\cal R}({\bf A})$.

\bigskip
\noindent
Let ${\bf x}\in{\cal R}({\bf AB})$.
Then by definition ${\bf x}={\bf ABb}$ for some $\bf b$.
Let ${\bf a}={\bf Bb}$.
Then ${\bf x}={\bf Aa}$.
It follows that ${\bf x}\in{\cal R}({\bf A})$.
Hence ${\cal R}({\bf AB})\subseteq{\cal R}({\bf A})$.

\bigskip
\noindent
(b) Verify that ${\cal R}({\bf AB})={\cal R}({\bf A})$
if $\mathop{\rm rank}({\bf AB})=\mathop{\rm rank}({\bf A})$.

\bigskip
\noindent
The first step is to show that the dimensions of the vector spaces
${\cal R}({\bf AB})$ and ${\cal R}({\bf A})$ are equal.
By Theorem 11.4 (d) on p. 83,
$\mathop{\rm Dim}[{\cal R}({\bf X})]=\mathop{\rm rank}({\bf X})$.
Therefore
$\mathop{\rm rank}({\bf AB})=\mathop{\rm rank}({\bf A})$
implies
$$\mathop{\rm Dim}[{\cal R}({\bf AB})]
=\mathop{\rm Dim}[{\cal R}({\bf A})]
$$
From part (a) above we have
$${\cal R}({\bf AB})\subseteq{\cal R}({\bf A})$$
By method 3 on p. 82, if two vector spaces have the same dimension
and one is a subset of the other, then the two vector spaces
are equivalent. Hence
$${\cal R}({\bf AB})={\cal R}({\bf A})$$

\section{Section 25.7, problem 3}
Let
$${\bf A}=\left(\begin{matrix}
1 & 2 & 3 & 1\cr
2 & 4 & 6 & 1\cr
3 & 6 & 9 & 1
\end{matrix}\right)
$$
(a) Use Matlab to compute the non-full rank SVD of $\bf A$.

\bigskip
\noindent
Using R instead of Matlab, we have
\begin{verbatim}
> A = rbind(c(1,2,3,1),c(2,4,6,1),c(3,6,9,1))
> A
     [,1] [,2] [,3] [,4]
[1,]    1    2    3    1
[2,]    2    4    6    1
[3,]    3    6    9    1
> tmp = svd(A)
> U = tmp$u
> D = diag(tmp$d)
> V = tmp$v
> U
           [,1]       [,2]       [,3]
[1,] -0.2718819 -0.8714434  0.4082483
[2,] -0.5356711 -0.2153830 -0.8164966
[3,] -0.7994603  0.4406774  0.4082483
> D
         [,1]     [,2]         [,3]
[1,] 14.09173 0.000000 0.000000e+00
[2,]  0.00000 0.650392 0.000000e+00
[3,]  0.00000 0.000000 8.936188e-16
> V
           [,1]        [,2]          [,3]
[1,] -0.2655177  0.03047832 -9.636144e-01
[2,] -0.5310354  0.06095663  1.445191e-01
[3,] -0.7965531  0.09143495  2.248587e-01
[4,] -0.1140394 -0.99347623 -1.665335e-15
> U %*% D %*% t(V) # check UDV' = A
     [,1] [,2] [,3] [,4]
[1,]    1    2    3    1
[2,]    2    4    6    1
[3,]    3    6    9    1
\end{verbatim}

\bigskip
\noindent
(b) Use Matlab to compute the full rank SVD of $\bf A$
and compute the Moore-Penrose inverse of $\bf A$.
Verify that the Moore-Penrose conditions are satisfied.

\bigskip
\noindent
From $\bf D$ above we see that we only need the first two columns
of $\bf U$ and $\bf V$.
Using R instead of Matlab we have
\begin{verbatim}
> U1 = U[,1:2]
> D1 = D[1:2,1:2]
> V1 = V[,1:2]
> U1 %*% D1 %*% t(V1) # check
     [,1] [,2] [,3] [,4]
[1,]    1    2    3    1
[2,]    2    4    6    1
[3,]    3    6    9    1
> U1
           [,1]       [,2]
[1,] -0.2718819 -0.8714434
[2,] -0.5356711 -0.2153830
[3,] -0.7994603  0.4406774
> D1
         [,1]     [,2]
[1,] 14.09173 0.000000
[2,]  0.00000 0.650392
> V1
           [,1]        [,2]
[1,] -0.2655177  0.03047832
[2,] -0.5310354  0.06095663
[3,] -0.7965531  0.09143495
[4,] -0.1140394 -0.99347623
\end{verbatim}

\bigskip
\noindent
Now compute ${\bf A}^+={\bf V}_1{\bf D}_1^{-1}{\bf U}_1^T$ and check.
\begin{verbatim}
> G = V1 %*% solve(D1) %*% t(U1)
> G
            [,1]          [,2]        [,3]
[1,] -0.03571429 -3.920475e-16  0.03571429
[2,] -0.07142857 -3.469447e-17  0.07142857
[3,] -0.10714286 -1.734723e-17  0.10714286
[4,]  1.33333333  3.333333e-01 -0.66666667
> # check condition 1 (use subtraction so all zeroes means yes)
> round( A %*% G %*% A - A, 6)
     [,1] [,2] [,3] [,4]
[1,]    0    0    0    0
[2,]    0    0    0    0
[3,]    0    0    0    0
> # check condition 2
> round( G %*% A %*% G - G, 6)
     [,1] [,2] [,3]
[1,]    0    0    0
[2,]    0    0    0
[3,]    0    0    0
[4,]    0    0    0
> # check condition 3
> round( t(A %*% G) - A %*% G, 6)
     [,1] [,2] [,3]
[1,]    0    0    0
[2,]    0    0    0
[3,]    0    0    0
> # check condition 4
> round( t(G %*% A) - G %*% A, 6)
     [,1] [,2] [,3] [,4]
[1,]    0    0    0    0
[2,]    0    0    0    0
[3,]    0    0    0    0
[4,]    0    0    0    0
> # Finally, check against result of ginv()
> library(MASS)
> round( G - ginv(A), 6)
     [,1] [,2] [,3]
[1,]    0    0    0
[2,]    0    0    0
[3,]    0    0    0
[4,]    0    0    0
\end{verbatim}

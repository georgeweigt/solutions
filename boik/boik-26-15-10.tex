\section{Section 25.15, problem 10}
Suppose that $\mathbf X$ is $n\times p$ with rank $p$ and
$\bm{\widehat\beta}=(\mathbf X^T\mathbf X)^{-1}\mathbf X^T
\mathbf y$ is the OLS estimator of $\bm\beta$ in the model
$\mathbf y\sim N(\mathbf X\bm\beta,\sigma^2\mathbf I_n)$.

\bigskip
\noindent
(a) Find
\[
Q=\max_{\mathbf c\ne\mathbf0}
{(\mathbf c^T\bm{\widehat\beta})^2
\over\mathbf c^T(\mathbf X^T\mathbf X)^{-1}\mathbf c}
\]

\bigskip
\noindent
Note that
\[
(\mathbf c^T\bm{\widehat\beta})^2
=\mathbf c^T\bm{\widehat\beta}\bm{\widehat\beta}^T\mathbf c
\]
By Theorem 9.11 on p.\ 75 we have
\[
\max_{\mathbf c\ne\mathbf0}
{\mathbf c^T\bm{\widehat\beta}\bm{\widehat\beta}^T\mathbf c
\over\mathbf c^T(\mathbf X^T\mathbf X)^{-1}\mathbf c}
=\hbox{maximum eigenvalue of}\;
\mathbf X^T\mathbf X\bm{\widehat\beta}\bm{\widehat\beta}^T
\]
We have
\begin{align*}
\mathbf X^T\mathbf X\bm{\widehat\beta}\bm{\widehat\beta}^T
&=
\mathbf X^T\mathbf X(\mathbf X^T\mathbf X)^{-1}
\mathbf X^T\mathbf y\mathbf y^T\mathbf X(\mathbf X^T\mathbf X)^{-1}\\
&=
\mathbf X^T\mathbf y\mathbf y^T\mathbf X(\mathbf X^T\mathbf X)^{-1}
\end{align*}
%
By Theorem 9.4 (c) on p.\ 71, if $\lambda$ is an eigenvalue of
the above then $\lambda$ is an eigenvalue of
\[
\mathbf y^T\mathbf X(\mathbf X^T\mathbf X)^{-1}\mathbf X^T\mathbf y\\
=
\mathbf y^T\mathbf H\mathbf y
\]
Since $\mathbf y^T\mathbf{Hy}$ is a scalar we have
\[
Q=\mathbf y^T\mathbf{Hy}
\]

\bigskip
\noindent
(b) Give an expression for the maximizer.

\bigskip
\noindent
By Corollary 9.11.1 on p.\ 76, the maximizer of $Q$ is the eigenvector
of $\mathbf X^T\mathbf X\bm{\widehat\beta}\bm{\widehat\beta}^T$
that corresponds to the largest eigenvalue of same.
Hence we want to solve for the eigenvector $\mathbf c$ in
\[
\mathbf X^T\mathbf y\mathbf y^T\mathbf X(\mathbf X^T\mathbf X)^{-1}
\mathbf c
=\mathbf y^T\mathbf{Hyc}
\]
Let $\mathbf c=\mathbf X^T\mathbf y$. Then
\[
\mathbf X^T\mathbf y\mathbf y^T\mathbf X(\mathbf X^T\mathbf X)^{-1}
\mathbf X^T\mathbf y
=
\mathbf X^T\mathbf y\mathbf y^T\mathbf{Hy}
=
\mathbf y^T\mathbf{HyX}^T\mathbf y
\]
since $\mathbf y^T\mathbf{Hy}$ is a scalar. Hence
\[
\mathbf c=\mathbf X^T\mathbf y
\]

\bigskip
\noindent
(c) Find the distribution of $Q$.

\bigskip
\noindent
We have
\[
{(\mathbf c^T\bm{\widehat\beta})^2
\over\mathbf c^T(\mathbf X^T\mathbf X)^{-1}\mathbf c}
=
{[\mathbf y^T\mathbf X(\mathbf X^T\mathbf X)^{-1}
\mathbf X^T\mathbf y]^2
\over\mathbf y^T\mathbf X(\mathbf X^T\mathbf X)^{-1}\mathbf X^T\mathbf y
}
=\mathbf y^T\mathbf{Hy}
\]
Let $\mathbf A=\mathbf H/\sigma^2$.
Then $\mathbf A\sigma^2\mathbf I$ is idempotent.
By Theorem 14.7 on p.\ 146 we have $\mathbf y^T\mathbf{Ay}
\sim\chi^2(p,\eta)$ where
\[
\eta={1\over2}\bm\beta^T\mathbf X^T\mathbf A\mathbf X\bm\beta
=
{1\over2\sigma^2}\bm\beta^T\mathbf X^T\mathbf{HX}\bm\beta
=
{1\over2\sigma^2}\bm\beta^T\mathbf X^T\mathbf X\bm\beta
\]
Hence $Q/\sigma^2\sim\chi^2(p,\eta)$.



\section*{26.4.2}
Let $\bf A$ be a $p\times p$ symmetric matrix with rank $r$.
Justify your answers to the following T/F questions.

\bigskip
\noindent
(a) T or F: ${\bf A}^{-{}}$ must be symmetric for any $r$.

\bigskip
\noindent
False. By Theorem 8.6 (b) on p. 54,
if $\bf A$ is singular (which is the case
when $r<p$) then a non-symmetric generalized inverse of $\bf A$
always exists.

\bigskip
\noindent
(b) T or F: ${\bf A}^{-{}}$ can be chosen to be symmetric for
any $r$.

\bigskip
\noindent
True. If $r=p$ then ${\bf A}^{-{}}={\bf A}^{-1}$.
Since $\bf A$ is symmetric we have
$${\bf A}^{-1}=({\bf A}^T)^{-1}=({\bf A}^{-1})^T$$
therefore ${\bf A}^{-1}$ is symmetric as well.
If $r<p$ then $\bf A$ is singular.
Then by Theorem 8.6 (a) on p. 54
a symmetric generalized inverse always exists.

\bigskip
\noindent
(c) T or F: ${\bf A}^{-{}}$ must be symmetric when $r=p$.

\bigskip
\noindent
True, by the above first argument for part (b).

\bigskip
\noindent
(d) T or F: $({\bf A}^{-{}})^T=({\bf A}^T)^{-{}}$.

\bigskip
\noindent
True. We have
$${\bf A}^T({\bf A}^{-{}})^T{\bf A}^T
=({\bf AA}^{-{}}{\bf A})^T={\bf A}^T$$
Hence $({\bf A}^{-{}})^T$ is a generalized inverse of ${\bf A}^T$.

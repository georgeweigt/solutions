\section{Section 25.8, problem 15}
Let $\bf X$ be an $n\times p$ matrix with rank $r$ and let
$\bf\Omega$ be an $n\times n$ positive definite matrix.

\bigskip
\noindent
(a) Prove that ${\bf\Omega}^{-1}$ is positive definite.

\bigskip
\noindent
Since $\bf\Omega$ is positive definite,
none of its eigenvalues are zero.
Therefore ${\bf\Omega}^{-1}$ exists.
For every eigenvalue $\lambda$ of $\bf\Omega$ and its associated eigenvector $\bf x$
we have
$${\bf\Omega}{\bf x}=\lambda{\bf x}
\quad\Longrightarrow\quad
\lambda^{-1}{\bf x}={\bf\Omega}^{-1}{\bf x}
$$
Hence $\lambda^{-1}$ is an eigenvalue of ${\bf\Omega}^{-1}$.
Since $\lambda>0$ implies $\lambda^{-1}>0$, all eigenvalues
of ${\bf\Omega}^{-1}$ are greater than zero.
Hence ${\bf\Omega}^{-1}$ is positive definite.

\bigskip
\noindent
(b) Verify that
$\mathop{\rm Dim}[{\cal R}({\bf X})\cap
{\cal N}({\bf X}'{\bf\Omega}^{-1})]=0$.

\bigskip
\noindent
Let $\bf v$ be a vector such that ${\bf v}\in{\cal R}({\bf X})$ and
${\bf v}\in{\cal N}({\bf X}'{\bf\Omega}^{-1})$.
Then ${\bf v}={\bf Xw}$ for some $\bf w$
and
$${\bf X}'{\bf\Omega}^{-1}{\bf v}
={\bf X}'{\bf\Omega}^{-1}{\bf Xw}={\bf 0}$$
We can multiply both sides of the above equation by ${\bf w}'$
and obtain
$${\bf w}'{\bf X}'{\bf\Omega}^{-1}{\bf Xw}=0$$
Since ${\bf\Omega}^{-1}>0$, by Theorem 9.8 we can factor it and
obtain
$${\bf w}'{\bf X}'{\bf\Omega}^{-1/2}{\bf\Omega}^{-1/2}{\bf Xw}
=({\bf\Omega}^{-1/2}{\bf Xw})'({\bf\Omega}^{-1/2}{\bf Xw)}=0$$
where ${\bf\Omega}^{-1/2}{\bf\Omega}^{-1/2}={\bf\Omega}^{-1}$,
${\bf\Omega}^{-1/2}$ is symmetric and ${\bf\Omega}^{-1/2}$ is
nonsingular.
Then by Theorem 8.14 on p. 59 we have
$${\bf\Omega}^{-1/2}{\bf Xw}=\bf0$$
Since ${\bf\Omega}^{-1/2}$ is nonsingular
we must have ${\bf Xw}={\bf 0}$ hence ${\bf v}=\bf0$.
Therefore
$$\mathop{\rm Dim}[{\cal R}({\bf X})\cap
{\cal N}({\bf X}'{\bf\Omega}^{-1})]=\mathop{\rm Dim}(\{{\bf0}\})=0$$

\bigskip
\noindent
(c) Verify that
${\cal R}({\bf X})+{\cal N}({\bf X}'{\bf\Omega}^{-1})=\mathbb{R}^n$.

\bigskip
\noindent
By definition of column space and null space we have
$${\cal R}({\bf X})+{\cal N}({\bf X}'{\bf\Omega}^{-1})\subseteq
\mathbb{R}^n$$
Since
${\bf\Omega}^{-1}$ has full row rank we have
$\mathop{\rm Dim}[{\cal N}({\bf X}'{\bf\Omega}^{-1})]
=\mathop{\rm Dim}[{\cal N}({\bf X}')]$.
Then by Theorem 11.6 (c) on p. 83 we have
$$\mathop{\rm Dim}[{\cal R}({\bf X})]
+\mathop{\rm Dim}[{\cal N}({\bf X}'{\bf\Omega}^{-1})]=n$$
We have already established that
$\mathop{\rm Dim}[{\cal R}({\bf X})\cap
{\cal N}({\bf X}'{\bf\Omega}^{-1})]=0$.
Therefore by Theorem 11.13 on p. 86 we have
$$
\mathop{\rm Dim}[{\cal R}({\bf X})+{\cal N}({\bf X}'{\bf\Omega}^{-1})]
=\mathop{\rm Dim}[{\cal R}({\bf X})]
+\mathop{\rm Dim}[{\cal N}({\bf X}'{\bf\Omega}^{-1})]=n
$$
At this point
we have established a subset relation and shown equal dimension
with $\mathbb{R}^n$.
Therefore, by method 3 on p. 82 we conclude that
$${\cal R}({\bf X})+{\cal N}({\bf X}'{\bf\Omega}^{-1})=\mathbb{R}^n$$




\section*{26.5.4}
Let $\bf Q$ be an $n\times n$ nonsingular matrix and let $\bf A$
be an $n\times n$ matrix.
Verify that if $\lambda$ is an eigenvalue of $\bf A$,
then $\lambda$ is an eigenvalue of ${\bf Q}^{-1}\bf AQ$.

\bigskip
\noindent
Since $\lambda$ is an eigenvalue of $\bf A$ we have
${\bf A}x=\lambda x$ where $x$ is the associated eigenvector.
It follows that
$${\bf Q}^{-1}{\bf A}({\bf QQ}^{-1})x{\bf Q}
={\bf Q}^{-1}\lambda x\bf Q$$
The above equation can be rewritten as
$$({\bf Q}^{-1}{\bf A}{\bf Q})({\bf Q}^{-1}x{\bf Q})
=\lambda({\bf Q}^{-1}x\bf Q)$$
Hence $\lambda$ is an eigenvalue of ${\bf Q}^{-1}\bf AQ$
and ${\bf Q}^{-1}x\bf Q$ is its eigenvector.

\bigskip
\noindent
The book uses the following proof in Theorem 9.4.
\begin{eqnarray*}
|{\bf A}-\lambda{\bf I}|=0
&\Longrightarrow&|{\bf Q}^{-1}|
\times|{\bf A}-\lambda{\bf I}|
\times|{\bf Q}|=0\\
&\Longrightarrow&
|{\bf Q}^{-1}({\bf A}-\lambda{\bf I}){\bf Q}|=0\\
&\Longrightarrow&
|{\bf Q}^{-1}{\bf AQ}-\lambda{\bf I}|=0
\end{eqnarray*}
Therefore $\lambda$ is an eigenvalue of ${\bf Q}^{-1}{\bf AQ}$.

\section{Section 25.13, problem 4}
(a) Give a basis set of estimable functions.

\bigskip
\noindent
By the SAS output, $L_1\beta_0+L_2\beta_1+2L_2\beta_3+L_4\beta_4$
is an estimable function.
Let $\bm\ell^T=\begin{pmatrix} L_1 & L_2 & L_4\end{pmatrix}$ and let
\[
\mathbf K^T=
\begin{pmatrix}
1 & 0 & 0 & 0\\
0 & 1 & 2 & 0\\
0 & 0 & 0 & 1
\end{pmatrix}
\]
Then
%\[
%\bm\ell^T\mathbf K^T=
%\begin{pmatrix}
%L_1 & L_2 & 2L_2 & L_4
%\end{pmatrix}
%\]
%and
\[
\bm\ell^T\mathbf K^T\bm\beta=L_1\beta_0+L_2\beta_1+2L_2\beta_3+L_4\beta_4
\]
Hence a basis set of estimable functions is
\[
\bm\psi=\mathbf K^T\bm\beta=
\begin{pmatrix}
\beta_0\\
\beta_1+2\beta_2\\
\beta_3
\end{pmatrix}
\]

\bigskip
\noindent
(b) The investigator wants to construct a confidence interval for
$\psi=\beta_2-\beta_3$.
Verify or refute the claim that $\psi$ is estimable.

\bigskip
\noindent
$\psi$ is not estimable because there is no linear combination
of the basis functions that yields $\beta_2-\beta_3$, i.e., there
is no solution to
\[
a\beta_0+b(\beta_1+2\beta_2)+c\beta_3=\beta_2-\beta_3
\]
Setting $a=0$ and $c=-1$ and solving for $b$ we obtain
\[
b={\beta_2\over\beta_1+2\beta_2}
\]
Since $\beta_1$ and $\beta_2$ are unknown, there is no solution.

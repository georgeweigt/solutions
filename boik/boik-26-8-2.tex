\section{Section 25.8, problem 2}
Consider the vector space
$$
V=\left\{
{\bf x}; {\bf x}=\left(\begin{matrix}
a\cr
a+b\cr
a+b\cr
-b
\end{matrix}\right),\,
-\infty<a<\infty,\,
-\infty<b<\infty
\right\}
$$
Determine which of the following sets of vectors are spanning sets
and which are basis sets.

\bigskip
\noindent
(a) ${\bf v}_1=\left(\begin{matrix}1\cr0\cr0\cr1\end{matrix}\right)$ and
${\bf v}_2=\left(\begin{matrix}1\cr2\cr2\cr-1\end{matrix}\right)$

\bigskip
\noindent
The above set of vectors form a basis set.
$$
\alpha{\bf v}_1+\beta{\bf v}_2=\left(\begin{matrix}
\alpha+\beta\cr
2\beta\cr
2\beta\cr
\alpha-\beta
\end{matrix}\right)
$$
Let $a=\alpha+\beta$ and $b=\beta-\alpha$.
Then $a+b=2\beta$ and we obtain the original definition
of the vector space.
In addition, ${\bf v}_1$ and ${\bf v}_2$ are linearly independent.

\bigskip
\noindent
(b) ${\bf v}_1=\left(\begin{matrix}1\cr1\cr0\cr0\end{matrix}\right)$ and
${\bf v}_2=\left(\begin{matrix}0\cr0\cr1\cr-1\end{matrix}\right)$

\bigskip
\noindent
The above set of vectors do not form a spanning set
(${\bf v}_1\not\in V$, ${\bf v}_2\not\in V$)
and hence cannot form a basis
set either.

\bigskip
\noindent
(c) 
${\bf v}_1=\left(\begin{matrix}2\cr1\cr1\cr1\end{matrix}\right)$,
${\bf v}_2=\left(\begin{matrix}3\cr1\cr1\cr2\end{matrix}\right)$, and
${\bf v}_3=\left(\begin{matrix}3\cr2\cr2\cr2\end{matrix}\right)$

\bigskip
\noindent
The above set of vectors do not form a spanning set, ${\bf v}_3\not\in V$.
However, if we restrict the set to just ${\bf v}_1$ and ${\bf v}_2$,
we have a spanning set that is also a basis set.

\bigskip
\noindent
(d)
${\bf v}_1=\left(\begin{matrix}1\cr0\cr0\cr0\end{matrix}\right)$,
${\bf v}_2=\left(\begin{matrix}0\cr1\cr1\cr0\end{matrix}\right)$, and
${\bf v}_3=\left(\begin{matrix}0\cr0\cr0\cr1\end{matrix}\right)$

\bigskip
\noindent
The above set of vectors do not form a spanning set, none are members of $V$.




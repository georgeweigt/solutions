
\section{Section 25.15, problem 1}
Consider a two-way classification with interaction and with
$n_{ij}>0$ observations per cell. Write the model as
\[
y_{ijk}=\mu+\tau_i+\theta_j+\gamma_{ij}+\varepsilon_{ijk}
\]
where $\varepsilon_{ijk}\stackrel{\rm iid}{\sim}
N(0,\sigma^2)$. Denote the $a\times b$ matrix of expected cell means
by $\mathbf M=\{\mu_{ij}\}$, denote the population marginal means
for Factor A (equally weight all cell means) by $\bm\mu_A$ and for
Factor B by $\bm\mu_B$.

\bigskip
\noindent
(a) Define ``main effect contrast of Factor A.''

\bigskip
\noindent
See p. 213. It is a contrast of marginal means of Factor A.

\bigskip
\noindent
(b) Suppose $a=3$ and $b=4$. The researcher wants to compare the
first level of Factor A with the average of the remaining two levels
of Factor A. Give the SAS commands required (i) to estimate and
(ii) to test this main effect contrast.

\bigskip
\noindent
The following SAS command does both.
\begin{verbatim}
estimate 'foo' A 8 -4 -4 A*B 2 2 2 2 -1 -1 -1 -1 -1 -1 -1 -1 /divisor=8
\end{verbatim}

\bigskip
\noindent
(c) Suppose $a=3$ and $b=4$. The researcher wants to test a basis
set of contrasts among the levels of Factor A.
Give the SAS command that will product the $F$ test.

\bigskip
\noindent
The CONTRAST statement produces an $F$ test.
\begin{verbatim}
proc glm data = foo;
class A B;
model y = A|B;
contrast 'H0' intercept 1 A 1 0 0 B .25 .25 .25 .25 A*B .25 .25 .25 .25 0 0 0 0 0 0 0 0,
              intercept 1 A 0 1 0 B .25 .25 .25 .25 A*B 0 0 0 0 .25 .25 .25 .25 0 0 0 0,
              intercept 1 A 0 0 1 B .25 .25 .25 .25 A*B 0 0 0 0 0 0 0 0 .25 .25 .25 .25;
run;
\end{verbatim}

\bigskip
\noindent
(d) Define ``simple main effect contrast of Factor A.''

\bigskip
\noindent
See p.\ 215. It is a contrast of cell means at a
fixed level of Factor B. For example, $\mathbf M_{11}-\mathbf M_{21}$.

\bigskip
\noindent
(e) Suppose $a=3$ and $b=4$. The researcher wants to compare the first
level of Factor A with the average of the remaining two levels of
Factor A at level 2 of Factor B. Give the SAS commands required
(i) to estimate and (ii) to test this simple main effect contrast.

\begin{align*}
\mu_{12}-{\mu_{22}+\mu_{32}\over2}
&=\mu+\tau_1+\theta_2+\gamma_{12}-
{\mu+\tau_2+\theta_2+\gamma_{22}+\mu+\tau_3+\theta_2+\gamma_{32}
\over2}\\
&=\tau_1-{1\over2}\tau_2-{1\over2}\tau_3+\gamma_{12}
-{1\over2}\gamma_{22}-{1\over2}\gamma_{32}
\end{align*}

\begin{verbatim}
estimate 'foo' A 2 -1 -1 A*B 0 2 0 0 0 -1 0 0 0 -1 0 0 /divisor=2;
\end{verbatim}

\bigskip
\noindent
(f) Define ``interaction contrast.''

\bigskip
\noindent
It is a contrast of cell means in which all the row coefficients
sum to zero and all the column coefficients sum to zero.

\bigskip
\noindent
(g) Define ``product interaction contrast.''



 

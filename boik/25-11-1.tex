
\section{Section 25.11, problem 1}
Let $\mathbf y$ be a random $n$-vector and let $\mathbf z$
be a random $m$-vector. Suppose that
\[
\begin{pmatrix}
\mathbf y\cr
\mathbf z
\end{pmatrix}\sim\left[
\begin{pmatrix}
\bm\mu_y\cr
\bm\mu_z
\end{pmatrix},\begin{pmatrix}
\bm\Sigma_{yy},\bm\Sigma_{yz}\cr
\bm\Sigma_{zy},\bm\Sigma_{zz}
\end{pmatrix}
\right]
\]
Let $\mathbf A$: $n\times n$;
$\mathbf B$: $b\times n$;
and $\mathbf C$: $c\times m$ be matrices of constants.

\bigskip
\noindent
(a) Prove that $\mathop{\rm Cov}(\mathbf{By},\mathbf{Cz})
=\mathbf{B\Sigma}_{yz}\mathbf C^T$.

\bigskip
\noindent
We are given $\mathop{\rm Cov}(\mathbf y,\mathbf z)=\bm\Sigma_{yz}$.
Then $\mathop{\rm Cov}(\mathbf{By},\mathbf{Cz})
=\mathbf{B\Sigma}_{yz}\mathbf C^T$
follows directly from Theorem 14.1 (c) on p.\ 134.

\bigskip
\noindent
(b) Prove that $\mathop{\rm Var}(\mathbf{By})
=\mathbf{B\Sigma}_{yy}\mathbf B^T$.

\bigskip
\noindent
We are given $\mathop{\rm Var}(\mathbf y)=\bm\Sigma_{yy}$.
Then $\mathop{\rm Var}(\mathbf{By})
=\mathbf{B\Sigma}_{yy}\mathbf B^T$
follows directly from Theorem 14.1 (d) on p.\ 134.

\bigskip
\noindent
(c) Prove that
$\mathrm E(\mathbf y^T\mathbf{Ay})
=\mathop{\rm trace}(\mathbf{A\Sigma}_{yy})
+\bm\mu_y^T\mathbf A\bm\mu_y$
whether or not $\mathbf A$ is symmetric.

\bigskip
\noindent
By properties of the trace operator we have
\begin{eqnarray*}
\mathop{\rm trace}(\mathbf A\bm\Sigma_{yy})
&=&\mathop{\rm trace}[(\mathbf A\bm\Sigma_{yy})^T]\\
&=&\mathop{\rm trace}(\bm\Sigma_{yy}^T\mathbf A^T)\\
&=&\mathop{\rm trace}(\mathbf A^T\bm\Sigma_{yy}^T)
\end{eqnarray*}
Then by symmetry of $\bm\Sigma_{yy}$ we have
$$
\mathop{\rm trace}(\mathbf A\bm\Sigma_{yy})
=\mathop{\rm trace}(\mathbf A^T\bm\Sigma_{yy})
$$
Therefore
\begin{eqnarray*}
\mathop{\rm trace}(\mathbf{A\Sigma}_{yy})
&=&
{1\over2}\left[
\mathop{\rm trace}(\mathbf{A\Sigma}_{yy})
+
\mathop{\rm trace}(\mathbf A^T\bm\Sigma_{yy})
\right]\\
&=&
\mathop{\rm trace}\left[
{1\over2}(\mathbf A+\mathbf A^T)\bm\Sigma_{yy}
\right]\\
&=&\mathop{\rm trace}(\mathbf A^*\bm\Sigma_{yy})
\end{eqnarray*}
where $\mathbf A^*={1\over2}(\mathbf A+\mathbf A^T)$.
Therefore if $\mathbf A$ is not symmetric then it can be replaced
with $\mathbf A^*$ with no change to the value of the trace.

\bigskip
\noindent
By Theorem 9.5 on p.\ 73 the quadratic forms
$\mathbf y^T\mathbf{Ay}$
and
$\bm\mu_y^T\mathbf A\bm\mu_y$
do not require $\mathbf A$ to be symmetric.
By this and the trace result above,
$\mathrm E(\mathbf y^T\mathbf{Ay})
=\mathop{\rm trace}(\mathbf{A\Sigma}_{yy})
+\bm\mu_y^T\mathbf A\bm\mu_y$
whether or not $\mathbf A$ is symmetric.

\bigskip
\noindent
(d) Verify that the matrix $\mathbf T$ of the quadratic form
$\mathbf y^T\mathbf{Ty}$ can be assumed wlog to be symmetric.

\bigskip
\noindent
The above follows directly from Theorem 9.5 on p.\ 73.




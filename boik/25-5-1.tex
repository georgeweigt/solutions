\section{Section 25.5, problem 1}
(a) Let $\bf A$ be an $n\times n$ symmetric matrix.
Show that ${\bf A}={\bf 0}$ iff all eigenvalues of $\bf A$ are zero.

\bigskip
\noindent
Since $\bf A$ is symmetric, by Theorem 9.3 it can be diagonalized
as ${\bf A}={\bf U\Lambda U}'$ where $\bf\Lambda$ is a diagonal
matrix of eigenvalues and $\bf U$ is an orthogonal matrix.
Therefore ${\bf\Lambda}=\bf0$ implies ${\bf A}=\bf0$.
We can also write ${\bf U}'{\bf AU}=\bf\Lambda$ from which it follows
that ${\bf A}=\bf0$ implies ${\bf\Lambda=0}$.
Therefore ${\bf A}={\bf 0}$ iff all eigenvalues of $\bf A$ are zero.

\bigskip
\noindent
(b) If $\bf A$ is not symmetric, then the above result does not hold.
Compute the eigenvalues of
$${\bf A}=\left(\begin{matrix}
0 & 1\cr
0 & 0
\end{matrix}\right)
$$

\bigskip
\noindent
The eigenvalues are the solutions to the following equation.
$$\left|\begin{matrix}
-\lambda & 1\cr
0 & -\lambda
\end{matrix}\right|=\lambda^2=0
$$
Hence both eigenvalues are zero but ${\bf A}\ne\bf 0$.

\section{Section 25.1, problem 6}
(a) Let $\bf A$ be a $p\times q$ matrix and let $\bf B$ be a
$q\times p$ matrix. Verify that
$\mathop{\rm tr}({\bf AB})=\mathop{\rm tr}({\bf BA})$.

\bigskip
\noindent
The matrix product $\bf AB$ is a square matrix.
Therefore $\mathop{\rm tr}({\bf AB})=\mathop{\rm tr}({\bf BA})$
by Theorem 8.11.

\bigskip
\noindent
Theorem 8.11 (a) is proved in the book, but here is another proof.
We have
\begin{eqnarray*}
\mathop{\rm tr}({\bf AB})&=&\sum_{i=1}^p
\left(\sum_{j=1}^q A_{ij}B_{ji}\right)\\
\mathop{\rm tr}({\bf BA})&=&\sum_{j=1}^q
\left(\sum_{i=1}^p B_{ji}A_{ij}\right)
\end{eqnarray*}
Since scalar sums and products are individually commutative,
we can reorder terms and factors to obtain
$$\mathop{\rm tr}({\bf AB})=\sum_{i=1}^p\sum_{j=1}^q A_{ij}B_{ji}
=\sum_{j=1}^q\sum_{i=1}^p B_{ji}A_{ij}
=\mathop{\rm tr}({\bf BA})$$

\bigskip
\noindent
(b) Let $\bf A$ be a $p\times q$ matrix, let $\bf B$ be a
$q\times s$ matrix, and let $\bf C$ be an $s\times p$ matrix.
Use the previous result to verify that
$\mathop{\rm tr}({\bf ABC})=\mathop{\rm tr}({\bf BCA})
=\mathop{\rm tr}({\bf CAB})$.

\bigskip
\noindent
Let $\bf D=BC$ which is a $q\times p$ matrix.
Then $\bf AD$ is a square $p\times p$ matrix.
By the above result, $\mathop{\rm tr}({\bf AD})=\mathop{\rm tr}({\bf DA})$.
Hence $\mathop{\rm tr}({\bf ABC})=\mathop{\rm tr}({\bf BCA})$.

\bigskip
\noindent
Let $\bf E=CA$ which is an $s\times q$ matrix.
Then $\bf BE$ is a square $q\times q$ matrix.
By the above result, $\mathop{\rm tr}({\bf BE})=\mathop{\rm tr}({\bf EB})$.
Hence $\mathop{\rm tr}({\bf BCA})=\mathop{\rm tr}({\bf CAB})$.

\section{Section 25.9, problem 19}
Let $\bf X$ be an $n\times p$ matrix with rank $r$ and let $\bf G$
be any generalized inverse of $\bf X$.

\bigskip
\noindent
(a) Verify that $\bf XG$, $\bf GX$, ${\bf I}_n-\bf XG$, and
${\bf I}_p-\bf GX$ are projection operators.

$$({\bf XG})^2=({\bf XGX}){\bf G}={\bf XG}$$
$$({\bf GX})^2={\bf G}({\bf XGX})={\bf GX}$$

$$({\bf I}_n-{\bf XG})^2={\bf I}_n-2{\bf XG}-({\bf XG})^2
={\bf I}_n-{\bf XG}$$
$$({\bf I}_p-{\bf GX})^2={\bf I}_p-2{\bf GX}-({\bf GX})^2
={\bf I}_p-{\bf GX}$$

\bigskip
\noindent
(b) Determine the two spaces associated with each of the
projection operators.

\bigskip
\noindent
By Theorems 11.5 and 11.7 we have the following subset relations.
Then the equivalence of vector spaces follows from the equivalence
of matrix rank.
%
%\begin{eqnarray*}
%{\cal R}({\bf XG})\subseteq{\cal R}({\bf X})
%&\Longrightarrow&{\cal R}({\bf XG})={\cal R}({\bf X})\\
%%
%{\cal R}({\bf GX})\subseteq{\cal R}({\bf G})
%&\Longrightarrow&{\cal R}({\bf GX})={\cal R}({\bf G})\\ \\
%%
%{\cal N}({\bf G})\subseteq{\cal N}({\bf XG})
%&\Longrightarrow&{\cal N}({\bf G})={\cal N}({\bf XG})\\
%%
%{\cal N}({\bf X})\subseteq{\cal N}({\bf GX})
%&\Longrightarrow&{\cal N}({\bf X})={\cal N}({\bf GX})
%\end{eqnarray*}
%
By Theorem 11.5 on p. 83 we have
$${\cal R}({\bf XG})\subseteq{\cal R}({\bf X})\quad\hbox{and}\quad
{\cal R}({\bf X})={\cal R}({\bf XGX})\subseteq{\cal R}({\bf XG})$$
Therefore
$${\cal R}({\bf XG})={\cal R}({\bf X})$$
%
By Theorem 11.7 on p. 85 we have
$${\cal N}({\bf GX})\subseteq{\cal N}({\bf XGX})={\cal N}({\bf X})
\quad\hbox{and}\quad
{\cal N}({\bf X})\subseteq{\cal N}({\bf GX})$$
Therefore
$${\cal N}({\bf GX})={\cal N}({\bf X})$$
By Theorem 11.18 we have 
${\cal R}({\bf I}-{\bf P})={\cal N}({\bf P})$
and
${\cal N}({\bf I}-{\bf P})={\cal R}({\bf P})$.
Hence

\begin{center}
\begin{tabular}{|l|l|l|}
\hline
Operator $\bf P$ & onto ${\cal R}({\bf P})$ & along ${\cal N}({\bf P})$\\
\hline
${\bf XG}$ & ${\cal R}({\bf X})$ & ${\cal N}({\bf XG})$\\
${\bf GX}$ & ${\cal R}({\bf GX})$ & ${\cal N}({\bf X})$\\
${\bf I}_n-\bf XG$ & ${\cal N}({\bf XG})$ & ${\cal R}({\bf X})$\\
${\bf I}_p-\bf GX$ & ${\cal N}({\bf X})$ & ${\cal R}({\bf GX})$\\
\hline
\end{tabular}
\end{center}

\bigskip
\noindent
(c) Do these spaces depend on the choice of generalized inverse?

\bigskip
\noindent
Yes.


\section*{26.14.17}
Consider the linear model $\mathbf y\sim N(\mathbf X\bm\beta,
\sigma^2\mathbf I_n)$ where $\mathbf X: N\times(k+1)$ is the
design matrix for the one-way classification model
$y_{ij}=\mu+\tau_i+\varepsilon_{ij}$.
Partition the design matrix as $\mathbf X=
\begin{pmatrix}\mathbf X_1 & \mathbf X_2\end{pmatrix}$
where $\mathbf X_1=\mathbf 1_N$.
Consider testing $H_0:E(\mathbf y)\in\mathcal R(\mathbf X_1)$
against $H_a:E(\mathbf y)\in\mathcal R(\mathbf X)$,
$E(\mathbf y)\not\in\mathcal R(\mathbf X_1)$.

\bigskip
\noindent
(a) Verify that the null and alternative hypotheses are equivalent
to $H_{0'}:\tau_i=\tau_{i'}$ for all $i$, $i'$ versus
$H_{a'}:\tau_i\ne\tau_{i'}$ for some $i$, $i'$.

\bigskip
\noindent
What they mean by $\tau_i=\tau_{i'}$ is
$\tau_1=\tau_2=\cdots=\tau_k$.
We have to show that $H_0$ true $\Leftrightarrow$
$H_{0'}$ true and $H_a$ true $\Leftrightarrow$ $H_{a'}$ true.

\bigskip
\noindent
%$H_0\Rightarrow H_{0'}$.
Let $E(\mathbf y)\in\mathcal R(\mathbf X_1)$.
Then there is an $\alpha$ such that $E(\mathbf y)=\alpha\mathbf1_N$.
It follows that $E(y_{ij})=\mu+\tau_i=\alpha$ for all $i$, $j$.
Therefore $\tau_1=\tau_2=\cdots=\tau_k=\alpha-\mu$.
Hence $H_0\Rightarrow H_{0'}$.

\bigskip
\noindent
Let $\tau_1=\tau_2=\cdots=\tau_k=\theta$.
Then $E(y_{ij})=\mu+\theta$ for all $i$, $j$.
It follows that $E(\mathbf y)=(\mu+\theta)\mathbf 1_N$.
Therefore $E(\mathbf y)\in\mathcal R(\mathbf X_1)$.
Hence $H_{0'}\Rightarrow H_0$.

\bigskip
\noindent
Let $E(\mathbf y)\in\mathcal R(\mathbf X)$,
$E(\mathbf y)\not\in\mathcal R(\mathbf X_1)$.
Then there is a $y_{ij}$ and a $y_{kl}$ such that
$E(y_{ij})\ne E(y_{kl})$.


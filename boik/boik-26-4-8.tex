\section{Section 25.4, problem 8}
Let $\bf A$ be a $p\times q$ matrix with rank $r$.
Let ${\bf P}_1$ be a $p\times p$ permutation matrix defined so
that the first $r$ rows of ${\bf P}_1{\bf A}$ are linearly
independent.
Similarly, let ${\bf P}_2$ be a $q\times q$ permutation matrix defined
so that the first $r$ columns of ${\bf AP}_2$ are linearly
independent.
Partition ${\bf P}_1{\bf AP}_2$ as
$${\bf P}_1{\bf AP}_2=\left(\begin{matrix}
{\bf A}_{11} & {\bf A}_{12}\cr
{\bf A}_{21} & {\bf A}_{22}
\end{matrix}\right)
$$
where ${\bf A}_{11}$ is $r\times r$.

\bigskip
\noindent
(a) Prove that ${\bf A}_{11}$ has full rank.

\bigskip
\noindent
Since only the first $r$ rows of ${\bf P}_1{\bf A}{\bf P}_2$
are linearly independent, the remaining $p-r$ rows are linearly dependent.
Therefore there is a matrix $\bf B$ such that
$$\left(\begin{matrix}
{\bf I}_r\cr
{\bf B}
\end{matrix}\right){\bf A}_{11}=\left(\begin{matrix}
{\bf A}_{11}\cr
{\bf A}_{21}
\end{matrix}\right)
$$
Since only the first $r$ columns of ${\bf P}_1{\bf A}{\bf P}_2$
are linearly independent, the remaining $q-r$ columns are
linearly dependent.
Therefore there is a matrix $\bf C$ such that
$$\left(\begin{matrix}
{\bf A}_{11}\cr
{\bf A}_{21}
\end{matrix}\right)
\left(\begin{matrix}
{\bf I}_r & {\bf C}
\end{matrix}\right)=\left(\begin{matrix}
{\bf A}_{11} & {\bf A}_{12}\cr
{\bf A}_{21} & {\bf A}_{22}
\end{matrix}\right)
$$
Hence
$$
\left(\begin{matrix}
{\bf I}_r\cr
{\bf B}
\end{matrix}\right){\bf A}_{11}
\left(\begin{matrix}
{\bf I}_r & {\bf C}
\end{matrix}\right)=\left(\begin{matrix}
{\bf A}_{11} & {\bf A}_{12}\cr
{\bf A}_{21} & {\bf A}_{22}
\end{matrix}\right)={\bf P}_1{\bf A}{\bf P}_2
$$
Since the rank of ${\bf P}_1{\bf A}{\bf P}_2$ is $r$,
by Theorem 8.18, the rank of ${\bf A}_{11}$ cannot be less than
$r$. Therefore ${\bf A}_{11}$ has full rank.

\bigskip
\noindent
(b) Prove that
$${\bf G}={\bf P}_2
\left(\begin{matrix}
{\bf A}_{11}^{-1} & {\bf 0}\cr
{\bf 0} & {\bf 0}
\end{matrix}\right)
{\bf P}_1
$$
is a generalized inverse of $\bf A$.

\bigskip
\noindent
Note that from part (a) we have
$${\bf P}_1{\bf A}{\bf P}_2
=
\left(\begin{matrix}
{\bf I}_r\cr
{\bf B}
\end{matrix}\right){\bf A}_{11}
\left(\begin{matrix}
{\bf I}_r & {\bf C}
\end{matrix}\right)
=
\left(\begin{matrix}
{\bf A}_{11} & {\bf A}_{11}{\bf C}\cr
{\bf B}{\bf A}_{11} & {\bf B}{\bf A}_{11}{\bf C}
\end{matrix}\right)
$$
By matrix multiplication we have
\begin{eqnarray*}
{\bf AGA}
&=&
({\bf P}_1'{\bf P}_1){\bf AGA}({\bf P}_2{\bf P}_2')\\
%&=&
%{\bf P}_1{\bf A}{\bf G}{\bf A}{\bf P}_2\\
&=&
{\bf P}_1'
({\bf P}_1{\bf A}{\bf P}_2)
\left(\begin{matrix}
{\bf A}_{11}^{-1} & {\bf 0}\cr
{\bf 0} & {\bf 0}
\end{matrix}\right)
({\bf P}_1{\bf A}{\bf P}_2)
{\bf P}_2'\\
&=&
{\bf P}_1'
\left(\begin{matrix}
{\bf A}_{11} & {\bf A}_{11}{\bf C}\cr
{\bf B}{\bf A}_{11} & {\bf B}{\bf A}_{11}{\bf C}
\end{matrix}\right)
%
\left(\begin{matrix}
{\bf A}_{11}^{-1} & {\bf 0}\cr
{\bf 0} & {\bf 0}
\end{matrix}\right)
%
\left(\begin{matrix}
{\bf A}_{11} & {\bf A}_{11}{\bf C}\cr
{\bf B}{\bf A}_{11} & {\bf B}{\bf A}_{11}{\bf C}
\end{matrix}\right)
{\bf P}_2'\\
&=&
{\bf P}_1'
\left(\begin{matrix}
{\bf A}_{11} & {\bf A}_{11}{\bf C}\cr
{\bf B}{\bf A}_{11} & {\bf B}{\bf A}_{11}{\bf C}
\end{matrix}\right)
{\bf P}_2'\\
&=&
{\bf P}_1'({\bf P}_1{\bf A}{\bf P}_2){\bf P}_2'\\
&=&
{\bf A}
\end{eqnarray*}
Hence
$${\bf A}{\bf G}{\bf A}={\bf A}$$
Therefore $\bf G$ is a generalized inverse of $\bf A$.

\bigskip
\noindent
(c) Use the above result to find a generalized inverse of
$${\bf A}=\left(\begin{matrix}
1 & 2 & 3 & 1\cr
2 & 4 & 6 & 1\cr
3 & 6 & 9 & 1
\end{matrix}\right)
$$
The third row is two times the second row minus the first row.
Hence keep the first two rows.
$${\bf P}_1=\left(\begin{matrix}
1 & 0 & 0\cr
0 & 1 & 0\cr
0 & 0 & 1
\end{matrix}\right)
$$
Columns two and three are multiples of column one.
Hence use columns one and four.
$${\bf P}_2=\left(\begin{matrix}
1 & 0 & 0 & 0\cr
0 & 0 & 1 & 0\cr
0 & 0 & 0 & 1\cr
0 & 1 & 0 & 0
\end{matrix}\right)
$$
Hence
$${\bf A}_{11}=\left(\begin{matrix}
1 & 1\cr
2 & 1
\end{matrix}\right),\quad{\bf A}_{11}^{-1}=
\left(\begin{matrix}
-1 & 1\cr
2 & -1
\end{matrix}\right)
$$
Therefore
$${\bf G}=
{\bf P}_2\left(\begin{matrix}
-1 & 1 & 0 \cr
2 & -1 & 0\cr
0 & 0 & 0\cr
0 & 0 & 0
\end{matrix}\right){\bf P}_1
=\left(\begin{matrix}
-1 & 1 & 0\cr
0 & 0 & 0\cr
0 & 0 & 0\cr
2 & -1 & 0
\end{matrix}\right)
$$

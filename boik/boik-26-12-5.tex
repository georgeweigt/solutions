%\section*{Lesson learned from 26.12.5}
%Theorem 11.29 on p.\ 101 is indispensable for doing
%problems involving $\mathbf P=\mathbf X(\mathbf X^T
%\bm\Omega^{-1}\mathbf X)^{-{}}\mathbf X^T\bm\Omega^{-1}$.
%The theorem says
%\[\mathbf P\bm\Omega=\bm\Omega\mathbf P^T\]
%and
%\[\bm\Omega^{-1}\mathbf P=\mathbf P^T\bm\Omega^{-1}\]

\section*{26.12.5}
Consider the linear model
\[
\mathbf y=\mathbf X\bm\beta+\bm\varepsilon
\]
where $\mathbf X$ is $n\times p$ with rank $r$.
Let $\bm{\tilde\beta}$ be any solution to the normal equations
$\mathbf X^T\bm\Omega^{-1}\mathbf X\bm\beta
=\mathbf X^T\bm\Omega^{-1}\mathbf y$ where $\bm\Omega$
is positive definite.
Define $\mathbf{\hat y}$ and $\mathbf e$ as
$\mathbf{\hat y}=\mathbf X\bm{\tilde\beta}$ and
$\mathbf e=\mathbf y-\mathbf{\hat y}$.

\bigskip
\noindent
(a) What vector spaces do $\mathbf y$, $\mathbf{\hat y}$,
and $\mathbf e$ live in?

\bigskip
\noindent
$\mathbf y\in{\cal R}(\bm\Omega)$,
$\mathbf{\hat y}\in{\cal R}(\mathbf X)$,
$\mathbf e\in{\cal N}(\mathbf X^T\bm\Omega^{-1})$.

\bigskip
\noindent
(b) Verify that $\mathbf{\hat y}\perp\mathbf e$ in the metric
$\bm\Omega^{-1}$.
That is, verify that $\mathbf{\hat y}^T\bm\Omega^{-1}\mathbf e=0$.

\bigskip
\noindent
By Theorem 11.29 on p.\ 101 we have
$\mathbf P^T\bm\Omega^{-1}=\bm\Omega^{-1}\mathbf P$.
Therefore
\begin{align*}
\mathbf{\hat y}^T\bm\Omega^{-1}\mathbf e
&=(\mathbf{Py})^T\bm\Omega^{-1}(\mathbf I-\mathbf P)\mathbf y\\
&=\mathbf y^T\mathbf P^T\bm\Omega^{-1}(\mathbf I-\mathbf P)
\mathbf y\\
&=\mathbf y^T\bm\Omega^{-1}\mathbf P(\mathbf I-\mathbf P)\mathbf y\\
&=0
\end{align*}

\bigskip
\noindent
(c) Find $\mathrm E(\mathbf{\hat y})$ and
$\mathop{\rm Var}(\mathbf{\hat y})$.

\bigskip
\noindent
Let $\mathbf P=\mathbf X(\mathbf X^T\bm\Omega^{-1}\mathbf X)\mathbf X^T
\bm\Omega^{-1}$.
Then $\mathbf{PX}=\mathbf X$ and we have
%
\begin{align*}
\mathrm E(\mathbf{\hat y})
&=\mathrm E(\mathbf{Py})\\
&=\mathbf P\mathrm E(\mathbf y)\\
&=\mathbf{PX}\bm\beta\\
&=\mathbf X\bm\beta\\
\\
\mathop{\rm Var}(\mathbf{\hat y})
&=\mathop{\rm Var}(\mathbf{Py})\\
&=\mathbf P\mathop{\rm Var}(\mathbf y)\mathbf P^T\\
&=\mathbf P\sigma^2\bm\Omega\mathbf P^T\\
&=\sigma^2\mathbf X(\mathbf X^T\bm\Omega^{-1}\mathbf X)^{-{}}
\mathbf X^T\bm\Omega^{-1}\bm\Omega\bm\Omega^{-1}
\mathbf X[(\mathbf X^T\bm\Omega^{-1}\mathbf X)^{-{}}]^T
\mathbf X^T\\
&=\sigma^2\mathbf X(\mathbf X^T\bm\Omega^{-1}\mathbf X)^{-{}}
\mathbf X^T\bm\Omega^{-1}\mathbf X[(\mathbf X^T\bm\Omega^{-1}
\mathbf X)^{-{}}]^T\mathbf X^T\\
&=\sigma^2\mathbf X[(\mathbf X^T\bm\Omega^{-1}
\mathbf X)^{-{}}]^T\mathbf X^T\\
&=\sigma^2\bm\Omega\mathbf P^T\\
&=\sigma^2\mathbf P\bm\Omega
\end{align*}

\bigskip
\noindent
(d) Find $\mathrm E(\mathbf e)$ and $\mathop{\rm Var}(\mathbf e)$.

\bigskip
\noindent
Use $\mathbf P\bm\Omega=\bm\Omega\mathbf P^T$ from Theorem 11.29
on p.\ 101.
\begin{align*}
\mathrm E(\mathbf e)
&=\mathrm E(\mathbf y-\mathbf{\hat y})\\
&=\mathrm E(\mathbf y)-\mathrm E(\mathbf{\hat y})\\
&=\mathbf X\bm\beta-\mathbf X\bm\beta\\
&=0\\
\\
\mathop{\mathrm Var}(\mathbf e)
&=\mathop{\rm Var}[(\mathbf I-\mathbf P)\mathbf y]\\
&=(\mathbf I-\mathbf P)\mathop{\rm Var}(\mathbf y)
(\mathbf I-\mathbf P)^T\\
&=(\mathbf I-\mathbf P)\sigma^2\bm\Omega
(\mathbf I-\mathbf P^T)\\
&=\sigma^2(\mathbf I-\mathbf P)(\bm\Omega-\bm\Omega\mathbf P^T)\\
&=\sigma^2(\bm\Omega-\bm\Omega\mathbf P^T
-\mathbf P\bm\Omega+\mathbf P\bm\Omega\mathbf P^T)\\
&=\sigma^2(\bm\Omega-\mathbf P\bm\Omega
-\mathbf P\bm\Omega+\mathbf P\bm\Omega)\\
&=\sigma^2(\bm\Omega-\mathbf P\bm\Omega)\\
&=\sigma^2(\mathbf I-\mathbf P)\bm\Omega
\end{align*}

\bigskip
\noindent
(e) Find $\mathop{\rm Cov}(\mathbf{\hat y},\bm\Omega^{-1}\mathbf e)$.

\bigskip
\noindent
\begin{align*}
\mathop{\rm Cov}(\mathbf{\hat y},\bm\Omega^{-1}\mathbf e)
&=\mathop{\rm Cov}(\mathbf{Py},\bm\Omega^{-1}(\mathbf I
-\mathbf P)\mathbf y\\
&=\mathbf P\mathop{\rm Cov}(\mathbf y,\mathbf y)
[\bm\Omega^{-1}(\mathbf I-\mathbf P)]^T\\
&=\mathbf P\sigma^2\bm\Omega(\mathbf I-\mathbf P^T)\bm\Omega^{-1}\\
&=\sigma^2\mathbf P\bm\Omega(\bm\Omega^{-1}-\mathbf P^T\bm\Omega^{-1})\\
&=\sigma^2\mathbf P\bm\Omega(\bm\Omega^{-1}-\bm\Omega^{-1}\mathbf P)\\
&=\sigma^2\mathbf P(\mathbf I-\mathbf P)\\
&=0
\end{align*}


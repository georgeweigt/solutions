\section*{26.12.3}
Let $\mathbf X$ be the design matrix that corresponds to a
one-way classification with $k$ levels.
That is, $\mathbf X=\begin{pmatrix}\mathbf 1_N
&\oplus_{j=1}^k\mathbf 1_{n_j}\end{pmatrix}$, where $n_j$ is the sample size
in the $j^{\rm th}$ treatment for $j=1,\ldots,k$
and $N=\sum_{j=1}^kn_j$.

\bigskip
\noindent
(a) Verify that
$$\mathbf X^T\mathbf X=
\begin{pmatrix}
N & \mathbf n^T\cr
\mathbf n & \mathbf D
\end{pmatrix},\,
\hbox{where
$\mathbf n=\begin{pmatrix}
n_1\cr
n_2\cr
\vdots\cr
n_k
\end{pmatrix}
$ and
$\mathbf D=\begin{pmatrix}
n_1 & 0 & \cdots & 0\cr
0 & n_2 & \cdots & 0\cr
\vdots & \vdots & \ddots & \vdots\cr
0 & 0 & \cdots & n_k
\end{pmatrix}
$}
$$

\bigskip
\noindent
The trick is to split up the leading column of ones in $\mathbf X$
and then do a block multiply.
We have
\begin{align*}
\mathbf X^T\mathbf X
%&= \begin{pmatrix}
%\mathbf 1_N^T\\
%\oplus_{j=1}^k\mathbf 1_{n_j}^T
%\end{pmatrix}
%\begin{pmatrix}
%\mathbf 1_N & \oplus_{j=1}^k\mathbf 1_{n_j}
%\end{pmatrix}\\
&=\left(\begin{array}{llll}
\mathbf 1_{n_1}^T & \mathbf 1_{n_2}^T & \cdots & \mathbf 1_{n_k}^T\\
\mathbf 1_{n_1}^T & \mathbf 0 & \cdots & \mathbf 0\\
\mathbf 0 & \mathbf 1_{n_2}^T & \cdots & \mathbf 0\\
\vdots & \vdots & \ddots & \vdots\\
\mathbf 0 & \mathbf 0 & \cdots & \mathbf 1_{n_k}^T
\end{array}\right)
\left(\begin{array}{lllll}
\mathbf 1_{n_1} & \mathbf 1_{n_1} & \mathbf 0 & \cdots & \mathbf 0\\
\mathbf 1_{n_2} & \mathbf 0 & \mathbf 1_{n_2} & \cdots & \mathbf 0\\
\vdots & \vdots & \vdots & \ddots & \vdots\\
\mathbf 1_{n_k} & \mathbf 0 & \mathbf 0 & \cdots & \mathbf 1_{n_k}
\end{array}\right)\\
&=\begin{pmatrix}
N & n_1 & n_2 & \cdots & n_k\\
n_1 & n_1 & 0 & \cdots & 0\\
n_2 & 0 & n_2 & \cdots & 0\\
\vdots & \vdots & \vdots & \ddots & \vdots\\
n_k & 0 & 0 & \cdots & n_k
\end{pmatrix}\\
&=\begin{pmatrix}
N & \mathbf n^T\cr
\mathbf n & \mathbf D
\end{pmatrix}
%
\end{align*}

\bigskip
\noindent
(b) Verify that $\mathop{\rm rank}(\mathbf X^T\mathbf X)=k$.

\bigskip
\noindent
By inspection there are $k$ linearly independent columns in $\mathbf X$, hence
$\mathop{\rm rank}(\mathbf X)=k$.
It follows that
$\mathop{\rm rank}(\mathbf X^T\mathbf X)
=\mathop{\rm rank}(\mathbf X)=k$.

\bigskip
\noindent
(c) Partition $\mathbf X^T\mathbf X$ as
$$\mathbf X^T\mathbf X=\begin{pmatrix}
\mathbf A_{11} & \mathbf A_{12}\\
\mathbf A_{21} & \mathbf A_{22}
\end{pmatrix}
$$
where $\mathbf A_{11}$ is $k\times k$.
Verify that $\mathbf A_{11}$ is nonsingular.
Hint: use results in \#7 on page 64 of the notes to
show that the determinant of $\mathbf A_{11}$ is non-zero
(i.e., partition $\mathbf A_{11}$).

\bigskip
\noindent
From part (a) we have
\begin{equation*}
\mathbf A_{11}=\begin{pmatrix}
N & n_1 & n_2 & \cdots & n_{k-1}\\
n_1 & n_1 & 0 & \cdots & 0\\
n_2 & 0 & n_2 & \cdots & 0\\
\vdots & \vdots & \vdots & \ddots & \vdots\\
n_{k-1} & 0 & 0 & \cdots & n_{k-1}
\end{pmatrix}
\end{equation*}
Partition $\mathbf A_{11}$ as
\begin{equation*}
\mathbf A_{11}
\begin{pmatrix}
N & \mathbf B_{12}\\
\mathbf B_{21} & \mathbf B_{22}
\end{pmatrix}
\end{equation*}
Then by the result on p.\ 64 we have
\begin{align*}
|\mathbf A_{11}|
&=|\mathbf B_{22}|
\left(N-\mathbf B_{12}\mathbf B_{22}^{-1}\mathbf B_{21}\right)\\
&=\left(\prod_{j=1}^{k-1}n_j\right)
\left(N-\sum_{j=1}^{k-1}n_j\right)\\
&=\left(\prod_{j=1}^{k-1}n_j\right)n_k\\
&=\prod_{j=1}^kn_j
\end{align*}
Therefore $\mathbf A_{11}$ is nonsingular.

\bigskip
\noindent
(d) Using R the result is
\[
\bm{\tilde\beta}_R=\left(\begin{array}{r}
28.80\\
-16.63\\
-11.37\\
-0.80\\
0.00
\end{array}\right)
\]
which matches the SAS solution because
\begin{align*}
\bar y_1&=12.17\\
\bar y_2&=17.43\\
\bar y_3&=28.00\\
\bar y_4&=28.80
\end{align*}
Hence
\[
\bm{\tilde\beta}_{SAS}
=\begin{pmatrix}
\bar y_4\\
\bar y_1-\bar y_4\\
\bar y_2-\bar y_4\\
\bar y_3-\bar y_4\\
0
\end{pmatrix}
=\begin{pmatrix}
28.80\\
12.17-28.80\\
17.43-28.80\\
28.00-28.80\\
0
\end{pmatrix}
=
\left(\begin{array}{r}
28.80\\
-16.63\\
-11.37\\
-0.80\\
0.00
\end{array}\right)
=\bm{\tilde\beta}_R
\]


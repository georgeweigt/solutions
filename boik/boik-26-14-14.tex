\section*{26.14.14}
(a) Verify that $\mu_i-\overline{\mu}=\tau_i-\overline{\tau}_i$
where $\overline{\mu}=N^{-1}\sum_{i=1}^kn_i\mu_i
=k^{-1}\sum_{i=1}^k\mu_i$.

\bigskip
\noindent
The planned study has $n$ observations
per group, hence $n_i=n$ for all $i$.
Recall that $\mu_i=\mu_0-\tau_i$ hence $\tau_i=\mu_i-\mu_0$.
It follows that
\begin{align*}
\overline{\tau}&={1\over N}\sum_{i=1}^kn(\mu_i-\mu_0)\\
&={1\over N}\sum_{i=1}^kn\mu_i-{1\over N}\sum_{i=1}^kn\mu_0\\
&=\overline{\mu}-\mu_0
\end{align*}
Therefore
\[
\tau_i-\overline{\tau}=(\mu_i-\mu_0)-(\overline{\mu}-\mu_0)
=\mu_i-\overline{\mu}
\]

\bigskip
\noindent
(b) Verify that the noncentrality parameter for the numerator of the
test statistic simplifies to
\[
\lambda
=
{\displaystyle\sum_{i=1}^kn_i(\tau_i-\overline{\tau})^2\over2\sigma^2}
=
{\displaystyle\sum_{i=1}^kn(\mu_i-\overline{\mu})^2\over2\sigma^2}
=
{\displaystyle n\sum_{i=1}^k(\mu_i-\overline{\mu})^2\over2\sigma^2}
\]

\bigskip
\noindent
Note that
\[
\mathbf X\bm\beta=
\begin{pmatrix}
\mathbf 1_{n_1} & \mathbf1_{n_1} & \mathbf 0 & \mathbf 0
& \cdots & \mathbf 0\\
\mathbf 1_{n_2} & \mathbf 0 & \mathbf1_{n_2} & \mathbf0 & \cdots & \mathbf0\\
\vdots\\
\mathbf1_{n_k} & \mathbf0 & \mathbf0 & \mathbf0 & \cdots
& \mathbf1_{n_k}
\end{pmatrix}
\begin{pmatrix}
\mu\\
\tau_1\\
\tau_2\\
\vdots\\
\tau_k
\end{pmatrix}
=
\begin{pmatrix}
\mu\mathbf1_{n_1}+\tau_1\mathbf1_{n_1}\\
\mu\mathbf1_{n_2}+\tau_2\mathbf1_{n_2}\\
\vdots\\
\mu\mathbf1_{n_k}+\tau_k\mathbf1_{n_k}
\end{pmatrix}
\]
and
\begin{align*}
N^{-1}\mathbf1_N^T\mathbf X\bm\beta
&=
N^{-1}
\begin{pmatrix}
\mathbf1_{n_1}^T&\mathbf1_{n_2}^T
&\cdots&\mathbf1_{n_k}^T
\end{pmatrix}\mathbf X\bm\beta\\
&=N^{-1}\sum_{i=1}^k(n_i\mu+n_i\tau_i)\\
&=\mu+\overline{\tau}
\end{align*}
%
From Exercise \#8 and Theorem 14.7 we have
\begin{align*}
\lambda&={1\over2}\bm\beta^T\mathbf X^T
(\mathbf H-\mathbf H_0)\mathbf X\bm\beta/\sigma^2\\
&=
{1\over2\sigma^2}(
\bm\beta^T\mathbf X^T\mathbf X\bm\beta
-\bm\beta^T\mathbf X^T\mathbf 1_NN^{-1}
\mathbf 1_N^T\mathbf X\bm\beta)\\
&=
{1\over2\sigma^2}\left[
\sum_{i=1}^k
(\mu\mathbf 1_{n_i}^T+\tau_i\mathbf 1_{n_i}^T)
(\mu\mathbf 1_{n_i}+\tau_i\mathbf 1_{n_i})-
\sum_{i=1}^k(n_i\mu+n_i\tau_i)(\mu+\overline{\tau})
\right]\\
&=
{1\over2\sigma^2}\left[
\sum_{i=1}^k(n_i\mu^2+2n_i\mu\tau_i+n_i\tau_i^2)
-\sum_{i=1}^k(n_i\mu^2+n_i\mu\overline{\tau}+n_i\tau_i\mu
+n_i\tau_i\overline{\tau})
\right]\\
&=
{1\over2\sigma^2}\left[
N\mu^2+2N\mu\overline{\tau}+\sum_{i=1}^kn_i\tau_i^2
-N\mu^2-N\mu\overline{\tau}-N\mu\overline{\tau}
-N\overline{\tau}^2
\right]\\
&=
{1\over2\sigma^2}\left[
\sum_{i=1}^kn_i\tau_i^2-N\overline{\tau}^2
\right]\\
&=
{1\over2\sigma^2}\left[
\sum_{i=1}^kn_i\tau_i^2-2N\overline{\tau}^2+N\overline{\tau}^2
\right]\\
&=
{1\over2\sigma^2}\left[
\sum_{i=1}^kn_i\tau_i^2
-\sum_{i=1}^k2n_i\tau_i\overline{\tau}
+\sum_{i=1}^kn_i\overline{\tau}^2
\right]\\
&=
{1\over2\sigma^2}\sum_{i=1}^k
(n_i\tau_i^2-2n_i\tau_i\overline{\tau}+n_i\overline{\tau}^2)\\
&=
{1\over2\sigma^2}\sum_{i=1}^nn_i(\tau_i-\overline{\tau})^2
\end{align*}
As mentioned in part (a), for the planned study $n_i=n$ for all $i$.
Hence $n_i$ becomes $n$ and can be factored out of the sum.
The equivalence of $\tau_i-\overline{\tau}$ and
$\mu_i-\overline{\mu}$ was established in part (a).
Therefore
\[
\lambda
=
{\displaystyle\sum_{i=1}^kn_i(\tau_i-\overline{\tau})^2\over2\sigma^2}
=
{\displaystyle\sum_{i=1}^kn(\mu_i-\overline{\mu})^2\over2\sigma^2}
=
{\displaystyle n\sum_{i=1}^k(\mu_i-\overline{\mu})^2\over2\sigma^2}
\]


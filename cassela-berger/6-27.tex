\section*{6.27}
Let $X_1,\ldots,X_n$ be a random sample from the inverse Gaussian
distribution with pdf
$$f(x\mid\mu,\lambda)=\left({\lambda\over2\pi x^3}\right)^{1/2}
e^{-\lambda(x-\mu)^2\over2\mu^2x},\qquad0<x<\infty
$$

\bigskip
\noindent
(a) Show that the statistics
$$\bar X={1\over n}\sum_{i=1}^n X_i
\quad\hbox{and}\quad
T={n\over\sum_{i=1}^n\left({1\over X_i}-{1\over\bar X}\right)}
$$
are sufficient and complete.

\bigskip
\noindent
Note that
$${1\over T}={1\over n}\sum_{i=1}^n\left(
{1\over X_i}-{1\over\bar X}\right),\qquad
{n\over T}+{n\over\bar X}=\sum_{i=1}^n{1\over X_i}
$$
We have
\begin{eqnarray*}
f(\mathbb X\mid\mu,\lambda)&=&\prod_{i=1}^nf(X_i\mid\mu,\lambda)\\
&=&
\left(\prod_{i=1}^nX_i\right)^{-3n/2}
\left({\lambda\over2\pi}\right)^{n/2}
\exp\left(-{\lambda\over2\mu^2}\sum_{i=1}^n{(X_i-\mu)^2\over X_i}\right)\\
&=&
\left(\prod_{i=1}^nX_i\right)^{-3n/2}
\left({\lambda\over2\pi}\right)^{n/2}
\exp\left(
-{\lambda\over2\mu^2}\sum_{i=1}^n{X_i^2-2\mu X_i+\mu^2\over X_i}\right)\\
&=&
\left(\prod_{i=1}^nX_i\right)^{-3n/2}
\left({\lambda\over2\pi}\right)^{n/2}
\exp\left[
-{\lambda\over2\mu^2}\sum_{i=1}^n\left(
X_i-2\mu+{\mu^2\over X_i}\right)\right]\\
&=&
\left(\prod_{i=1}^nX_i\right)^{-3n/2}
\left({\lambda\over2\pi}\right)^{n/2}
\exp\left(
-{n\lambda\over2\mu^2}\bar X
+{n\lambda\over\mu}
-{\lambda\over2}\sum_{i=1}^n{1\over X_i}\right)\\
&=&
\underbrace{
\left(\prod_{i=1}^nX_i\right)^{-3n/2}}_{h(\mathbb X)}
\underbrace{
\left({\lambda\over2\pi}\right)^{n/2}
\exp\left(
-{n\lambda\over2\mu^2}\bar X
+{n\lambda\over\mu}
-{\lambda\over2}\left(
{n\over T}+{n\over\bar X}\right)\right)
}_{g(\bar X,T,\mu,\lambda)}
\end{eqnarray*}
Therefore by the Factorization Theorem, $\bar X$ and $T$ are
sufficient.

\bigskip
\noindent
Now show that $\bar X$ and $T$ are complete.
We have
\begin{eqnarray*}
f(x\mid\mu,\lambda)&=&
\left({\lambda\over2\pi x^3}\right)^{1/2}
e^{-\lambda(x-\mu)^2\over2\mu^2x}I_{(0,\infty)}(x)\\
&=&
\underbrace{
I_{(0,\infty)}(x){1\over x^{3/2}}}_{h(x)}
\underbrace{
\left({\lambda\over2\pi}\right)^{1/2}\exp(\lambda/\mu)
}_{c(\mu,\lambda)}
\exp\left(-{\lambda x\over2\mu^2}-{\lambda\over2x}\right)
\end{eqnarray*}
Therefore the pdf is an exponential family.
By theorem 6.2.25 the statistic
$$\left(\sum X_i,\sum{1\over X_i}\right)$$
is complete.
Let
$$E_\theta h\left(\bar X,T\right)
=E_\theta g\left(n\bar X,{n\over T}+{n\over\bar X}\right)
=E_\theta g\left(\sum X_i,\sum{1\over X_i}\right)=0$$
for all $\theta$.
Then since $\sum X_i$ and $\sum {1\over X_i}$ are complete we have
$P_\theta (h(\bar X, T)=0)=1$ for all $\theta$.
Therefore $\bar X$ and $T$ are complete.

\bigskip
\noindent
(b) For $n=2$, show that $\bar X$ has an inverse Gaussian
distribution, $n\lambda/T$ has a $\chi^2_{n-1}$ distribution,
and they are independent.

%\bigskip
%\noindent
%The mgf for inverse Gaussian is
%$$M_X(t)=\exp\left({\lambda\over\mu}\left(
%1-\sqrt{1-{2\mu^2t\over\lambda}}\right)\right)
%$$
%Since $X_1$ and $X_2$ are iid we have
%$$M_{X_1+X_2}(t)=M_{X_1}(t)\,M_{X_2}(t)
%=\exp\left({2\lambda\over\mu}\left(
%1-\sqrt{1-{2\mu^2t\over\lambda}}\right)\right)
%$$
%Hence
%$$M_{\bar X}(t)=M_{X_1+X_2}(t/2)=
%\exp\left({2\lambda\over\mu}\left(
%1-\sqrt{1-{\mu^2t\over\lambda}}\right)\right)
%$$
%Substituting $\theta/2$ for $\lambda$ we have
%$$M_{\bar X}(t)=
%\exp\left({\theta\over\mu}\left(
%1-\sqrt{1-{2\mu^2t\over\theta}}\right)\right)
%$$
%Therefore the distribution of $\bar X$ is inverse Gaussian.

\bigskip
\noindent
%The joint distribution is
%$$f(x,y)={\lambda\over2\pi}x^{-3/2}y^{-3/2}
%\exp(2\lambda/\mu)
%\exp\left(
%-{\lambda x\over2\mu^2}
%-{\lambda\over2x}
%-{\lambda y\over2\mu^2}
%-{\lambda\over2y}
%\right)
%$$
The transform is
$$\bar X={X_1+X_2\over2},\quad
S=n/T={1\over X_1}+{1\over X_2}-{2\over\bar X}$$
After a lengthy calculation the inverse transform is
$$
X_1=\bar X\left(1-{1\over2}\sqrt{4S\bar X\over S\bar X+2}\right),
\qquad
X_2=\bar X\left(1+{1\over2}\sqrt{4S\bar X\over S\bar X+2}\right)
$$
The associated Jacobian is
$$J=|\det J_{T^{-1}}|={2\bar X^{3/2}\over S^{1/2}(S\bar X+2)^{3/2}}$$
The joint distribution is
$$f(\bar x,s)=f(X_1)f(X_2)\cdot J$$
According to the Schwarz and Samanta paper this works out to be
$$f(\bar x,s)=
\underbrace{
\left({2\lambda\over2\pi\bar x^3}\right)^{1/2}
\exp\left(-{2\lambda(\bar x-\mu)^2\over2\mu^2\bar x}\right)
}_{\bar X\sim IG(\mu,2\lambda)}
\underbrace{
{1\over\Gamma(1/2)}\left(\lambda\over2s\right)^{1/2}
\exp\left(-{\lambda\over2}s\right)
}_{\lambda S\sim\chi^2_1}
$$
Hence $\bar X\sim IG(\mu,2\lambda)$,
$\lambda S\sim\chi^2_1$, and $\bar X\amalg S$.


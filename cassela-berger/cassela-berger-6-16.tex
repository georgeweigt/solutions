\documentclass[12pt]{article}
\usepackage{amsmath}
\usepackage{amssymb}
\usepackage{mathrsfs} % \mathscr

\parindent=0pt

\begin{document}

6.16
A famous example in genetic modeling is a genetic linkage
multinomial model, where we observe the multinomial vector
$(x_1,x_2,x_3,x_4)$ with cell probabilities given by
$({1\over2}+{\theta\over4},{1\over4}(1-\theta),
{1\over4}(1-\theta),{\theta\over4})$.

\bigskip
\noindent
(a) Show that this is a curved exponential family.

\bigskip
\noindent
Let
$$p_1={1\over2}+{\theta\over4},\quad
p_2,p_3={1\over4}(1-\theta),\quad
p_4={\theta\over4}
$$
Then
\begin{eqnarray*}
f(\mathbb X\mid\theta)&=&
{(x_1+x_2+x_3+x_4)!\over x_1!\,x_2!\,x_3!\,x_4!}
p_1^{x_1}
p_2^{x_2}
p_3^{x_3}
p_4^{x_4}\\
&=&{(x_1+x_2+x_3+x_4)!\over x_1!\,x_2!\,x_3!\,x_4!}
\exp\left[
x_1\log p_1
+x_2\log p_2
+x_3\log p_3
+x_4\log p_4
\right]
\end{eqnarray*}
Hence
\begin{eqnarray*}
h(\mathbb X)&=&{(x_1+x_2+x_3+x_4)!\over x_1!\,x_2!\,x_3!\,x_4!}\\
c(\theta)&=&1\\
t_i(\mathbb X)&=&x_i\\
w_i(\theta)&=&\log p_i
\end{eqnarray*}
We have $k=4$ and the dimension of the parameter is $d=1$.
Since $d<k$ this is a curved exponential family.

\bigskip
\noindent
(b) Find a sufficient statistic for $\theta$.

\bigskip
\noindent
Try $T(\mathbb X)=(x_1,x_2+x_3,x_4)$.
$$f(\mathbb X\mid\theta)
=
\underbrace{
{(x_1+x_2+x_3+x_4)!\over x_1!\,x_2!\,x_3!\,x_4!}
}_{h(\mathbb X)}
\underbrace{
\left({1\over2}+{\theta\over4}\right)^{x_1}
\left({1\over4}-{\theta\over4}\right)^{x_2+x_3}
\left({\theta\over4}\right)^{x_4}
}_{g(T(\mathbb X),\theta)}
$$
By the factorization theorem,
$T(\mathbb X)$ is sufficient.

\bigskip
\noindent
(c) Find a minimal sufficient statistic for $\theta$.

\bigskip
\noindent
Try $T(\mathbb X)=(x_1,x_2+x_3,x_4)$ again.
$$h=
{f(\mathbb X\mid\theta)\over f(\mathbb Y\mid\theta)}
=K
\left({1\over2}+{\theta\over4}\right)^{x_1-y_1}
\left({1\over4}-{\theta\over4}\right)^{x_2+x_3-y_2-y_3}
\left({\theta\over4}\right)^{x_4-y_4}
$$
where
$$K={y_1!\,y_2!\,y_3!\,y_4!\,(x_1+x_2+x_3+x_4)!\over
x_1!\,x_2!\,x_3!\,x_4!\,(y_1+y_2+y_3+y_4)!}
$$
If $T(\mathbb X)=T(\mathbb Y)$
then $h$ does not depend on $\theta$.
If $x_1\ne y_1$ then $h$ depends on $\theta$.
If $x_2+x_3\ne y_2+y_3$ then $h$ depends on $\theta$.
If $x_4\ne y_4$ then $h$ depends on $\theta$.
Hence if $T(\mathbb X)\ne T(\mathbb Y)$ then $h$ depends
on $\theta$.
Therefore $T(\mathbb X)$ is minimal sufficient.

\end{document}

\documentclass[12pt]{article}
\usepackage{amsmath}
\usepackage{amssymb}

\parindent=0pt

\begin{document}

1.1
For each of the following experiments, describe the sample space.

\bigskip
\noindent
{\it Toss a coin and the outcome is head or tails.
What if we toss the coin four times in a row?
We can get head or tails on each toss so there are sixteen possible outcomes
of four tosses.
Now we have a new idea, the idea of an experimental outcome
that is the union of individual events.
What if we toss four coins at the same time?
If we cannot distinguish between the coins then there are four
possible outcomes.
}

\bigskip
\noindent
(a) Toss a coin four times.

\bigskip
The sample space is the set of all possible outcomes of four tosses.
\[
S=\left\{\begin{array}{llll}
HHHH, &HHHT, &HHTH, &HHTT,\\
HTHH, &HTHT, &HTTH, &HTTT,\\
THHH, &THHT, &THTH, &THTT,\\
TTHH, &TTHT, &TTTH, &TTTT
\end{array}\right\}
\]

\bigskip
\noindent
(b) Count the number of insect-damaged leaves on a plant.

\bigskip
The sample space is the set of nonnegative integers, $S=\{0,1,2,\ldots\}$.

\bigskip
\noindent
(c) Measure the lifetime (in hours) of a particular brand of light bulb.

\bigskip
The sample space is the set of nonnegative integers, $S=\{0,1,2,\ldots\}$.

\bigskip
\noindent
(d) Record the weights of 10-day-old rats.

\bigskip
The sample space is the set of positive numbers, $S=(0,\infty)$.

\bigskip
\noindent
(e) Observe the proportion of defectives in a shipment of electronic
components.

\bigskip
The sample space is the set of
numbers between zero and one inclusive,
$S=[0,1]$.
%Note that it is not necessary to
%say {\it rational} numbers between
%zero and one.

\end{document}

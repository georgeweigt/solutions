\documentclass[12pt]{article}
\usepackage[margin=0.5in]{geometry}
\usepackage{amssymb,amsmath,bm,mathrsfs}
\usepackage{graphicx}

\begin{document}

\section*{2.1}
(a) Determine the first-order inclusion probabilities.

\begin{align*}
\pi_1=\pi_8&={1\over 8}+5\times{1\over8}\times{1\over5}+{1\over8}
\times{1\over6}={13\over48}=0.271\\
\\
\pi_2=\pi_7&={1\over8}+4\times{1\over8}\times{1\over5}+{1\over8}
\times{1\over6}={59\over240}=0.246\\
\\
\pi_3=\pi_4=\pi_5=\pi_6&={1\over8}+3\times{1\over8}\times{1\over5}
+2\times{1\over8}\times{1\over6}={29\over120}=0.242
\end{align*}

(b) Determine the second-order inclusion probabilities.
Is the design induced by the proposed sampling scheme measurable?

We have
\begin{align*}
\pi_{kk}&=\pi_k\\
\pi_{kl}&=0&\hbox{for $|k-l|=1$}\\
\pi_{18}=\pi_{81}&={1\over8}\times{1\over6}+{1\over8}\times{1\over6}={1\over24}\\
\pi_{1k}=\pi_{k1}&={1\over8}\times{1\over6}+{1\over8}\times
{1\over5}={11\over240}&3\le k\le7\\
\pi_{8k}=\pi_{k8}&={1\over8}\times{1\over6}+{1\over8}\times
{1\over5}={11\over240}&2\le k\le6\\
\pi_{kl}=\pi_{lk}&={1\over8}\times{1\over5}+{1\over8}\times{1\over5}={1\over20}&
\hbox{otherwise}
\end{align*}
Hence
\[
\{\pi_{kl}\}=
\left(
\begin{array}{cccccccc}
 0.2708 &  0.0000 &  0.0458 &  0.0458 &  0.0458 &  0.0458 &  0.0458 &  0.0417\\
 0.0000 &  0.2458 &  0.0000 &  0.0500 &  0.0500 &  0.0500 &  0.0500 &  0.0458\\
 0.0458 &  0.0000 &  0.2417 &  0.0000 &  0.0500 &  0.0500 &  0.0500 &  0.0458\\
 0.0458 &  0.0500 &  0.0000 &  0.2417 &  0.0000 &  0.0500 &  0.0500 &  0.0458\\
 0.0458 &  0.0500 &  0.0500 &  0.0000 &  0.2417 &  0.0000 &  0.0500 &  0.0458\\
 0.0458 &  0.0500 &  0.0500 &  0.0500 &  0.0000 &  0.2417 &  0.0000 &  0.0458\\
 0.0458 &  0.0500 &  0.0500 &  0.0500 &  0.0500 &  0.0000 &  0.2458 &  0.0000\\
 0.0417 &  0.0458 &  0.0458 &  0.0458 &  0.0458 &  0.0458 &  0.0000 &  0.2708\\
\end{array}\right)
\]
The design is not measurable because not all $\pi_{kl}>0$.

%\newpage

(c) Determine the covariances of the sample membership indicators.

We have $\mathop{\rm Cov}(I_k,I_l)=\Delta_{kl}
=\pi_{kl}-\pi_k\pi_l$ hence
\[
\{\mathop{\rm Cov}(I_k,I_l)\}=
\left(
\begin{array}{rrrrrrrr}
 0.1975 &  -0.0666 &  -0.0196 &  -0.0196 &  -0.0196 &  -0.0196 &  -0.0207 &  -0.0317\\
 -0.0666 &  0.1854 &  -0.0594 &  -0.0094 &  -0.0094 &  -0.0094 &  -0.0104 &  -0.0207\\
 -0.0196 &  -0.0594 &  0.1833 &  -0.0584 &  -0.0084 &  -0.0084 &  -0.0094 &  -0.0196\\
 -0.0196 &  -0.0094 &  -0.0584 &  0.1833 &  -0.0584 &  -0.0084 &  -0.0094 &  -0.0196\\
 -0.0196 &  -0.0094 &  -0.0084 &  -0.0584 &  0.1833 &  -0.0584 &  -0.0094 &  -0.0196\\
 -0.0196 &  -0.0094 &  -0.0084 &  -0.0084 &  -0.0584 &  0.1833 &  -0.0594 &  -0.0196\\
 -0.0207 &  -0.0104 &  -0.0094 &  -0.0094 &  -0.0094 &  -0.0594 &  0.1854 &  -0.0666\\
 -0.0317 &  -0.0207 &  -0.0196 &  -0.0196 &  -0.0196 &  -0.0196 &  -0.0666 &  0.1975\\
\end{array}
\right)
\]

(d) Verify that Result 2.6.2 holds in the present application.

For this design $n=2$.
\begin{align*}
\sum_U\pi_k&=2\times{13\over48}+2\times{59\over240}
+4\times{29/120}=2\quad\hbox{which is indeed $n$}
\\
\underset{k\ne l}{\sum\sum}\pi_{kl}&=2\quad
\hbox{which is indeed $n(n-1)$}
\\
\sum_{l\ne1}\pi_{1l}&=0.271\quad\hbox{which is indeed $(n-1)\pi_1$}\\
\sum_{l\ne2}\pi_{2l}&=0.246\quad\hbox{which is indeed $(n-1)\pi_2$}\\
\sum_{l\ne3}\pi_{3l}&=0.242\quad\hbox{which is indeed $(n-1)\pi_3$}\\
\sum_{l\ne4}\pi_{4l}&=0.242\quad\hbox{which is indeed $(n-1)\pi_4$}\\
\sum_{l\ne5}\pi_{5l}&=0.242\quad\hbox{which is indeed $(n-1)\pi_5$}\\
\sum_{l\ne6}\pi_{6l}&=0.242\quad\hbox{which is indeed $(n-1)\pi_6$}\\
\sum_{l\ne7}\pi_{7l}&=0.246\quad\hbox{which is indeed $(n-1)\pi_7$}\\
\sum_{l\ne8}\pi_{8l}&=0.271\quad\hbox{which is indeed $(n-1)\pi_8$}
\end{align*}

The following C code was used to generate the above tables and results.

\begin{verbatim}
	/* calculate 2nd order inclusion probabilities */

	for (i = 0; i < 8; i++)
		for (j = 0; j < 8; j++) {
			if (abs(i - j) < 2)
				continue;
			if (i == 0 && j == 7 || i == 7 && j == 0)
				p[i][j] = 1.0 / 24.0;
			else if (i == 0 || i == 7 || j == 0 || j == 7)
				p[i][j] = 11.0 / 240.0;
			else
				p[i][j] = 1.0 / 20.0;
		}

	p[0][0] = 13.0 / 48.0;
	p[1][1] = 59.0 / 240.0;
	p[2][2] = 29.0 / 120.0;
	p[3][3] = 29.0 / 120.0;
	p[4][4] = 29.0 / 120.0;
	p[5][5] = 29.0 / 120.0;
	p[6][6] = 59.0 / 240.0;
	p[7][7] = 13.0 / 48.0;

	printf("covariance of indicators\n");

	for (i = 0; i < 8; i++)
		for (j = 0; j < 8; j++)
			cov[i][j] = p[i][j] - p[i][i] * p[j][j];

	printf("2nd property of Result 2.6.2\n");

	sum = 0.0;
	for (i = 0; i < 8; i++)
		for (j = 0; j < 8; j++)
			if (i != j)
				sum += p[i][j];
	printf("%g\n", sum);

	printf("3rd property of Result 2.62\n");

	for (i = 0; i < 8; i++) {
		sum = 0.0;
		for (j = 0; j < 8; j++)
			if (i != j)
				sum += p[i][j];
		printf("%g\n", sum);
	}
\end{verbatim}

%\newpage

\section*{2.4}
We have $n=300$ and $N=800$.
The first and second order inclusion probabilities are
\[
E(I_k)=\pi_k={n\over N}={3\over 8}\qquad E(I_kI_l)=\pi_{kl}={n(n-1)\over N(N-1)}={897\over 6392},\quad k\ne l
\]
For the expectation we have
\begin{align*}
E(n_s)&=E\left(\sum_{k=1}^{800}n_kI_k\right)\\
&=\sum_{k=1}^{800}n_kE(I_k)\\
%&=\sum_{k=1}^{800}n_k\pi_k\\
&={3\over8}\sum_{k=1}^{800}n_k\\
&={3\over8}\times1600\\
&=600
\end{align*}

For the variance we have
\begin{align*}
\mathop{\rm Var}(n_s)&=\mathop{\rm Var}\left(\sum_{k=1}^{800}n_kI_k\right)\\
&=\sum_{k=1}^{800}\sum_{l=1}^{800}\mathop{\rm Cov}(n_kI_k,n_lI_l)\\
&=\sum_{k=1}^{800}\sum_{l=1}^{800}n_kn_l\mathop{\rm Cov}(I_k,I_l)\\
&=\sum_{k=1}^{800}\sum_{l=1}^{800}n_kn_l(\pi_{kl}-\pi_k\pi_l)\\
&=140.8
\end{align*}

The following R code was used to calculate the variance.
\begin{verbatim}
n = 300
N = 800
nk = c(rep(1,250),rep(2,350),rep(3,150),rep(4,50))
nknl = nk %o% nk
Cov = matrix(1,800,800) * n * (n - 1) / N / (N - 1)
diag(Cov) = n / N
Cov = Cov - (n / N) ^ 2
Var = sum(nknl * Cov)
print(Var)
\end{verbatim}

%\newpage

\section*{2.9}
The key is to write $E_{BE}(S^2_{ys})$ as a sum of conditional
expectations.
\begin{align*}
E_{BE}(S^2_{ys})&=\sum_{n=2}^NE_{SI}(S^2_{ys}\mid n_s=n)P(n_s=n)\\
&=
S^2_{yU}\sum_{n=2}^NP(n_s=n)\qquad
\hbox{This step is by exercise 2.8, proved below.}\\
&=
S^2_{yU}\bigg[1-P(n_s=0)-P(n_s=1)\bigg]\\
&=
S^2_{yU}\bigg[1-(1-\pi)^N-N\pi(1-\pi)^{N-1}\bigg]\\
\end{align*}
Therefore
\begin{align*}
{E_{BE}(S^2_{ys})-S^2_{yU}\over S^2_{yU}}
&={E_{BE}(S^2_{ys})\over S^2_{yU}}-1\\
&=
1-(1-\pi)^N-N\pi(1-\pi)^{N-1}-1\\
&=
-(1-\pi)^{N-1}(1-\pi+N\pi)\\
&=
-(1-\pi)^{N-1}[1+(N-1)\pi]
\end{align*}
Proof that $E_{SI}(S^2_{ys})=S^2_{yU}$.
%First a few preliminaries. We have
%\begin{align*}
%S^2_{ys}&={1\over n-1}\sum_s(y_k-\bar y_s)^2\\
%&={1\over2n(n-1)}\underset{s}{\sum\sum}(y_k-y_l)^2\\
%&={1\over2n(n-1)}\underset{U}{\sum\sum}(y_k-y_l)^2I_kI_l
%\end{align*}
%and
%\begin{align*}
%S^2_{yU}&={1\over N-1}\sum_U(y_k-\bar y_U)^2\\
%&={1\over 2N(N-1)}\underset{U}{\sum\sum}(y_k-y_l)^2
%\end{align*}
%Recall that for the SI design we have
%\[
%\pi_{kl}={n(n-1)\over N(N-1)}
%\]
%Now we can put it all together.
\begin{align*}
E_{SI}(S^2_{ys})&=E\left[
{1\over2n(n-1)}\underset{U}{\sum\sum}(y_k-y_l)^2I_kI_l
\right]\\
&=
{1\over2n(n-1)}\underset{U}{\sum\sum}(y_k-y_l)^2E(I_kI_l)\\
&=
{1\over2n(n-1)}\underset{U}{\sum\sum}(y_k-y_l)^2\pi_{kl}\\
&=
{1\over2n(n-1)}{n(n-1)\over N(N-1)}\underset{U}{\sum\sum}(y_k-y_l)^2\\
&=
{1\over2N(N-1)}\underset{U}{\sum\sum}(y_k-y_l)^2\\
&=S^2_{yU}
\end{align*}

%\newpage

\section*{2.}
Recall that $X=4,6,6,20$ acres of land and $Y=1,3,5,15$ acres or corn.

(a) For $\pi_k\propto 1+0.5x_k$ we have
\[
c(1+0.5\cdot4)+c(1+0.5\cdot6)+c(1+0.5\cdot6)=1
\]
hence $c={1\over11}$ and the first-order inclusion probabilities
are
\[
\pi_1={3\over11}\quad\pi_2={4\over11}\quad\pi_3={4\over11}
\quad\pi_4=1
\]
The HT estimate of total acres of corn is
\[
\widehat t_\pi=\sum_{k\in U}{y_kI_k\over\pi_k}
\]
hence

\begin{center}
\begin{tabular}{ccccc}
Sample & Measured & $\widehat t_\pi$ (in class) & $\widehat t_\pi$ (alternate)\\
\hline
1, 4 & 1, 15 & 18 & 18.67 & $=1\cdot{11\over3}+15$\\
2, 4 & 3, 15 & 24 & 23.25 & $=3\cdot{11\over4}+15$\\
3, 4 & 5, 15 & 30 & 28.75 & $=5\cdot{11\over4}+15$
\end{tabular}
\end{center}

\begin{center}
%\includegraphics[scale=0.5]{Rplot-2.pdf}
\end{center}

\begin{verbatim}
# R code for the above plot
plot(c(18,24,30),c(1/3,1/3,1/3),ylab="Probability",xlab="Acres of Corn",pch=23)
points(c(18.67,23.25,28.75),c(3/11,4/11,4/11),pch=19)
legend("topleft",c("In class","Alternate"),pch=c(23,19))
title("Distribution of HT Estimators")
\end{verbatim}

%\newpage
(b) Show directly from the randomization distribution that the HT
estimator is unbiased under the alternative design.

\[
E(\widehat t_\pi)=
{3\over11}\cdot18.67
+{4\over11}\cdot23.25
+{4\over11}\cdot28.75
=24
\]
hence the estimator is unbiased because $\sum Y=1+3+5+15=24$.

(c) Compute the design variance of the HT estimator under the
alternative design directly from the randomization distribution.

\[
E(\widehat t_\pi^2)=
{3\over11}\cdot18.67^2
+{4\over11}\cdot23.25^2
+{4\over11}\cdot28.75^2
=592.17
\]
\[
\mathop{\rm Var}(\widehat t_\pi)
=E(\widehat t_\pi^2)-[E(\widehat t_\pi)]^2
=592.17-24^2=16.17
\]

(d) For the alternative design write down all of the second-order
inclusion probabilities $\{\pi_{kl}\}$ in a $4\times4$ table
and write down all of the design covariances $\{\Delta_{kl}\}$
in a $4\times4$ table.

\[
\{\pi_{kl}\}=
\begin{pmatrix}
{3\over11} & 0 & 0 & {3\over11}\\ \\
0 & {4\over11} & 0 & {4\over11}\\ \\
0 & 0 & {4\over11} & {4\over11}\\ \\
{3\over11} & {4\over11} & {4\over11} & 1
\end{pmatrix}
\]

\[
\{\Delta_{kl}\}
=\{\pi_{kl}\}-\{\pi_k\pi_l\}=\left(\begin{array}{rrrr}
  0.19834711 &-0.09917355 &-0.09917355    &0.00000000\\
 -0.09917355  &0.23140496 &-0.13223140    &0.00000000\\
 -0.09917355 &-0.13223140  &0.23140496    &0.00000000\\
  0.00000000  &0.00000000  &0.00000000    &0.00000000
\end{array}\right)
\]

\begin{verbatim}
# R code for covariance table
pi.kl = rbind(c(3/11,0,0,3/11),c(0,4/11,0,4/11),c(0,0,4/11,4/11),c(3/11,4/11,4/11,1))
pi.k = c(3/11,4/11,4/11,1)
Delta = pi.kl - pi.k %o% pi.k
print(Delta)
\end{verbatim}

%\newpage

(e) Use the design covariances computed above and either of the forms
of $V_p(\widehat t_\pi)$ to calculate the design variance of the
HT estimator under the alternative design.
Compare to (c) and compare to the design variance of HT under
the original in-class design.

\[
V_p(\widehat t_\pi)=\underset{k, l\in U}{\sum\sum}
\Delta_{kl}{y_k\over\pi_k}{y_l\over\pi_l}=16.17
\]

\begin{verbatim}
# R code for the above calculation
y = c(1,3,5,15)
pi.k = c(3/11,4/11,4/11,1)
yp = y / pi.k
V = sum(Delta * yp %o% yp)
print(V)
\end{verbatim}

Another way to compute the variance:

\[
V_p(\widehat t_\pi)=\underset{k,l\in U}{\sum\sum}
\left({\pi_{kl}\over\pi_k\pi_l}-1\right)y_ky_l=16.17
\]

\begin{verbatim}
# R code for the above calculation
pi.k = c(3/11,4/11,4/11,1)
pi.kl = rbind(c(3/11,0,0,3/11),c(0,4/11,0,4/11),c(0,0,4/11,4/11),c(3/11,4/11,4/11,1))
y = c(1,3,5,15)
p = pi.kl / (pi.k %o% pi.k) - 1
V = sum(p * (y %o% y))
print(V)
\end{verbatim}

For the original in-class design:

\[
V_p(\widehat t_\pi)=\underset{k,l\in U}{\sum\sum}
\left({\pi_{kl}\over\pi_k\pi_l}-1\right)y_ky_l=24
\]

\begin{verbatim}
# R code for the above calculation
pi.k = c(1/3,1/3,1/3,1)
pi.kl = rbind(c(1/3,0,0,1/3),c(0,1/3,0,1/3),c(0,0,1/3,1/3),c(1/3,1/3,1/3,1))
y = c(1,3,5,15)
p = pi.kl / (pi.k %o% pi.k) - 1
V = sum(p * (y %o% y))
print(V)
\end{verbatim}

Check using the discrete distribution.

\[
V_p(\widehat t_\pi)=
{1\over3}\cdot18^2+{1\over3}\cdot24^2+{1\over3}\cdot30^2
-24^2=600-576=24
\]

Comparisons:
The result 16.17 matches (c) exactly.
The variance of the alternate design is less than
the in-class design.

%\newpage

\section*{3.}
(a) True values of total population and average revenue using the MEANS procedure.
\begin{verbatim}
proc means data = mu284 sum;
var P85;
run;                                    
                              Analysis Variable : P85

                                             Sum
                                    ------------
                                         8339.00
                                    ------------

proc means data = mu284 mean stddev stderr;
var RMT85;
run;                           
                             Analysis Variable : RMT85
                             
                           Mean         Std Dev       Std Error
                    --------------------------------------------
                     245.0880282     596.3325394      35.3858260
                    --------------------------------------------
\end{verbatim}

(b) Estimates of total population and average revenue from sample using the SURVEYMEANS procedure.
\begin{verbatim}
proc surveyselect data = mu284 method = srs n = 90 seed = 2007 out = sample1;
run;
data sample;
set sample1;
w = 284/90;
run;

proc surveymeans data = sample rate = 0.3169014085 sum;
var P85;
weight w;
run;
                      Variable             Sum         Std Dev
                      ----------------------------------------
                      P85          7084.222222      641.342070
                      ----------------------------------------

proc surveymeans data = sample rate = 0.3169014085 mean;
var RMT85;
run;
                                                     Std Error
                      Variable            Mean         of Mean
                      ----------------------------------------
                      RMT85         188.900000       18.655218
                      ----------------------------------------
\end{verbatim}

(c) Compare estimated and actual standard errors.

\begin{center}
\begin{tabular}{|l|r|r|}
\hline
& Standard Deviation & Standard Error of\\
& of Total Population & Average Revenue\\
\hline
Actual & 0.00 & 35.39\\
Estimator & 641.34 & 18.66\\
\hline
\end{tabular}
\end{center}

(d) The true value of the total population is 8339 which falls within the confidence interval.
\begin{verbatim}
proc surveymeans data = sample rate = 0.3169014085 clsum;
var P85;
weight w;
run;
                         Variable        95% CL for Sum
                          ---------------------------------
                          P85         5809.88919 8358.55525
                          ---------------------------------
\end{verbatim}

The true value of the average revenue is 245 which falls outside of the confidence interval.
\begin{verbatim}
proc surveymeans data = sample rate = 0.3169014085 clm;
var RMT85;
run;
                          Variable       95% CL for Mean
                          ---------------------------------
                          RMT85       151.832479 225.967521
                          ---------------------------------
\end{verbatim}


\section*{3.10}
(a) Compute the standard error of the $\pi$ estimator for each ordering.
\[
STDERR=\sqrt{V_{SY}(\hat t_\pi)}
=
\left[a\sum_{r=1}^at_{s_r}^2-t^2\right]^{1/2}
\]
\begin{align*}
O_1:\quad STDERR&=\big[10\cdot286087326-51874^2\big]^{1/2}=13036.9\\
O_2:\quad STDERR&=\big[10\cdot272064230-51874^2\big]^{1/2}=5452.6\\
O_3:\quad STDERR&=\big[10\cdot270214732-51874^2\big]^{1/2}=3351.9
\end{align*}

\bigskip

(b) Compute that standard error of the $\pi$ estimator under the SI design
with $n=28$.
\[
SE=\sqrt{V_{SI}(\hat t_\pi)}
=\left[N^2\left({1\over n}-{1\over N}\right)S_{yU}^2\right]^{1/2}
=\left[280^2\times\left({1\over28}-{1\over280}\right)\times35932\right]^{1/2}
=9515.7
\]

(c) Compute the homogeneity measure $\delta$ for each of the three orderings,
and compute $\delta_{\rm min}$ as defined in section 3.4.

We have
\[
SST=(N-1)S_{yU}^2=279\cdot35932=10025028
\]
\[
\delta=1-{N-1\over N-a}\cdot{SST-SSB\over SST}
=1-{279\over270}\cdot{10025028-SSB\over10025028}
\]

\begin{center}
\begin{tabular}{|c|r|r|}
\hline
Order & $SSB=V_{SY}(\hat t_\pi)/N$ & $\delta$\\
\hline
1 & $607005$ & $0.02923$\\
2 & $106180$ & $-0.02239$\\
3 & $40127$ & $-0.02920$\\
\hline
\end{tabular}
\end{center}

\[
\delta_{\rm min}=-{a-1\over N-a}=-{10-1\over280-10}=-0.03333
\]

%\newpage

\section*{3.14}
(a) Compute an unbiased estimate of the total of P85 for this population of size four.

We have $y_k$ from P85 and $x_k$ from P75.

\begin{center}
\begin{tabular}{|r|r|r|}
\hline
$k$ & $y_k$ & $x_k$\\
\hline
16 & 653 & 671\\
137 & 424 & 446\\
114 & 229 & 247\\
29 & 153 & 138\\
\hline
\end{tabular}
\end{center}

\[
T_N=\sum_U x_k=671+446+247+138=1502
\]
%From equation 3.6.7 on p.\ 92 we have
\[
\pi_{16}=2\cdot{x_{16}\over T_N}=2\cdot{671\over1502}=0.8935
\]
\[
\pi_{114}=2\cdot{x_{114}\over T_N}=2\cdot{247\over1502}=0.3289
\]
From p.\ 93 the estimate is
\[
\hat t_\pi=\sum_s{y_k\over\pi_k}={653\over0.8935}+{229\over0.3289}
=1427
\]

(b) Compute an unbiased estimate of the variance. Also, compute the associated
$cve$.

From p.\ 92 we have
\[
c_k={x_k(T_N-x_k)\over T_N(T_N-2x_k)}=2.32, 0.514, 0.205, 0.102
\]
\[
\sum_Uc_k=3.141
\]
\[
\pi_{16,114}={2x_{16}x_{114}\over T_N\sum_U c_k}
{T_N-x_{16}-x_{114}\over(T_N-2x_{16})(T_N-2x_{114})}=0.2544
\]
An unbiased estimate of the variance is
\[
\widehat V(\hat t_\pi)=-{\pi_{16,114}-\pi_{16}\pi_{114}\over\pi_{16,114}}
\left({y_{16}\over\pi_{16}}-{y_{114}\over\pi_{114}}\right)^2=185.5
\]
The associated $cve$ (estimated coefficient of variation) from equation 2.7.4 on
p.\ 42 is
\[
cve={\sqrt{\widehat V(\hat t_\pi)}\over\hat t_\pi}={\sqrt{185.5}\over1427}=0.009544
\]

%\newpage

\section*{3.20}
We have
\begin{align*}
E_{SY}(\widehat V)&=\sum_{r=1}^aE_{SY}(\widehat V\mid s=s_r)P(s=s_r)\\
&=
{1\over a}\sum_{r=1}^a\left(
\underset{k,l\in s_r}{\sum\sum}
{\pi_{kl}-\pi_k\pi_l\over\pi_{kl}}{y_k\over\pi_k}{y_l\over\pi_l}\right)\\
&=
{1\over a}\sum_{r=1}^a\left(
\underset{k,l\in s_r}{\sum\sum}
{a^{-1}-a^{-1}a^{-1}\over a^{-1}}
{y_k\over a^{-1}}
{y_l\over a^{-1}}\right)\\
&=(a-1)\sum_{r=1}^a\left(\underset{k,l\in s_r}{\sum\sum}y_ky_l\right)\\
&=(a-1)\sum_{r=1}^at_{s_r}^2
\end{align*}
Therefore
\begin{align*}
E_{SY}(\widehat V)-V_{SY}(\hat t_\pi)
&=
(a-1)\sum_{r=1}^at_{s_r}^2-\left(a\sum_{r=1}^at_{s_r}^2-t^2\right)\\
&=
t^2-\sum_{r=1}^at_{s_r}^2\\
&=
\left(\sum_{r=1}^at_{s_r}\right)^2-\sum_{r=1}^at_{s_r}^2\\
&=
\sum_{r=1}^a\sum_{r'=1}^at_{s_r}t_{s_{r'}}-\sum_{r=1}^at_{s_r}t_{s_r}\\
&=
\underset{r\ne r'}{\sum\sum}t_{s_r}t_{s_{r'}}
\end{align*}

%\newpage

\section*{Appendix}

Below is a one page excerpt of the R code for Exercise 3.10.

\begin{verbatim}
rm(list=ls(all=TRUE))

DATA = read.table("mu284.txt",header=T)

# Remove the four largest municipalities.
p = order(DATA$P85,decreasing=TRUE)[1:4]
DATA = DATA[-p,]

rmt85 = DATA$RMT85
a = 10
s = 10 * (0:27)

# O1: Appendix ordering

tvec = c(
sum(rmt85[s + 1]),
sum(rmt85[s + 2]),
sum(rmt85[s + 3]),
sum(rmt85[s + 4]),
sum(rmt85[s + 5]),
sum(rmt85[s + 6]),
sum(rmt85[s + 7]),
sum(rmt85[s + 8]),
sum(rmt85[s + 9]),
sum(rmt85[s + 10]))

print(tvec)
print(sum(tvec))
print(sum(tvec^2))

t = sum(tvec)
tbar = sum(tvec) / a

V = a * sum(tvec^2) - t^2
SE = sqrt(V)
print(SE)
V = a * sum((tvec - tbar)^2)
SE = sqrt(V)
print(SE)
V1 = V
\end{verbatim}


\section*{3.15}
First, a few preliminaries.
The following statistics will be needed to determine the sample allocations.
\[
S_{xU_h}^2={1\over N_h-1}\sum_{U_h}x_k^2-{1\over N_h(N_h-1)}
\left(\sum_{U_h}x_k\right)^2
\]

\begin{center}
\begin{tabular}{|c|r|r|r|r|}
\hline
$h$ & $N_h$ & $\sum_{U_h}x_k$ & $\sum_{U_h}x_k^2$ & $S_{xU_h}$\\
\hline
1 & 44 & 1,518 & 52,764 & 3.023\\
2 & 168 & 7,524 & 339,344 & 3.772\\
3 & 56 & 3,198 & 184,168 & 5.290\\
4 & 16 & 1,260 & 100,016 & 7.262\\
\hline
\end{tabular}
\end{center}

\[
N=\sum_{h=1}^HN_h=284\qquad
\sum_{h=1}^HN_hS_{xU_h}=1179\qquad
\sum_Ux_k=13500
\]

(a) Allocate a total sample size of $n=40$.

\begin{center}
\begin{tabular}{|c|r|r|r|}
\hline
& Proportional Allocation & $x$-Optimal Allocation &
Allocation Prop. to $x$-Total\\
$h$ & $n_h=n\cdot N_h/N$ &
$n_h=nN_hS_{xU_h}\big/\sum_{h=1}^HN_hS_{xU_h}$ &
$n_h=n\sum_{U_h}x_k\big/\sum_Ux_k$\\
\hline
1 & 6 & 5 & 4\\
2 & 24 & 21 & 23\\
3 & 8 & 10 & 9\\
4 & 2 & 4 & 4\\
\hline
\end{tabular}
\end{center}

(b) Compare the variance of an SI design to the
variances of STSI for the three allocation methods.
\[
V_{SI}(\hat t_\pi)=N^2\left({1\over n}-{1\over N}\right)S_{yU}^2=91059
\qquad(n=40,\,N=284,\,S_{yU}^2=52.56222)
\]
\[
V_{STSI}(\hat t_\pi)=\sum_{h=1}^HN_h^2\left(
{1\over n_h}-{1\over N_h}\right)S_{yU_h}^2
\]

\begin{center}
\begin{tabular}{|l|r|}
\hline
Allocation & $V_{STSI}(\hat t_\pi)$ \\
\hline
Proportional & 44093\\
$x$-Optimal & 45229\\
$x$-Total & 45354\\
\hline
\end{tabular}
\end{center}

$V_{SI}(\hat t_\pi)=91059$ is higher than STSI for all allocation methods.

%\newpage

(i) SI sample of size 30 from MU284.

\begin{verbatim}
data mu284;
w = 284/30;
infile 'c:\sas\mu284b.txt';
input LABEL   P85   P75   RMT85   CS82   SS82   S82   ME84   REV84   REG   CL;
if S82 le 40 then STRATUM = 1;
else if S82 le 50 then STRATUM = 2;
else if S82 le 70 then STRATUM = 3;
else STRATUM = 4;
run;

proc surveyselect data = mu284 method = srs n = 30 seed = 222007 out = sample;
run;

proc surveymeans data = sample rate = .1056338028 sum varsum clsum;
var SS82;
weight w;
run;

  Variable             Sum         Std Dev      Var of Sum        95% CL for Sum
  ---------------------------------------------------------------------------------
  SS82         6560.400000      379.907937          144330    5783.40103 7337.39897
  ---------------------------------------------------------------------------------
\end{verbatim}

(ii) STSI sample size of 30 with proportional allocation.

\begin{verbatim}
proc sort data = mu284;
by S82;
run;

proc surveyselect data = mu284 method = srs n = 30 seed = 222007 out = sample;
strata STRATUM / alloc = prop;
run;

proc surveymeans data = sample rate = .1056338028 sum varsum clsum;
var SS82;
weight samplingweight;
run;

  Variable             Sum         Std Dev      Var of Sum        95% CL for Sum
  ---------------------------------------------------------------------------------
  SS82         6108.666667      336.077238          112948    5421.31154 6796.02180
  ---------------------------------------------------------------------------------
\end{verbatim}

%\newpage

(iii) STSI sample of size 30 with $x$-optimal allocation.

\begin{verbatim}
proc surveyselect data = mu284 method = srs n = 30 seed = 222007 out = sample;
strata STRATUM / alloc = (.1128 .5374 .2512 .0985);
run;

proc surveymeans data = sample rate = .1056338028 sum varsum clsum;
var SS82;
weight samplingweight;
run;

  Variable             Sum         Std Dev      Var of Sum        95% CL for Sum
  ---------------------------------------------------------------------------------
  SS82         6106.666667      244.287107           59676    5607.04344 6606.28990
  ---------------------------------------------------------------------------------
\end{verbatim}

(iv) STSI sample of size 30 with allocation proportional to $x$-total.

\begin{verbatim}
proc surveyselect data = mu284 method = srs n = 30 seed = 222007 out = sample;
strata STRATUM / alloc = (.1124 .5573 .2369 .0933);
run;

proc surveymeans data = sample rate = .1056338028 sum varsum clsum;
var SS82;
weight samplingweight;
run;

  Variable             Sum         Std Dev      Var of Sum        95% CL for Sum
  ---------------------------------------------------------------------------------
  SS82         6010.196078      302.121114           91277    5392.28902 6628.10314
  ---------------------------------------------------------------------------------
\end{verbatim}

%\newpage

(v) Systematic $\pi$ps sample of size 30 with S82 as size measure.

Assumption: No $x_k>450$ because the sampling interval is
$a=\sum_Ux_k/n=13500/30=450$.
If $x_k>450$ then SAS might repeat $y_k$ in the sample.

\begin{verbatim}
proc surveyselect data = mu284 method = pps_sys n = 30 seed = 222007 out = sample;
size S82;
control S82;
run;

proc surveymeans data = sample rate = .1056338028 sum varsum clsum;
var SS82;
weight samplingweight;
run;

  Variable             Sum         Std Dev      Var of Sum        95% CL for Sum
  ---------------------------------------------------------------------------------
  SS82         6496.201433      195.695379           38297    6095.95944 6896.44342
  ---------------------------------------------------------------------------------
\end{verbatim}

\section*{Appendix}
The following R code was used for parts (a) and (b).

\begin{verbatim}
rm(list=ls(all=TRUE))

DATA = read.table("mu284.txt",header=T)

Nh = c(44,168,56,16)
x = c(1518,7524,3198,1260)
x2 = c(52764,339344,184168,100016)

s82 = DATA$S82
st1 = DATA[s82 <= 40,]
st2 = DATA[s82 >= 41 & s82 <= 50,]
st3 = DATA[s82 >= 51 & s82 <= 70,]
st4 = DATA[s82 >= 71,]

print("Sx, method 1")
Sx = sqrt(1 / (Nh - 1) * x2 - 1 / Nh / (Nh - 1) * x^2)
print(Sx)

print("Sx, method 2")
Sx = c(sd(st1$S82),sd(st2$S82),sd(st3$S82),sd(st4$S82))
print(Sx)

n = 40
N = sum(Nh)

print("proportional allocation")
nh = n * Nh / N
print(round(nh))

print("x-optimal allocation")
nh = n * Nh * Sx / sum(Nh * Sx)
print(round(nh))

print("alloc. prop. to x-total")
nh = n * x / sum(x)
print(round(nh))

y = c(sum(st1$SS82),sum(st2$SS82),sum(st3$SS82),sum(st4$SS82))
y2 = c(sum(st1$SS82^2),sum(st2$SS82^2),sum(st3$SS82^2),sum(st4$SS82^2))

S2y = 1 / (Nh - 1) * y2 - 1 / Nh / (Nh - 1) * y^2
print("S2y, method 1")
print(S2y)

S2y = c(var(st1$SS82),var(st2$SS82),var(st3$SS82),var(st4$SS82))
print("S2y, method 2")
print(S2y)

print("V_SI")
V = N^2 * (1 / n - 1 / N) * var(DATA$SS82)
print(V)

print("V_STSI")

nh = c(6,24,8,2)
V = sum(Nh^2 * (1 / nh - 1 / Nh) * S2y)
print(V)

nh = c(5,21,10,4)
V = sum(Nh^2 * (1 / nh - 1 / Nh) * S2y)
print(V)

nh = c(4,23,9,4)
V = sum(Nh^2 * (1 / nh - 1 / Nh) * S2y)
print(V)
\end{verbatim}


\section*{4.8}
(a) Compute an unbiased estimate of the total of the variable CS82.

First calculate $\hat t_i$ for each selected cluster $i$.
\[
\pi_{k\mid i}={n_i\over N_i}\qquad
\hat t_i=\sum_{k\in s_i}{y_k\over\pi_{k\mid i}}
\]
%
\begin{align*}
\hat t_{11}&={8\over2/6}+{10\over2/6}=54\\
\hat t_5&={13\over2/5}+{12\over2/5}=62.5\\
\hat t_{20}&={11\over2/5}+{8\over2/5}=47.5\\
\hat t_{48}&={2\over2/5}+{7\over2/5}=22.5\\
\hat t_{14}&={16\over2/7}+{10\over2/7}=91
\end{align*}
Now calculate $\hat t_\pi$.
\[
\pi_{\mathrm Ii}={n_{\mathrm I}\over N_{\mathrm I}}={1\over10}\qquad
\hat t_\pi=\sum_{i\in s_{\mathrm I}}{\hat t_i\over\pi_{\mathrm Ii}}
={54+62.5+47.5+22.5+92\over1/10}=2775
\]

(b) Compute unbiased estimates of $V_{PSU}$, $V_{SSU}$, and
$V=V_{PSU}+V_{SSU}$.
Also, compute the corresponding $cve$.

For SI we have
\[
\pi_{kl\mid i}={n_i(n_i-1)\over N_i(N_i-1)}\quad\hbox{when $k\ne l$}
\]
Hence
\[
\widehat V_i=\underset{k,l\in s_i}{\sum\sum}{\pi_{kl\mid i}-\pi_{k\mid i}\pi_{l\mid i}
\over\pi_{kl\mid i}}{y_k\over\pi_{k\mid i}}{y_l\over\pi_{l\mid i}}
\]
\begin{center}
\begin{tabular}{|r|r|}
\hline
$i$ & $\widehat V_i$\\
\hline
11 & 24.00\\
5 & 3.75\\
20 & 33.75\\
48 & 93.75\\
14 & 315.00\\
\hline
\end{tabular}
\end{center}
%
\[
\pi_{\mathrm Ii}={n_{\rm I}\over N_{\rm I}}={5\over50}={1\over10}
\]
\[
\pi_{\mathrm Iij}={n_{\rm I}(n_{\rm I}-1)\over N_{\rm I}(N_{\rm I}-1)}
={5(5-1)\over 50(50-1)}={2\over245}\quad\hbox{when $i\ne j$}
\]
\[
\widehat V_{PSU}=\underset{i,j\in s_{\rm I}}{\sum\sum}
{\pi_{\mathrm Iij}-\pi_{\mathrm Ii}\pi_{\mathrm Ij}
\over\pi_{\mathrm Iij}}
{\hat t_i\over\pi_{\mathrm Ii}}
{\hat t_j\over\pi_{\mathrm Ij}}
-\sum_{i\in s_{\rm I}}{1\over\pi_{\mathrm Ii}}
\left({1\over\pi_{\mathrm Ii}}-1\right)\widehat V_i
=277256.2-42322.5=234,933.7
\]
\[
\widehat V_{SSU}=\sum_{i\in s_{\rm I}}{\widehat V_i\over\pi_{\mathrm Ii}^2}=47,025
\]
\[
\widehat V=\widehat V_{PSU}+\widehat V_{SSU}=281,958.7
\]
\[
cve(\hat t_\pi)={\sqrt{\widehat V}\over\hat t_\pi}={531\over2775}=0.19
\]

(c) Compute simplified variance estimates according to equations (4.6.1)
and (4.6.2), as well as the corresponding $cve$s.
Comment on your conclusions.

\[
\widehat V^*=-{1\over2}\underset{i,j\in s_{\rm I}}{\sum\sum}
{\pi_{\mathrm Iij}-\pi_{\mathrm Ii}\pi_{\mathrm Ij}
\over\pi_{\mathrm Iij}}
\left({\hat t_i\over\pi_{\mathrm Ii}}-{\hat t_j\over\pi_{\mathrm Ij}}\right)^2
=277,256.3
\]
%
\[
\widehat V^{**}={1\over n_{\rm I}(n_{\rm I}-1)}\sum_{i\in s_{\rm I}}
\left({\hat t_i\over\pi_{\rm Ii}/n_{\rm I}}-\hat t_\pi\right)^2=308,062.5
\]
%
\[
{\sqrt{\widehat V^*}\over\hat t_\pi}={527\over2775}=0.19
\]
%
\[
{\sqrt{\widehat V^{**}}\over\hat t_\pi}={555\over2775}=0.20
\]
Both $\widehat V^*$ and $\widehat V^{**}$ are close to the unbiased estimator
$\widehat V$ in part (b).

%\newpage

\section*{4.22}
Show that in multistage sampling where the first stage design is SI
(of $n_{\rm I}$ PSUs from $N_{\rm I}$), the simplified variance estimator
\[
\widehat V^{**}=N_{\rm I}^2{1\over n_{\rm I}(n_{\rm I}-1)}
\sum_{i\in s_{\rm I}}\left[
\hat t_i-\left(\sum_{i\in s_{\rm I}}\hat t_i/n_{\rm I}\right)\right]^2
\]
overestimates the variance of the $\pi$ estimator.

By the identity
\[
\sum(y_i-\overline{y})^2=\sum y_i^2-{1\over n}\left(\sum y_i\right)^2
\]
we have
\[
\widehat V^{**}=N_{\rm I}^2{1\over n_{\rm I}(n_{\rm I}-1)}
\left[\sum_{i\in s_{\rm I}}\hat t_i^2-{1\over n_{\rm I}}\left(\sum_{i\in s_{\rm I}}
\hat t_i\right)^2\right]
\]
%Add indicators and sum over the universe.
%\[
%\widehat V^{**}=N_{\rm I}^2{1\over n_{\rm I}(n_{\rm I}-1)}
%\left[\sum_{U_{\rm I}}\hat t_i^2I(i\in s_{\rm I})
%-{1\over n_{\rm I}}\left(\sum_{U_{\rm I}}
%\hat t_iI(i\in s_{\rm I})\right)^2\right]
%\]
Now take the expectation of the two sums individually.
We have
\begin{align*}
E\left[\sum_{s_{\rm I}}\hat t_i^2\right]
&=
E_{\rm I}\left[\sum_{s_{\rm I}}E_{\rm II}(\hat t_i^2)\right]\\
&=
E_{\rm I}\left[
\sum_{s_{\rm I}}t_i^2+\sum_{s_{\rm I}}V_i\right]\\
&=
\sum_{U_{\rm I}}\pi_{\mathrm Ii}t_i^2
+\sum_{U_{\rm I}}\pi_{\mathrm Ii}V_i
\end{align*}
and
%
%
\begin{align*}
E\left[-{1\over n_{\rm I}}\left(\sum_{s_{\rm I}}
\hat t_i\right)^2\right]
&=-{1\over n_{\rm I}}E_{\rm I}\left[
\underset{s_{\rm I}}{\sum\sum}
E_{\rm II}(\hat t_i\hat t_j)\right]\\
&=
-{1\over n_{\rm I}}E_{\rm I}\left[\underset{s_{\rm I}}{\sum\sum}
t_it_j+\sum_{s_{\rm I}}V_i\right]\\
&=-{1\over n_{\rm I}}\left(
\underset{U_{\rm I}}{\sum\sum}
\pi_{\mathrm Iij}t_it_j+\sum_{U_{\rm I}}\pi_{\mathrm Ii}V_i\right)\\
\end{align*}
%
%
%
Therefore
\begin{align*}
E(\widehat V^{**})&=
{N_{\rm I}^2\over n_{\rm I}(n_{\rm I}-1)}\left[
\sum_{U_{\rm I}}\pi_{\mathrm Ii}t_i^2
+\sum_{U_{\rm I}}\pi_{\mathrm Ii}V_i
-{1\over n_{\rm I}}\underset{U_{\rm I}}{\sum\sum}\pi_{\mathrm Iij}t_it_j
-{1\over n_{\rm I}}\sum_{U_{\rm I}}\pi_{\mathrm Ii}V_i
\right]\\
&=
{N_{\rm I}^2\over n_{\rm I}(n_{\rm I}-1)}\left[
{n_{\rm I}\over N_{\rm I}}\sum_{U_{\rm I}}t_i^2
%+{n_{\rm I}\over N_{\rm I}}\sum_{U_{\rm I}}V_i
-{1\over n_{\rm I}}\underset{U_{\rm I}}{\sum\sum}\pi_{\mathrm Iij}t_it_j
+{n_{\rm I}\over N_{\rm I}}\sum_{U_{\rm I}}V_i
-{1\over N_{\rm I}}\sum_{U_{\rm I}}V_i
\right]\\
&=
{N_{\rm I}\over n_{\rm I}-1}\sum_{U_{\rm I}}t_i^2
-{N_{\rm I}^2\over n_{\rm I}^2(n_{\rm I}-1)}\underset{U_{\rm I}}{\sum\sum}
\pi_{\mathrm Iij}t_it_j
+{N_{\rm I}\over n_{\rm I}}\sum_{U_{\rm I}}V_i
\end{align*}

%\newpage

Note that for the $V_i$ term in the above equation we have
\[
{N_{\rm I}\over n_{\rm I}}\sum_{U_{\rm I}}V_i=V_{SSU}
\]
Therefore
\[
E(\widehat V^{**})-V_{SSU}=
{N_{\rm I}\over n_{\rm I}-1}\sum_{U_{\rm I}}t_i^2
-{N_{\rm I}^2\over n_{\rm I}^2(n_{\rm I}-1)}
\underset{U_{\rm I}}{\sum\sum}\pi_{\mathrm Iij}t_it_j
\]
In order to show that $\widehat V^{**}$ overestimates the variance we need to show
that
\[
E(\widehat V^{**})-V_{SSU}
%{N_{\rm I}\over n_{\rm I}-1}\sum_{U_{\rm I}}t_i^2
%-{N_{\rm I}^2\over n_{\rm I}^2(n_{\rm I}-1)}\sum_{U_{\rm I}}\pi_{\mathrm Iij}t_it_j
\ge V_{PSU}
\]
For SI we have
\[
V_{PSU}={N_{\rm I}^2\over n_{\rm I}^2}\underset{U_{\rm I}}{\sum\sum}
(\pi_{\mathrm Iij}-\pi_{\mathrm Ii}\pi_{\mathrm Ij})t_it_j
=
{N_{\rm I}^2\over n_{\rm I}^2}\underset{U_{\rm I}}{\sum\sum}\pi_{\mathrm Iij}t_it_j
-\underset{U_{\rm I}}{\sum\sum}t_it_j
\]
The following side calculation is for combining the double sums.
\[
{N_{\rm I}^2\over n_{\rm I}^2(n_{\rm I}-1)}
+{N_{\rm I}^2\over n_{\rm I}^2}
=
{N_{\rm I}^2+N_{\rm I}^2(n_{\rm I}-1)\over n_{\rm I}^2(n_{\rm I}-1)}
={N_{\rm I}^2\over n_{\rm I}(n_{\rm I}-1)}
\]
Rearranging and collecting
terms we have the following expression for
$E(\widehat V^{**})-V_{SSU}\stackrel{?}{\ge}V_{PSU}$.
\[
{N_{\rm I}\over n_{\rm I}-1}\sum_{U_{\rm I}}t_i^2
+\underset{U_{\rm I}}{\sum\sum}t_it_j
\stackrel{?}{\ge}
{N_{\rm I}^2\over n_{\rm I}(n_{\rm I}-1)}
\underset{U_{\rm I}}{\sum\sum}\pi_{\mathrm Iij}t_it_j
\]
Partition the right hand side.
\[
{N_{\rm I}\over n_{\rm I}-1}\sum_{U_{\rm I}}t_i^2
+\underset{U_{\rm I}}{\sum\sum}t_it_j
\stackrel{?}{\ge}
{N_{\rm I}\over N_{\rm I}-1}
\underset{i\ne j}{\sum\sum}t_it_j
+
{N_{\rm I}\over n_{\rm I}-1}\sum_{U_{\rm I}}t_i^2
\]
Cancel.
\[
\underset{U_{\rm I}}{\sum\sum}t_it_j
\stackrel{?}{\ge}
{N_{\rm I}\over N_{\rm I}-1}
\underset{i\ne j}{\sum\sum}t_it_j
\]
Add a sum over $t_i^2$ to both sides to eliminate the $i\ne j$.
\[
\underset{U_{\rm I}}{\sum\sum}t_it_j
+{N_{\rm I}\over N_{\rm I}-1}\sum_{U_{\rm I}}t_i^2
\stackrel{?}{\ge}
{N_{\rm I}\over N_{\rm I}-1}
\underset{U_{\rm I}}{\sum\sum}t_it_j
\]
Rearrange.
\[
{N_{\rm I}\over N_{\rm I}-1}\sum_{U_{\rm I}}t_i^2
\stackrel{?}{\ge}
{1\over N_{\rm I}-1}
\underset{U_{\rm I}}{\sum\sum}t_it_j
\]
Multiply both sides by $N_{\rm I}-1$.
\[
N_{\rm I}\sum_{U_{\rm I}}t_i^2
\ge
\underset{U_{\rm I}}{\sum\sum}t_it_j
\]
The above is a true statement that will be proved on the next page.
It follows from the above that
\[
E(\widehat V^{**})-V_{SSU}\ge V_{PSU}
\]
Therefore $\widehat V^{**}$ overestimates the variance.

%\newpage

What follows is a proof that
\[
N\sum_{i=1}^Nt_i^2\ge\sum_{i=1}^N\sum_{j=1}^Nt_it_j
%=\left(\sum_{i=1}^N\sum_{j=1}^Nt_it_j\right)^2
\]
Begin with
\[
(t_i-t_j)^2\ge 0
\]
Hence
\[
t_i^2+t_j^2\ge2t_it_j
\]
It follows that
\[
Nt_i^2+\sum_{j=1}^Nt_j^2\ge2t_i\sum_{j=1}^Nt_j
\]
Sum over all $i$.
\[
\sum_{i=1}^N\left[Nt_i^2+\sum_{j=1}^Nt_j^2\right]
\ge\sum_{i=1}^N\left[2t_i\sum_{j=1}^Nt_j\right]
\]
Rearrange.
\[
N\sum_{i=1}^Nt_i^2+N\sum_{j=1}^Nt_j^2
\ge2\sum_{i=1}^Nt_i\sum_{j=1}^Nt_j
\]
Divide both sides by 2 and rearrange.
\[
N\sum_{i=1}^Nt_i^2
\ge
\sum_{i=1}^N\sum_{j=1}^Nt_it_j
\]
QED

%\newpage

\section*{4.23}
The cluster totals are $4M-8$, $4M-4$, $4M$, $4M+4$, and $4M+8$.
We have
\[
\bar t_{U_{\rm I}}={1\over N_{\rm I}}\sum_{U_{\rm I}}t_i=4M
\]
Hence
\begin{align*}
S_{tU_{\rm I}}^2&={1\over N_{\rm I}-1}
\sum_{U_{\rm I}}(t_i-\bar t_{U_{\rm I}})^2\\
&=
{50\over249}\left[
(4M-8-4M)^2+(4M-4-4M)^2+(4M-4M)^2
+(4M+4-4M)^2+(4M+8-4M)^2\right]\\
&=32.1285
\end{align*}
We are given $S_{yU_i}^2=V_0$. Since each cluster has four elements we have
\[
{1\over3}\sum_{U_i}(y_k-\bar y_{U_i})^2=V_0
\]
For a cluster with mean $M-2$ we have
\begin{align*}
\sum_{U_i}(y_k-M+2)^2
&=
\sum_{U_i}y_k^2+4\sum_{U_i}y_k-2M\sum_{U_i}y_k-16M+4M^2+16\\
&=
\sum_{U_i}y_k^2+4(4M-8)-2M\sum_{U_i}y_k-16M+4M^2+16\\
&=
\sum_{U_i}y_k^2-2M\sum_{U_i}y_k+4M^2-16\\
\end{align*}
%
For a cluster with mean $M-1$ we have
\begin{align*}
\sum_{U_i}(y_k-M+1)^2
&=
\sum_{U_i}y_k^2+2\sum_{U_i}y_k-2M\sum_{U_i}y_k-8M+4M^2+4\\
&=
\sum_{U_i}y_k^2+2(4M-4)-2M\sum_{U_i}y_k-8M+4M^2+4\\
&=
\sum_{U_i}y_k^2-2M\sum_{U_i}y_k+4M^2-4
\end{align*}
%
For a cluster with mean $M$ we have
\[
\sum_{U_i}(y_k-M)^2=\sum_{U_i}y_k^2-2M\sum_{U_i} y_k+4M^2
\]
%
For a cluster with mean $M+1$ we have
\begin{align*}
\sum_{U_i}(y_k-M-1)^2
&=
\sum_{U_i}y_k^2-2\sum_{U_i}y_k-2M\sum_{U_i}y_k+8M+4M^2+4\\
&=
\sum_{U_i}y_k^2-2(4M+4)-2M\sum_{U_i}y_k+8M+4M^2+4\\
&=
\sum_{U_i}y_k^2-2M\sum_{U_i}y_k+4M^2-4
\end{align*}
%
Finally, for a cluster with mean $M+2$ we have
\begin{align*}
\sum_{U_i}(y_k-M-2)^2
&=
\sum_{U_i}y_k^2-4\sum_{U_i}y_k-2M\sum_{U_i}y_k+16M+4M^2+16\\
&=
\sum_{U_i}y_k^2-4(4M+8)-2M\sum_{U_i}y_k+16M+4M^2+16\\
&=
\sum_{U_i}y_k^2-2M\sum_{U_i}y_k+4M^2-16\\
\end{align*}
It follows that by summing over all 250 clusters $U_i$ we obtain
\[
\sum_{k=1}^{1000}y_k^2-2M\sum_{k=1}^{1000}y_k+1000M^2-2000
=250\cdot3V_0=750V_0
\]
%
Hence the population variance is
\[
S_{yU}^2={1\over999}\sum_{k=1}^{1000}(y_k-M)^2={750V_0+2000\over999}
\]

(a) $V_0=5$
\[
V_{SIC}(\hat t_\pi)={250^2\over25}
\left(1-{25\over250}\right)\times32.1285=72,289
\]
\[
V_{SI}(\hat t_\pi)={1000^2\over 100}\left(1-{100\over1000}\right)
{750\cdot5+2000\over999}
=51,802
\]
Hence $V_{SI}<V_{SIC}$.

(b) $V_0=10$
\[
V_{SI}(\hat t_\pi)={1000^2\over 100}\left(1-{100\over1000}\right)
{750\cdot10+2000\over999}
=85,586
\]
Hence $V_{SIC}<V_{SI}$.

%\newpage

\section*{2.}
In the table below, {\tt muni} is the estimate of the number of municipalities.
The true values of P75, P85, and number of municipalities are all within the
confidence intervals of the estimates.
\begin{verbatim}
                                 The MEANS Procedure

                              Variable             Sum
                              ------------------------
                              P75              8182.00
                              P85              8339.00
                              ------------------------

                              The SURVEYMEANS Procedure

                                    Data Summary

                        Number of Strata                   5
                        Number of Clusters                10
                        Number of Observations            20
                        Sum of Weights            281.896848


                                     Statistics

          Variable             Sum         Std Dev        95% CL for Sum
          -----------------------------------------------------------------
          P75          8306.277004     1229.020764    5146.97855 11465.5755
          P85          8662.886910     1201.077191    5575.41970 11750.3541
          muni          281.896848       12.114596     250.75529   313.0384
          -----------------------------------------------------------------

dm "output;clear";
dm "log;clear";

* Read the mu284 data;

data mu284;
infile 'c:\sas\mu284.txt';
input LABEL   P85   P75   RMT85   CS82   SS82   S82   ME84   REV84   REG   CL;
drop RMT85 CS82 SS82 S82 ME84 REV84 REG;
run;

* Print the true sums;

proc means data = mu284 sum;
var P75 P85;
run;

* Compute the sum of P75 for each cluster;

proc sort data = mu284;
by CL;
run;
proc means data = mu284 noprint;
by CL; * must be sorted on CL;
var P75;
output out = clusters sum = P75TOT;
run;

* Stratify;

proc sort data = clusters;
by P75TOT;
run;
data clusters;
set clusters;
ST = 1 + floor((_n_ - 1) / 10);
run;

* Stage I : Choose two clusters from each stratum using Brewer's method;

proc surveyselect seed = 764532 data = clusters method = brewer out = selections;
strata ST; * must be sorted on ST;
size P75TOT;
run;

* Merge stage I selections back into mu284;

proc sort data = selections;
by CL;
run;
data mu284;
merge mu284 selections;
by CL; * both must be sorted on CL;
run;

* Drop the observations that are not in selected clusters;

data mu284;
set mu284;
if ST ^= .;
w = samplingweight; * save for later;
drop selectionprob samplingweight; * to prevent surveyselect warning;
run;

* Choose two samples within each cluster using SI;

proc sort data = mu284;
by CL;
run;
proc surveyselect seed = 198777 data = mu284 method = srs sampsize = 2 out = sample;
strata CL; * must be sorted on CL;
run;

* Correct the sampling weights;
* For example, cluster 2 is selected from stratum 4;
* Cluster 2 has _FREQ_ = 5 observations;
* 2 observations are selected from cluster 2;
* The selection probability is 2 / _FREQ_;
* Hence multiply the weight by _FREQ_ / 2;

data sample;
set sample;
w = w * _FREQ_ / 2;
muni = 1; * for estimating number of municipalities;
run;

* Print the estimated sums;
* There are ten clusters per stratum, hence total = 10;

proc surveymeans data = sample total = 10 sum clsum;
stratum ST;
cluster CL;	* only affects the variance calculation;
var P75 P85 muni;
weight w;
run;
\end{verbatim}

%\newpage

\section*{Appendix}

R code for Exercise 4.8.

\begin{verbatim}
rm(list=ls(all=TRUE))
DATA = read.table("mu284.txt",header=T)
CS82 = DATA$CS82
print(sum(CS82))

f = function(a,b,n,N) {
pikl = rbind(
c(n / N, n * (n - 1) / N / (N - 1)),
c(n * (n - 1) / N / (N - 1), n / N))
Delta = pikl - (n / N) ^ 2
sum(Delta / pikl * (c(a,b) %o% c(a,b)) / (n / N)^2)
}

V1 = f(8,10,2,6)
V2 = f(13,12,2,5)
V3 = f(11,8,2,5)
V4 = f(2,7,2,5)
V5 = f(16,10,2,7)

Vi = c(V1,V2,V3,V4,V5)
print(Vi)

nI = 5
NI = 50
pikl = matrix(2/245,5,5)
diag(pikl) = 1/10
Delta = pikl - (nI / NI)^2
ti = c(54,62.5,47.5,22.5,91)
V = sum(Delta / pikl * (ti %o% ti) / (nI / NI)^2)
print(V)

Vi = c(24,3.75,33.75,93.75,315)
print(sum(90*Vi))
print(V-sum(90*Vi))

# part (c)
V = -1/2 * sum(Delta / pikl * (outer(ti,ti,"-")*10)^2 )
print(V)

t = 10 * sum(ti)
piIi = nI / NI
V = 1 / nI / (nI - 1) * sum((ti / (piIi / nI)  - t) ^ 2)
print(V)
\end{verbatim}


\section*{5.2}
For BE we have
\begin{align*}
\hat t_{y\pi}&={1\over\pi}\sum_sy_k={1\over0.3}\times669,281=2,230,937\\
\hat t_{z\pi}&={1\over\pi}\sum_sz_k={1\over0.3}\times608,902=2,029,673
\end{align*}
Therefore the estimate of $R$ is
\[
\widehat R={\hat t_{y\pi}\over\hat t_{z\pi}}=1.099
\]
For BE we have $\Delta_{kl}=0$ for $k\ne l$ hence
\begin{align*}
\widehat V(\widehat R)
&=
{1\over\hat t_{z\pi}^2}
\underset{s}{\sum\sum}
{\Delta_{kl}\over\pi_{kl}}
{y_k-\widehat Rz_k\over\pi_k}
{y_l-\widehat Rz_l\over\pi_l}\\
&=
{1\over\hat t_{z\pi}^2}
{\pi-\pi^2\over\pi}
{1\over\pi}
{1\over\pi}
\sum_s(y_k-\widehat Rz_k)^2\\
&=
{1\over2,029,673^2}\times{0.3-0.09\over0.027}\times3,496,228,001\\
&=
0.0066
\end{align*}

\[
cve(\widehat R)={\sqrt{\widehat V(\widehat R)}\over\widehat R}
={\sqrt{0.0066}\over1.099}\approx7.4\%
\]

%\newpage

\section*{5.4}
(a) Compute an estimate of the percentage increase in total population from
1980 to 1983.
The 1983 population is $y$ and the 1980 population is $z$.

We have
\begin{align*}
\hat t_{y\pi}={N\over n}\sum_sy_k={124\over40}\times1560.4=4837.24\\
\hat t_{z\pi}={N\over n}\sum_sz_k={124\over40}\times1447.7=4487.87
\end{align*}
Hence the estimate is
\[
100\times{\hat t_{y\pi}-\hat t_{z\pi}\over\hat t_{z\pi}}=7.78
\]
(b) Calculate an estimate of the variance of the estimator in (a).

Let $\widehat R=\hat t_{y\pi}/\hat t_{z\pi}$. Then from p.\ 179 of the book
\begin{align*}
\widehat V(\widehat R)
&=
{n\over\left(\sum_sz_k\right)^2}
\left(1-{n\over N}\right){1\over n-1}
\sum_s(y_k-\widehat Rz_k)^2\\
&=
{n\over\left(\sum_sz_k\right)^2}
\left(1-{n\over N}\right){1\over n-1}
\left(\sum_sy_k^2-2\widehat R\sum_sy_kz_k+\widehat R^2\sum_sz_k^2\right)\\
&=
{40\over(1447.7)^2}
\left(1-{40\over124}\right)
{1\over40-1}
(609,833.24-2\times1.0778\times558,395.94+(1.0778)^2\times511,520.87)\\
&=0.00012
\end{align*}
The variance of the estimator in part (a) is
\[
\widehat V(100\widehat R-100)=100^2\times0.00012=1.2
\]

(c) The total of the variable P80 is 4,308.1.
Use this information to calculate an unbiased estimate of the percentage in (a).
Also, calculate an unbiased variance estimate of the unbiased estimator.
Comment on your results.

The unbiased estimate is
\[
100\times{\hat t_{y\pi}-t_z\over t_z}
=
100\times{4837.24-4308.1\over4308.1}=12.3
\]
We have
\[
S^2_{ys}={1\over n-1}\sum_sy_k^2-{1\over n(n-1)}\left(\sum_sy_k\right)^2
=14075.9
\]
Hence
\[
\widehat V(\hat t_{y\pi})={N^2\over n^2}\left(1-{n\over N}\right)S_{ys}^2
=3,665,380
\]
The variance of the unbiased estimator is
\[
\widehat V\left(100\cdot{\hat t_{y\pi}\over4308.1}-100\right)
=\left(100\over4308.1\right)^2\times3,665,380=1975
\]
Comment: The estimator in part (a) is more efficient
because the residuals $y_k-\widehat Rz_k$ are small.

%\newpage

\section*{5.8}
\[
\widehat B
=
{{N\over n}\sum_sz_ky_k\over{N\over n}\sum_sz_k^2}
=
{582,220\over72,768}=8.00
\]
From p.\ 179 of the book
\begin{align*}
\widehat V(\widehat B)
&=
{n\over\left(\sum_sz_k^2\right)^2}
\left(1-{n\over N}\right){1\over n-1}
\sum_s(z_ky_k-\widehat Bz_k^2)^2\\
&=
{n\over\left(\sum_sz_k^2\right)^2}
\left(1-{n\over N}\right){1\over n-1}
\left(\sum_sz_k^2y_k^2-2\widehat B\sum_sz_k^3y_k+\widehat B^2\sum_sz_k^4\right)\\
&=
{50\over72768^2}
\left(1-{50\over281}\right){1\over50-1}
%&\qquad\times
\left(4.13155\times10^{10}-16\times4,911,948,478
+64\times586,102,776\right)\\
&=0.037
\end{align*}
The confidence interval is
\[
8\pm1.96\times\sqrt{0.037}=8\pm0.38
\]

%\newpage

\section*{2.}
(a) Here are the SAS results for $n=25$ and $M=1000$.
\begin{verbatim}
                                 The MEANS Procedure

                      Variable            Mean        Variance
                      ----------------------------------------
                      theta          2.4020437       5.2311402
                      V              3.7993593       4.8782876
                      ----------------------------------------
\end{verbatim}
i. From the above table,
\begin{align*}
{1\over M}\sum_{i=1}^M\widehat\theta^{(i)}&=2.402\\
{1\over M}\sum_{i=1}^M\widehat V_{taylor}^{(i)}&=3.799\\
{\tt var}(\widehat\theta^{(i)})&=5.231
\end{align*}
ii. The true value of $\theta$ is $1.919$ hence
the approximate theoretical bias of $\widehat\theta$ is
\[
2.402-1.919=0.483
\]
iii. The approximate theoretical bias of $\widehat V_{taylor}$ is
\[
3.799-5.231=-1.432
\]

%\newpage

(b) Here are the SAS results for $n=75$ and $M=1000$.
\begin{verbatim}
                                 The MEANS Procedure

                      Variable            Mean        Variance
                      ----------------------------------------
                      theta          2.0638230       1.6836495
                      V              1.3602821       0.2800629
                      ----------------------------------------
\end{verbatim}
i. From the above table,
\begin{align*}
{1\over M}\sum_{i=1}^M\widehat\theta^{(i)}&=2.064\\
{1\over M}\sum_{i=1}^M\widehat V_{taylor}^{(i)}&=1.360\\
{\tt var}(\widehat\theta^{(i)})&=1.684
\end{align*}
ii. The true value of $\theta$ is $1.919$ hence
the approximate theoretical bias of $\widehat\theta$ is
\[
2.064-1.919=0.145
\]
iii. The approximate theoretical bias of $\widehat V_{taylor}$ is
\[
1.360-1.684=-0.324
\]

(c) Comment:
Increasing $n$ caused the estimated
bias of $\widehat\theta$ and $\widehat V$ to decrease.
Taylor linearization underestimated the variance
as cautioned in Remark 5.5.3 on p.\ 176.

*\newpage

\section*{Appendix}

\begin{verbatim}
data mu284;
infile 'c:\sas\mu284.txt';
input LABEL   P85   P75   RMT85   CS82   SS82   S82   ME84   REV84   REG   CL;
drop RMT85 CS82 SS82 S82 ME84 REV84 REG CL;
run;

* Sample M = 1000 times;

proc surveyselect seed = 32907 data = mu284 method = srs n = 25 reps = 1000 out = sample;
run;

* Calculate yk^2, yk * zk, zk^2;

data sample;
set sample;
yy = P85 * P85;
yz = P85 * P75;
zz = P75 * P75;
run;

* Calculate sums over yk, zk, yk^2, yk * zk, zk^2;

proc means data = sample sum noprint;
by replicate;
var P85 P75 yy yz zz;
output out = foo sum = sum_y sum_z sum_yy sum_yz sum_zz;
run;

* Calculate theta and V_taylor;

data foo;
set foo;
n = _FREQ_;
R = sum_y / sum_z;
V = 10000 * n / sum_z ** 2 * (1 - n / 284) / (n - 1) *
(sum_yy - 2 * R * sum_yz + R * R * sum_zz);
theta = 100 * R - 100;
run;

* Print the mean and variance of theta and V_taylor;

proc means data = foo mean var;
var theta V;
run;
\end{verbatim}


\section*{6.3}
For the estimated total we have
%
\begin{align*}
\hat t_y
&=
\sum_U(3x_{1k}+2x_{2k})+\sum_s{y_k-3x_{1k}-2x_{2k}\over n/N}\\
&=
3\sum_Ux_{1k}+2\sum_Ux_{2k}+{N\over n}\left(
\sum_sy_k-3\sum_sx_{1k}-2\sum_sx_{2k}\right)\\
&=
3\cdot2508+2\cdot6153-{281\over35}(1019-3\cdot321-2\cdot803)\\
&=
7386
\end{align*}
%
The sample variance statistic is
%
\begin{align*}
S^2
&=
{1\over n-1}\sum_s\bigg[y_k-3x_{1k}-2x_{2k}-\overline{y_k-3x_{1k}-2x_{2k}}\bigg]^2\\
&=
{1\over n-1}\sum_s\bigg[y_k-3x_{1k}-2x_{2k}-\bar y_k+3\bar x_{1k}+2\bar x_{2k}\bigg]^2\\
&=
{1\over n-1}\sum_s\bigg[
(y-\bar y_k)-3(x_{1k}-\bar x_{1k})-2(x_{2k}-\bar x_{2k})\bigg]^2\\
&=
{1\over n-1}\sum_s\bigg[(y_k-\bar y_k)^2-6(y_k-\bar y_k)(x_{1k}-\bar x_{1k})
-4(y_k-\bar y_k)(x_{2k}-\bar x_{2k})\\
&\qquad\qquad\qquad\qquad
{}+9(x_{1k}-\bar x_{1k})^2+12(x_{1k}-\bar x_{1k})(x_{2k}-\bar x_{2k})
+4(x_{2k}-\bar x_{2k})^2\bigg]\\
&=
S_{ys}^2-6S_{yx_1s}-4S_{yx_2s}+9S_{x_1s}^2+12S_{x_1x_2s}+4S_{x_2s}^2\\
&=
742.34-6\cdot82.24-4\cdot152.86
+9\cdot20.44+12\cdot8.22+4\cdot60.41\\
&=161.7
\end{align*}
%
Hence
%
\[
\widehat V(\hat t_y)={N^2\over n}\left(1-{n\over N}\right)S^2
={281^2\over35}\left(1-{35\over281}\right)161.7
=319362
\]
%
The 95\% confidence interval is
%
\[
7386\pm1.96\sqrt{319262}=7386\pm1108
\]

%\newpage

\section*{6.5}
Assuming SI sampling with $f=n/N$, the bias of $\hat t_{y,\rm alt}$ is
\begin{align*}
B(\hat t_{y,\rm alt})
&=
E(\hat t_{y,\rm alt})-t_y\\
&=
E\left[
\sum_Uy_k^0+\sum_sD_k\right]-\sum_Uy_k\\
&=
E\left[\sum_Uy_k^0+\sum_UD_kI(k\in s)\right]-\sum_Uy_k\\
&=
\sum_Uy_k^0+{n\over N}\sum_UD_k-\sum_Uy_k\\
&=
{n\over N}\sum_UD_k-\sum_UD_k\\
&=
\left({n\over N}-1\right)\sum_UD_k
\end{align*}
Therefore $\hat t_{y,\rm alt}$ is biased if $n<N$ and $\sum_UD_k\ne0$.

For the variance of $\hat t_{y,\rm alt}$ we have
\[
V(\hat t_{y,\rm alt})
=V\left(\sum_Uy_k^0+\sum_sD_k\right)
=\underset{U}{\sum\sum}\Delta_{kl}D_kD_l
\]
For the estimator (6.3.4) we have
\[
\hat t_{y,\rm dif}=\underset{U}{\sum\sum}\Delta_{kl}
{D_k\over\pi_k}{D_l\over\pi_l}
\]
Therefore, assuming SI sampling we have
\[
V(\hat t_{y,\rm alt})={n^2\over N^2}V(\hat t_{y,\rm dif})
\]
Consequently
\[
V(\hat t_{y,\rm alt})\le V(\hat t_{y,\rm dif})
\]
For the mean square error of $\hat t_{y,\rm alt}$ we have
\[
MSE(\hat t_{y,\rm alt})
=
V(\hat t_{y,\rm alt})+[B(\hat t_{y,\rm alt})]^2\\
=
{n^2\over N^2}V(\hat t_{y,\rm dif})+[B(\hat t_{y,\rm alt})]^2
\]
Since $\hat t_{y,\rm dif}$ is unbiased we have
$MSE(\hat t_{y,\rm dif})=V(\hat t_{y,\rm dif})$.
Therefore
\[
MSE(\hat t_{y,\rm alt})={n^2\over N^2}MSE(\hat t_{y,\rm dif})+[B(\hat t_{y,\rm alt})]^2
\]

%\newpage

\section*{6.14}
Show that it is always true that
\[
\sum_s g_{ks}{e_{ks}\over\pi_k}=\sum_s{e_{ks}\over\pi_k}
\]
We have
\begin{align*}
\sum_sg_{ks}{e_{ks}\over\pi_k}
&=
\sum_s\left[
1+(\mathbf t_x-\hat{\mathbf t}_{x\pi})^T
\left(
\sum_s{\mathbf x_k\mathbf x_k^T\over\sigma_k^2\pi_k}
\right)^{-1}
{\mathbf x_k\over\sigma_k^2}
\right]
{e_{ks}\over\pi_k}\\
%
&=
\sum_s{e_{ks}\over\pi_k}
+(\mathbf t_x-\hat{\mathbf t}_{x\pi})^T
\left(
\sum_s{\mathbf x_k\mathbf x_k^T\over\sigma_k^2\pi_k}
\right)^{-1}
\left(\sum_s{\mathbf x_k\over\sigma_k^2}{e_{ks}\over\pi_k}\right)\\
%
&=
\sum_s{e_{ks}\over\pi_k}
+(\mathbf t_x-\hat{\mathbf t}_{x\pi})^T
\left(
\sum_s{\mathbf x_k\mathbf x_k^T\over\sigma_k^2\pi_k}
\right)^{-1}
\left(\sum_s{\mathbf x_k\over\sigma_k^2}
{y_k-\mathbf x_k^T\widehat{\mathbf B}\over\pi_k}\right)\\
%
&=
\sum_s{e_{ks}\over\pi_k}
+(\mathbf t_x-\hat{\mathbf t}_{x\pi})^T
\left(
\sum_s{\mathbf x_k\mathbf x_k^T\over\sigma_k^2\pi_k}
\right)^{-1}
\left(
\sum_s{\mathbf x_ky_k\over\sigma_k^2\pi_k}
-\sum_s{\mathbf x_k\mathbf x_k^T\widehat{\mathbf B}\over\sigma_k^2\pi_k}
\right)\\
&=
\sum_s{e_{ks}\over\pi_k}
+(\mathbf t_x-\hat{\mathbf t}_{x\pi})^T
\left[
\left(
\sum_s{\mathbf x_k\mathbf x_k^T\over\sigma_k^2\pi_k}
\right)^{-1}
\left(\sum_s{\mathbf x_ky_k\over\sigma_k^2\pi_k}\right)
-\left(
\sum_s{\mathbf x_k\mathbf x_k^T\over\sigma_k^2\pi_k}
\right)^{-1}
\left(\sum_s{\mathbf x_k\mathbf x_k^T\over\sigma_k^2\pi_k}\right)
\widehat{\mathbf B}
\right]\\
&=
\sum_s{e_{ks}\over\pi_k}
+(\mathbf t_x-\hat{\mathbf t}_{x\pi})^T
\left[\widehat{\mathbf B}-\widehat{\mathbf B}\right]\\
&=
\sum_s{e_{ks}\over\pi_k}
\end{align*}

%\newpage

\section*{7.15}
Consider the BE design with constant inclusion probability $\pi$ for all elements.
Use Result 7.3.1 to find the approximate variance and the variance estimator
of the ratio estimator $\hat t_{yra}$ given by (7.3.2).
Show that roughly the same variance is obtained for the ratio estimator under SI
sampling of $n$ from $N$, if $n=N\pi$, which is the expected sample size.

The approximate variance is
\begin{align*}
AV(\hat t_{yra})
&=
\underset{U}{\sum\sum}
\Delta_{kl}{y_k-Bx_k\over\pi_k}{y_l-Bx_l\over\pi_l}\\
&=
{\pi-\pi^2\over\pi^2}\sum_U(y_k-Bx_k)^2 & \hbox{because $\Delta_{kl}=0$ for $k\ne l$}\\
&=
\left({1\over\pi}-1\right)
\sum_U(y_k-Bx_k)^2
\end{align*}
The variance estimator is
\begin{align*}
\widehat V(\hat t_{yra})
&=
\underset{s}{\sum\sum}
{\Delta_{kl}\over\pi_{kl}}
{g_{ks}(y_k-\widehat Bx_k)\over\pi_k}
{g_{ls}(y_l-\widehat Bx_l)\over\pi_l}\\
&=
{\pi-\pi^2\over\pi^3}\sum_sg_{ks}^2(y_k-\widehat Bx_k)^2\\
&=
{1\over\pi}\left({1\over\pi}-1\right)
\left({\sum_Ux_k\over{1\over\pi}\sum_s{x_k}}\right)^2
\sum_s(y_k-\widehat Bx_k)^2\\
&=
(1-\pi)\left({\sum_Ux_k\over\sum_s{x_k}}\right)^2
\sum_s(y_k-\widehat Bx_k)^2
\end{align*}
For SI with $n=N\pi$ the variance is
\begin{align*}
V(\hat t_{yra})
&=
{N^2\over N\pi}\left(1-{N\pi\over N}\right)
\left\{
{1\over N-1}\sum_s(y_k-Bx_k)^2-
{1\over N(N-1)}\left[\sum_s(y_k-Bx_k)\right]^2\right\}\\
&=
{N\over N-1}\left({1\over\pi}-1\right)\sum_s(y_k-Bx_k)^2
-{1\over N-1}\left({1\over\pi}-1\right)\left[\sum_s(y_k-Bx_k)\right]^2\\
&\approx
\left({1\over\pi}-1\right)\sum_s(y_k-Bx_k)^2
\end{align*}
This is roughly the same variance as for BE, especially for large $N$.
For the variance estimator we have
\begin{align*}
\widehat V(\hat t_{tya})
&=
{N^2\pi^2\over N^2}
\left(\sum_Ux_k\over\sum_sx_k\right)^2{N^2\over N\pi}
\left(1-{N\pi\over N}\right)
{1\over N\pi-1}\sum_s(y_k-\widehat Bx_k)^2\\
&={N\pi\over N\pi-1}(1-\pi)\left(\sum_Ux_k\over\sum_sx_k\right)^2
\sum_s(y_k-\widehat Bx_k)^2
\end{align*}
This is also roughly the same as for BE.

%\newpage

\section*{9.2}
(a) The inclusion probability is
\[
\pi^*=\pi_{ak}\pi_{k\mid s_a}={160\over284}\times{20\over40}=0.2817
\]
Therefore the estimated total of SS82 is
\begin{align*}
\hat t_{\pi^*}&=\sum_{h=1}^4\sum_{s_h}{y_k\over\pi^*}\\
&=20\cdot(17.05+19.75+22.40+32.25)/0.2817\\
&=6422
\end{align*}

(b) Unbiased estimators of $V_1$ and $V_2$.
\[
N=284
\qquad
n_a=160
\qquad
n_{ah}=40
\qquad
n_h=20
\]
\[
\delta_h={1\over n_h}{n_a-n_{ah}\over n_a-1}
={1\over20}\times{160-40\over 160-1}=0.03774
\]
\[
w_{ah}={n_{ah}\over n_a}={40\over160}=0.25
\]
\[
\hat{\bar y}_U={\hat t_{\pi^*}\over N}={6422\over 284}=22.61
\]
\begin{align*}
\widehat V_1
&=
{N^2\over n_a}\left(1-{n_a\over N}\right)
\left[
\sum_{h=1}^4w_{ah}(1-\delta_h)S_{ys_h}^2
+{n_a\over n_a-1}\sum_{h=1}^4w_{ah}(\bar y_{s_h}-\hat{\bar y}_U)^2
\right]\\
&=
12407.1\\
\\
\widehat V_2
&=N^2\sum_{h=1}^4w_{ah}^2{1-n_h/n_{ah}\over n_h}S_{ys_h}^2\\
&=
14534.5
\end{align*}

(c) Compute an approximately 95\% confidence interval for the total SS82.
\[
6422\pm1.96\sqrt{12407.1+14534.5}=6422\pm322
\]

%\newpage

(d) Estimate the variance that would be obtained with a single-pass SI sample
of size $n=80$.

We have
\begin{align*}
S_{s_1}^2&={1\over19}\sum_{s_1}y_k^2-{20\over19}\times17.05^2=19.945\\
S_{s_2}^2&={1\over19}\sum_{s_2}y_k^2-{20\over19}\times19.75^2=24.197\\
S_{s_3}^2&={1\over19}\sum_{s_3}y_k^2-{20\over19}\times22.40^2=28.359\\
S_{s_4}^2&={1\over19}\sum_{s_4}y_k^2-{20\over19}\times31.25^2=42.829
\end{align*}
Therefore
\[
\sum_{s_a}y_k^2=19\sum_{h=1}^4S_{ys_h}^2+20\sum_{h=1}^4\bar y_{s_h}^2=45373
\]
In addition
\[
\sum_{s_a}y_k=20\cdot(17.05+19.75+22.40+31.25)=1809
\]
Therefore
\[
S_{ys}^2
={\sum y_k^2\over n-1}-{\left(\sum y_k\right)^2\over n(n-1)}
={45373\over80-1}-{1809^2\over80(80-1)}=56.544
\]
Finally, the estimated variance for an SI sample of size $n=80$ is
\[
\widehat V(\hat t_{\pi})
={N^2\over n}\left(1-{n\over N}\right)S_{ys}^2
={284^2\over80}\left(1-{80\over284}\right)\times56.544=40949.5
\]

\end{document}

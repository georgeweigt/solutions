\section*{18.}
A doctor assumes that a patient has one of three diseases
$d_1$, $d_2$, or $d_3$. Before
any test, he assumes an equal probability for each disease.
He carries out a
test that will be positive with probability {.}8 if the patient has $d_1$,
{.}6 if he has
disease $d_2$, and {.}4 if he has disease $d_3$.
Given that the outcome of the test was
positive, what probabilities should the doctor now assign to the three possible
diseases?

\bigskip
\noindent
We have
\begin{eqnarray*}
P(d_i)&=&1/3\\
P(T\mid d_1)&=&0.8\\
P(T\mid d_2)&=&0.6\\
P(T\mid d_3)&=&0.4
\end{eqnarray*}
Use Bayes' formula.
\begin{eqnarray*}
P(d_1\mid T)&=&{P(T\mid d_1)P(d_1)
\over P(T\mid d_1)P(d_1)+P(T\mid d_2)P(d_2)+P(T\mid d_3)P(d_3)
}\\
&=&{(0.8)(1/3)\over(0.8)(1/3)+(0.6)(1/3)+(0.4)(1/3)}\\
&=&0.44
\end{eqnarray*}
%
\begin{eqnarray*}
P(d_2\mid T)&=&{P(T\mid d_2)P(d_2)
\over P(T\mid d_1)P(d_1)+P(T\mid d_2)P(d_2)+P(T\mid d_3)P(d_3)
}\\
&=&{(0.6)(1/3)\over(0.8)(1/3)+(0.6)(1/3)+(0.4)(1/3)}\\
&=&0.33
\end{eqnarray*}
%
\begin{eqnarray*}
P(d_3\mid T)&=&{P(T\mid d_3)P(d_3)
\over P(T\mid d_1)P(d_1)+P(T\mid d_2)P(d_2)+P(T\mid d_3)P(d_3)
}\\
&=&{(0.4)(1/3)\over(0.8)(1/3)+(0.6)(1/3)+(0.4)(1/3)}\\
&=&0.22
\end{eqnarray*}


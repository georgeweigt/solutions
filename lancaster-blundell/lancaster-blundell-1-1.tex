\documentclass[12pt]{article}
\usepackage{amsmath}
\usepackage{tikz}

\parindent=0pt

\begin{document}

(1.1)
Use Fermat's principle of least time to derive Snell's law.

\bigskip
\hrule

\begin{center}
\begin{tikzpicture}
\draw (0,2) -- (0,-2);
\draw[dashed] (-2,0) -- (2,0);
\draw (-1.5,1.8) node {air};
\draw (1.5,1.8) node {glass};
\draw[thick] (-2,-1.5) node[anchor=east] {$A$} -- (0,0);
\draw[thick,->] (0,0) -- (2,0.8) node[anchor=west] {$B$};
\draw (-1.5,-0.5) node {$\theta_1$};
\draw (1.5,0.3) node {$\theta_2$};
\end{tikzpicture}
\end{center}




\end{document}

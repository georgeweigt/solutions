\documentclass[12pt]{article}
\usepackage{amsmath}
\usepackage{tikz}
\parindent=0pt
\begin{document}

(1.1)
Use Fermat's principle of least time to derive Snell's law.

\bigskip
\hrule

\begin{center}
\begin{tikzpicture}
\draw (0,2) -- (0,-2);
\draw[dashed] (-2,0) -- (2,0);
\draw (-1.5,1.8) node {air};
\draw (1.5,1.8) node {glass};
\draw[thick] (-2,-1.5) node[anchor=east] {$A=(x_1,y_1)$} -- (0,0) node[anchor=north west] {$C=(x,y)$};
\draw[thick,->] (0,0) -- (2,0.8) node[anchor=west] {$B=(x_2,y_2)$};
\draw (-1.5,-0.5) node {$\theta_1$};
\draw (1.5,0.3) node {$\theta_2$};
\end{tikzpicture}
\end{center}

A light ray travels from $A$ to $B$ by going through $C$.
Let $v_1$ be the velocity of light in air and $v_2$ the velocity in glass.
The time $t$ to go from $A$ to $B$ is
\begin{equation*}
t=\frac{d_1}{v_1}+\frac{d_2}{v_2}
\end{equation*}
where
\begin{equation*}
d_1=\sqrt{(x-x_1)^2+(y-y_1)^2}\qquad
d_2=\sqrt{(x-x_2)^2+(y-y_2)^2}
\end{equation*}

Differentiate $t$ with respect to $y$ and set the result to zero to obtain an equation that minimizes $t$.
(The $x$ coordinate of $C$ is fixed by the boundary between air and glass.)
\begin{equation*}
\frac{dt}{dy}=\frac{y-y_1}{v_1d_1}+\frac{y-y_2}{v_2d_2}=0
\end{equation*}

Rewrite as
\begin{equation*}
\frac{y-y_1}{v_1d_1}=\frac{y_2-y}{v_2d_2}
\end{equation*}

It follows that
\begin{equation*}
\frac{\sin\theta_1}{v_1}=\frac{\sin\theta_2}{v_2}
\end{equation*}

Hence
\begin{equation*}
n_1\sin\theta_1=n_2\sin\theta_2
\end{equation*}
where
\begin{equation*}
n_1=\frac{c}{v_1}\qquad n_2=\frac{c}{v_2}
\end{equation*}

\end{document}

\documentclass[12pt]{article}
\usepackage{amsmath}
\parindent=0pt
\begin{document}

(14.2)
{\it A demonstration that the photon has spin-1, with
only two spin polarizations.}

\bigskip
A photon $\gamma$ propagates with momentum
$q^\mu=(|\mathbf q|,0,0,|\mathbf q|)$.
Working with a basis where the two transverse photon polarizations are
$\epsilon_{\lambda=1}^\mu(q)=(0,1,0,0)$
and $\epsilon_{\lambda=2}^\mu(q)=(0,0,1,0)$, it may be
shown, using Noether's theorem, that the operator
$\hat S^z$, whose eigenvalue is the $z$-component spin
angular momentum of the photon, obeys the commutation relation
\begin{equation*}
\left[\hat S^z,\hat a_{\mathbf q\lambda}^\dag\right]
=i\epsilon_\lambda^{\mu=1*}(q)\hat a_{\mathbf q\lambda=2}^\dag
-i\epsilon_\lambda^{\mu=2*}(q)\hat a_{\mathbf q\lambda=1}^\dag
\tag{14.36}
\end{equation*}

(i) Define creation operators for the circular polarizations via
\begin{equation*}
\begin{aligned}
\hat b_{\mathbf qR}^\dag&=-\frac{1}{\sqrt2}
\left(\hat a_{\mathbf q1}^\dag+i\hat a_{\mathbf q2}^\dag\right)
\\
\hat b_{\mathbf qL}^\dag&=\frac{1}{\sqrt2}
\left(\hat a_{\mathbf q1}^\dag-i\hat a_{\mathbf q2}^\dag\right)
\end{aligned}
\tag{14.37}
\end{equation*}

Show that
\begin{equation*}
\begin{aligned}
\left[\hat S^z,\hat b_{\mathbf qR}^\dag\right]&=\hat b_{\mathbf qR}^\dag
\\
\left[\hat S^z,\hat b_{\mathbf qL}^\dag\right]&=-\hat b_{\mathbf qL}^\dag
\end{aligned}
\tag{14.38}
\end{equation*}

\end{document}

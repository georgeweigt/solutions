\documentclass[12pt]{article}
\usepackage{amsmath}
\parindent=0pt
\begin{document}

(7.1)
For the Lagrangian $\mathcal L$ given by
\begin{equation*}
\mathcal L=\frac{1}{2}\partial^\mu\partial_\mu\phi
-\frac{1}{2}m^2\phi^2-\sum_{n=1}^\infty\lambda_n\phi^{2n+2}
\tag{7.20}
\end{equation*}
show that the equation of motion is given by
\begin{equation*}
(\partial^2+m^2)\phi
+\sum_{n=1}^\infty\lambda_n(2n+2)\phi^{2n+1}=0
\tag{7.21}
\end{equation*}

\bigskip
\hrule

\bigskip
For the Lagrangian $\mathcal L$ given in (7.20) we have
\begin{equation*}
\frac{\partial\mathcal L}{\partial\phi}=-m^2\phi-\sum_{n=1}^\infty\lambda_n(2n+2)\phi^{2n+1}
\tag{1}
\end{equation*}
and
\begin{equation*}
\frac{\partial\mathcal L}{\partial(\partial_\mu\phi)}
=\frac{\partial}{\partial(\partial_\mu\phi)}\left(\frac{1}{2}\partial^2\phi\right)
=\partial^\mu\phi
\tag{2}
\end{equation*}

Consider the following Euler-Lagrange equation.
\begin{equation*}
\partial_\mu\left(\frac{\partial\mathcal L}{\partial(\partial_\mu\phi)}\right)-\frac{\partial\mathcal L}{\partial\phi}=0
\tag{3}
\end{equation*}

Substitute (1) and (2) into (3) to obtain
\begin{equation*}
\partial_\mu\partial^\mu\phi+m^2\phi+\sum_{n=1}^\infty\lambda_n(2n+2)\phi^{2n+1}=0
\end{equation*}
which is equivalent to (7.21).

\bigskip
Note: Recall that
\begin{equation*}
\partial^\mu\partial_\mu\phi=\partial_\mu\partial^\mu\phi
=(\partial_0^2+\partial_1^2+\partial_2^2+\partial_3^2)\phi
\end{equation*}

\end{document}

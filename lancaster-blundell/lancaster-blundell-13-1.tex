\documentclass[12pt]{article}
\usepackage{amsmath}
\parindent=0pt
\begin{document}

(13.1)
(a) Show that the conserved charge in eqn 13.16
may be written
\begin{equation*}
\hat{\mathbf Q}_{N_c}=\int d^3p\,\hat{\mathbf A}_p^\dag
\mathbf J\hat{\mathbf A}_p
\tag{13.41}
\end{equation*}
where $\hat{\mathbf A}_p=(\hat a_{1p},\hat a_{2p},\hat a_{3p})$
and $\mathbf J=(J_x,J_y,J_z)$ are the spin-1 angular momentum
matrices from Chapter 9.

\bigskip
(b) Use the transformations from Exercise 3.3 to
find the form of the angular momentum matrices
appropriate to express the charge as
$\hat{\mathbf Q}_{N_c}=\int d^3p\,
\hat{\mathbf B}_p^\dag\mathbf J\hat{\mathbf B}_p$
where $\hat{\mathbf B}_p=(\hat b_{1p},\hat b_{0p},\hat b_{-1p})$.

\bigskip
\hrule

\bigskip
\begin{equation*}
%\mathbf Q_{N_c}=\int d^3x\,(\mathbf\Phi\times\partial_0\mathbf\Phi),
%\quad
\hat Q_{N_c}^a=-i\int d^3p\,\varepsilon^{abc}\hat a_{bp}^\dag\hat a_{cp}
\tag{13.16}
\end{equation*}

Recall that
\begin{equation*}
\varepsilon^{1bc}=\begin{pmatrix}
0&0&0
\\
0&0&1
\\
0&-1&0
\end{pmatrix},
\quad
\varepsilon^{2bc}=\begin{pmatrix}
0&0&-1
\\
0&0&0
\\
1&0&0
\end{pmatrix},
\quad
\varepsilon^{3bc}=\begin{pmatrix}
0&1&0
\\
-1&0&0
\\
0&0&0
\end{pmatrix}
\end{equation*}

It follows that
\begin{align*}
\varepsilon^{1bc}\hat a_{bp}^\dag\hat a_{cp}
&=(\hat a_{1p}^\dag,\hat a_{2p}^\dag,\hat a_{3p}^\dag)
\begin{pmatrix}
0&0&0
\\
0&0&1
\\
0&-1&0
\end{pmatrix}
\begin{pmatrix}\hat a_{1p}\\\hat a_{2p}\\\hat a_{3p}\end{pmatrix}
=\hat a_2^\dag\hat a_3-\hat a_3^\dag\hat a_2
\\
\varepsilon^{2bc}\hat a_{bp}^\dag\hat a_{cp}
&=(\hat a_{1p}^\dag,\hat a_{2p}^\dag,\hat a_{3p}^\dag)
\begin{pmatrix}
0&0&-1
\\
0&0&0
\\
1&0&0
\end{pmatrix}
\begin{pmatrix}\hat a_{1p}\\\hat a_{2p}\\\hat a_{3p}\end{pmatrix}
=\hat a_3^\dag\hat a_1-\hat a_1^\dag\hat a_3
\\
\varepsilon^{3bc}\hat a_{bp}^\dag\hat a_{cp}
&=(\hat a_{1p}^\dag,\hat a_{2p}^\dag,\hat a_{3p}^\dag)
\begin{pmatrix}
0&1&0
\\
-1&0&0
\\
0&0&0
\end{pmatrix}
\begin{pmatrix}\hat a_{1p}\\\hat a_{2p}\\\hat a_{3p}\end{pmatrix}
=\hat a_1^\dag\hat a_2-\hat a_2^\dag\hat a_1
\end{align*}

From Chapter 9
\begin{equation*}
J^x=i\begin{pmatrix}
0&0&0&0
\\
0&0&0&0
\\
0&0&0&-1
\\
0&0&1&0
\end{pmatrix},
\;
J^y=i\begin{pmatrix}
0&0&0&0
\\
0&0&0&1
\\
0&0&0&0
\\
0&-1&0&0
\end{pmatrix},
\;
J^z=i\begin{pmatrix}
0&0&0&0
\\
0&0&-1&0
\\
0&1&0&0
\\
0&0&0&0
\end{pmatrix}
\end{equation*}

Hence
\begin{equation*}
\hat{\mathbf A}_p^\dag\mathbf J\hat{\mathbf A}_p=?
\end{equation*}

\end{document}

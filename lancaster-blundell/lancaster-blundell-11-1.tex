\documentclass[12pt]{article}
\usepackage{amsmath}
\parindent=0pt
\begin{document}

(11.1)
One of the criteria we had for a successful theory
of a scalar field was that the commutator for space-like
separations would be zero. Let's see if our scalar
field has this feature. Show that
\begin{equation*}
[\hat\phi(x),\hat\phi(y)]=\int
\frac{d^3p}{(2\pi)^3}\frac{1}{2E_p}
\big(
\exp(-ip\cdot(x-y))-\exp(-ip\cdot(y-x))
\big)
\tag{11.51}
\end{equation*}

For space-like separation we are able to swap $(y-x)$
in the second term to $(x-y)$. This gives us zero,
as required.

\bigskip
\hrule

\bigskip
Consider equation (11.12).
\begin{equation*}
\hat\phi(x)=\int\frac{d^3p}{(2\pi)^\frac{3}{2}}\frac{1}{(2E_p)^\frac{1}{2}}
\left(\hat a_p\exp(-ip\cdot x)+\hat a_p^\dag\exp(ip\cdot x)\right)
\tag{11.12}
\end{equation*}

It follows from (11.12) that
\begin{equation*}
[\hat\phi(x),\hat\phi(y)]=\frac{1}{(2\pi)^3}
\int\frac{d^3p}{(2E_p)^\frac{1}{2}}\int\frac{d^3q}{(2E_q)^\frac{1}{2}}(PQ-QP)
\tag{1}
\end{equation*}
where
\begin{align*}
P&=\hat a_p\exp(-ip\cdot x)+\hat a_p^\dag\exp(ip\cdot x)
\\
Q&=\hat a_q\exp(-iq\cdot y)+\hat a_q^\dag\exp(iq\cdot y)
\end{align*}

Expanding the commutator in (1) we have
\begin{multline*}
PQ-QP=
[\hat a_p,\hat a_q]\exp(-ip\cdot x-iq\cdot y)
+[\hat a_p,\hat a_q^\dag]\exp(-ip\cdot x+iq\cdot y)
\\
{}+[\hat a_p^\dag,\hat a_q]\exp(ip\cdot x-iq\cdot y)
+[\hat a_p^\dag,\hat a_q^\dag]\exp(ip\cdot x+iq\cdot y)
\end{multline*}

Then from the commutation relations
\begin{equation*}
[\hat a_p,\hat a_q]=0
\qquad
[\hat a_p^\dag,\hat a_q^\dag]=0
\qquad
[\hat a_p,\hat a_q^\dag]=\delta(p-q)
\end{equation*}
we have
\begin{equation*}
PQ-QP=\delta(p-q)\big(\exp(-ip\cdot x+iq\cdot y)-\exp(ip\cdot x-iq\cdot y)\big)
\tag{2}
\end{equation*}

Substitute (2) into (1) to obtain
\begin{equation*}
[\hat\phi(x),\hat\phi(y)]=\frac{1}{(2\pi)^3}
\int\frac{d^3p}{2E_p}\big(\exp(-ip\cdot(x-y))-\exp(-ip\cdot(y-x))\big)
\end{equation*}

% FIXME The notation $\hat a(p)$ would be more symmetrical with $\hat\phi(x)$.

\end{document}

\documentclass[12pt]{article}
\usepackage{amsmath}
\parindent=0pt
\begin{document}

(10.3)
For the Lagrangian
\begin{equation*}
\mathcal L=\frac{1}{2}(\partial_\mu\phi)^2-\frac{1}{2}m^2\phi^2
\tag{10.42}
\end{equation*}
evaluate $T^{\mu\nu}$ and show that $T^{00}$ agrees with what
you would expect from the Hamiltonian for this Lagrangian.
Show that $\partial_\mu T^{\mu\nu}=0$. Derive expressions
for $P^0=\int d^3x\,T^{00}$ and $P^i=\int d^3x\,T^{0i}$.

\bigskip
\hrule

\bigskip
From the book near equation (10.27) we have
\begin{equation*}
T^{\mu\nu}=\Pi^\mu\partial^\nu\phi-g^{\mu\nu}\mathcal L
\end{equation*}

where
\begin{equation*}
\Pi^\mu=\frac{\partial\mathcal L}{\partial(\partial_\mu\phi)}
\end{equation*}

It follows from $g^{00}=1$ that
\begin{equation*}
T^{00}=\Pi^0\partial^0\phi-\frac{1}{2}(\partial_0\phi)^2+\frac{1}{2}m^2\phi^2
\end{equation*}

For the Lagrangian given in (10.42) we have
\begin{equation*}
\Pi^0=\frac{\partial\mathcal L}{\partial(\partial_0\phi)}=\partial_0\phi
\end{equation*}

Hence
\begin{equation*}
T^{00}=\partial_0\phi\partial^0\phi-\mathcal L
\end{equation*}

Noting that
\begin{equation*}
\partial_0\phi\partial^0\phi=(\partial_0\phi)^2
\end{equation*}

we have
\begin{equation*}
T^{00}=(\partial_0\phi)^2-\mathcal L=\frac{1}{2}(\partial_0\phi)^2+\frac{1}{2}m^2\phi^2
\end{equation*}

The expected Hamiltonian $H$ is
\begin{equation*}
H=p\dot q-\mathcal L
=\frac{\partial\mathcal L}{\partial(\partial_0\phi)}\partial_0\phi-\mathcal L
=T^{00}
\end{equation*}

\end{document}

\beginsection{Mandl and Shaw Problem 1.3}

For Thomson scattering of an unpolarized beam of photons, obtain the
differential cross-section for scattering through an angle $\theta$,
with the scattered radiation being linearly polarized in a given
direction. By considering two mutually perpendicular such directions,
use your result to re-derive Eq.~(1.69a) for the unpolarized differential
cross-section. Show that for $\theta=90^\circ$, the scattered beam is
100 per cent linearly polarized in the direction of the normal to the
plane of scattering.

\beginsection{Solution}

Thomson scattering is the scattering of a photon by an electron.
In this model the photon energy must be small so that the electron is
pretty much unaffected.
First let's review the cast of vectors:
\medskip
\settabs 5 \columns
\+& Incident & Scattered\cr
\+& photon & photon \cr
\smallskip
\+& $k$ & $k^\prime$ & Direction of propagation\cr
\+& $\hat k$ & ${\hat k}^\prime$ & Unit direction of propagation\cr
\+& $\epsilon_1$ & ${\epsilon_1}^\prime$ & Unit direction of electric field\cr
\+& $\epsilon_2$ & ${\epsilon_2}^\prime$ & Unit direction of magnetic field\cr
\medskip
The key thing to keep in mind is the projection property of
an orthonormal coordinate system. For orthonormal unit vectors
$\epsilon_1$, $\epsilon_2$ and $\hat k$ and for unit vector $u$
we have
$$(\epsilon_1\cdot u)^2+(\epsilon_2\cdot u)^2+(\hat k\cdot u)^2=|u|^2=1\eqno({\rm A})$$
The differential cross section formula is
$$\sigma_{\alpha\rightarrow\beta}\,d\Omega
=r_0^2(\epsilon_\alpha\cdot\epsilon_\beta^\prime)^2\,d\Omega$$
For an unpolarized beam we average over the initial polarization states.
Letting $\sigma_1$ be the differential cross section for final polarization
state $\epsilon_1^\prime$ we have
$$\eqalign{
\sigma_1&=\hbox{$1\over2$}\sum_{\alpha=1}^2 r_0^2(\epsilon_\alpha
\cdot\epsilon_1^\prime)^2\cr
&=\hbox{$1\over2$}r_0^2[(\epsilon_1\cdot\epsilon_1^\prime)^2
+(\epsilon_2\cdot\epsilon_1^\prime)^2]\cr
}$$
By A we have
$$(\epsilon_1\cdot\epsilon_1^\prime)^2
+(\epsilon_2\cdot\epsilon_1^\prime)^2=1-(\hat k\cdot\epsilon_1^\prime)^2$$
so we can put
$$\sigma_1=\hbox{$1\over2$}r_0^2[1-(\hat k\cdot\epsilon_1^\prime)^2]$$
Now what we want to do is express $\sigma_1$ in terms of the scattering angle
$\theta$, that is, the angle between $k$ and $k^\prime$.
Again by A we note that
$$(\hat k\cdot\epsilon_1^\prime)^2=1-(\hat k\cdot\epsilon_2^\prime)^2
-(\hat k\cdot\hat k^\prime)^2$$
so we can put
$$\sigma_1=\hbox{$1\over2$}r_0^2[(\hat k\cdot\epsilon_2^\prime)^2
+(\hat k\cdot\hat k^\prime)^2]$$
From $a\cdot b=|a||b|\cos\theta$ we have $(\hat k\cdot\hat k^\prime)^2
=\cos^2\theta$ and
$$\sigma_1=\hbox{$1\over2$}r_0^2[(\hat k\cdot\epsilon_2^\prime)^2
+\cos^2\theta]$$
Likewise for $\sigma_2$ we have
$$\sigma_2=\hbox{$1\over2$}r_0^2[(\hat k\cdot\epsilon_1^\prime)^2
+\cos^2\theta]$$
The unpolarized differential cross section is
$$\sigma=\sigma_1+\sigma_2=\hbox{$1\over2$}r_0^2[(\hat k\cdot\epsilon_1^\prime)^2
+(\hat k\cdot\epsilon_2^\prime)^2+2\cos^2\theta]$$
Again by A we have
$$\eqalign{
\sigma&=\hbox{$1\over2$}r_0^2[(1-\cos^2\theta)+2\cos^2\theta]\cr
&=\hbox{$1\over2$}r_0^2(1+\cos^2\theta)
}$$
which is Eq.~(1.69a).


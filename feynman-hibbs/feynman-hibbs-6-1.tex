\documentclass[12pt]{article}
\usepackage{amsmath}
\usepackage{amssymb}

\parindent=0pt

\newcommand\INT{\int_{\mathbb R^3}}

\begin{document}

6-1.
Suppose the potential can be written as $U+V$,
where $V$ is small but $U$ is large.
Suppose further that the kernel for motion in the potential of $U$ alone
can be worked out (for example, $U$ might be quadratic in $x$ and
independent of time).
Show that the motion in the complete potential $U+V$ is described by
equations (6.4), (6.11), (6.13), and (6.14) with $K_0$ replaced by $K_U$,
where $K_U$ is the kernel for motion in the potential $U$ alone.

\bigskip
\hrule

\bigskip
From equation (6.1) we have
\begin{equation*}
K_{U+V}(b,a)=\int_a^b
\exp\left(\frac{i}{\hbar}\int_{t_a}^{t_b}\left(\tfrac{1}{2}m\dot x^2-U-V\right)\,dt\right)
\mathcal Dx(t)
\end{equation*}

Consider equation (6.5).
\begin{equation*}
K_0(b,a)=\int_a^b\exp\left(\frac{i}{\hbar}\int_{t_a}^{t_b}\tfrac{1}{2}m\dot x^2\,dt\right)
\mathcal Dx(t)
\tag{6.5}
\end{equation*}

By hypothesis the kernel for $U$ is known hence
\begin{equation*}
K_U(b,a)=\int_a^b
\exp\left(\frac{i}{\hbar}\int_{t_a}^{t_b}\left(\tfrac{1}{2}m\dot x^2-U\right)\,dt\right)
\mathcal Dx(t)
\end{equation*}

In the expansion of $V$, replace
\begin{equation*}
\exp\left(\frac{i}{\hbar}\int_{t_a}^{t_b}\tfrac{1}{2}m\dot x^2\,dt\right)
\end{equation*}
with
\begin{equation*}
\exp\left(\frac{i}{\hbar}\int_{t_a}^{t_b}\left(\tfrac{1}{2}m\dot x^2-U\right)\,dt\right)
\end{equation*}

This corresponds to replacing $K_0$ with $K_U$ in (6.4), etc.

\end{document}

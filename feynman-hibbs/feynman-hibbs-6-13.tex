\documentclass[12pt]{article}
\usepackage{amsmath}
\usepackage{amssymb}

\parindent=0pt

\newcommand\INT{\int_{\mathbb R^3}}

\begin{document}

6-13.
Assume that $V(\mathbf r,t)$ is independent of time.
Substitute the free particle kernel $K_0$ into equation
(6.61) and integrate over $t_c$ to show that
\begin{multline*}
\psi(b)=
\exp\left(-\frac{iE_at_b}{\hbar}\right)
\exp\left(\frac{i\mathbf p_a\cdot\mathbf x_b}{\hbar}\right)
\\
{}-\exp\left(-\frac{iE_at_b}{\hbar}\right)
\frac{m}{2\pi\hbar^2}
\int
\frac{1}{R_{bc}}
\exp\left(\frac{ipR_{bc}}{\hbar}\right)
V(\mathbf x_c)
\exp\left(\frac{i\mathbf p_a\cdot\mathbf x_c}{\hbar}\right)
\,d^3\mathbf x_c
\tag{6.62}
\end{multline*}
where $R_{bc}$ is the distance from the variable point of integration
$\mathbf x_c$ to the final point $\mathbf x_b$ and $p$ is the
magnitude of the momentum of the electron.

\bigskip
\hrule

\bigskip
This is equation (6.61).
\begin{multline*}
\psi(b)=
\exp\left(-\frac{iE_at_b}{\hbar}\right)
\exp\left(\frac{i\mathbf p_a\cdot\mathbf x_b}{\hbar}\right)
\\
{}-\frac{i}{\hbar}\int_0^{t_b}\int
K_0(b,c)V(c)
\exp\left(-\frac{iE_at_c}{\hbar}\right)
\exp\left(\frac{i\mathbf p_a\cdot\mathbf x_c}{\hbar}\right)
\,d^3\mathbf x_c\,dt_c
\tag{6.61}
\end{multline*}

This is the integral over $t_c$ from (6.61) with $V(c)$ independent of time.
\begin{equation*}
I=\int_0^{t_b}K_0(b,c)
\exp\left(-\frac{iE_at_c}{\hbar}\right)
\,dt_c
\end{equation*}

Substitute $E_a=p^2/2m$.
\begin{equation*}
I=\int_0^{t_b}K_0(b,c)
\exp\left(-\frac{ip^2t_c}{2m\hbar}\right)
\,dt_c
\end{equation*}

Substitute $K_0$ from problem 4-12.
\begin{equation*}
I=\int_0^{t_b}
\left(\frac{m}{2\pi i\hbar(t_b-t_c)}\right)^{3/2}
\exp\left(\frac{imR_{bc}^2}{2\hbar(t_b-t_c)}\right)
\exp\left(-\frac{ip^2t_c}{2m\hbar}\right)
\,dt_c
\end{equation*}

Let
\begin{align*}
f&=\left(\frac{m}{2\pi i\hbar(t_b-t_c)}\right)^{3/2}
\\
g&=\frac{m R_{bc}^2}{2(t_b-t_c)}-\frac{p^2t_c}{2m}
\\
\lambda&=\frac{1}{\hbar}
\end{align*}

Then
\begin{equation*}
I=\int_0^{t_b}f\exp(i\lambda g)\,dt_c
\end{equation*}

The phase of the exponential is stationary (i.e., $g'=0$) for
\begin{equation*}
t_c=t_b-\frac{mR_{bc}}{p}
\end{equation*}

By the method of stationary phase
\begin{equation*}
I\approx\pm\left(\frac{2\pi i}{\lambda g''}\right)^{1/2}
f\exp(i\lambda g)\bigg|_{t_c}
\end{equation*}

Hence
\begin{equation*}
I\approx-\frac{im}{2\pi\hbar R_{bc}}
\exp\left(\frac{ipR_{bc}}{\hbar}-\frac{ip^2t_b}{2m\hbar}\right)
\end{equation*}

The integral can also be written as
\begin{equation*}
-\frac{im}{2\pi\hbar}
\frac{1}{R_{bc}}
\exp\left(\frac{ipR_{bc}}{\hbar}\right)
\exp\left(-\frac{iE_at_b}{\hbar}\right)
\tag{1}
\end{equation*}

Substitute (1) into (6.61) to obtain (6.62).

\bigskip
Note that the method of stationary phase requires $0<t_c<t_b$ so the above
solution is valid for physical values that satisfy
\begin{equation*}
0<\frac{mR_{bc}}{p}<t_b
\end{equation*}

Hence (6.62) is not true in general but is {\it probably} true if $t_b$ is large.

\bigskip
Just for the fun of it, check physical dimensions.
\begin{equation*}
\frac{mR_{bc}}{p}\propto
\frac{\text{mass}\times\text{length}}{\text{mass}\times\text{length}/\text{time}}
=\text{time}
\end{equation*}

\end{document}

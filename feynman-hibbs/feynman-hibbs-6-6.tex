\documentclass[12pt]{article}
\usepackage{amsmath}
\usepackage{amssymb}

\parindent=0pt

\newcommand\INT{\int_{\mathbb R^3}}

\begin{document}

6-6.
Suppose the potential is that of a central force.
Thus $V(\mathbf r)=V(r)$.
Show that $v(\breve{\mathbf p})$ can be written as
\begin{equation*}
v(\breve{\mathbf p})=v(\breve p)=\frac{4\pi\hbar}{\breve p}
\int_0^\infty\sin\left(\frac{\breve pr}{\hbar}\right)V(r)r\,dr
\tag{6.45}
\end{equation*}

\bigskip
\hrule

\bigskip
Consider equation (6.39).
\begin{equation*}
v(\breve{\mathbf p})=\int\exp\left(\frac{i\breve{\mathbf p}\cdot\mathbf r}{\hbar}\right)
V(\mathbf r)d^3\mathbf r
\tag{6.39}
\end{equation*}

Convert (6.39) to polar coordinates.
\begin{equation*}
v(\breve{\mathbf p})=\int_0^{2\pi}\int_0^\pi\int_0^\infty
\exp\left(\frac{i\breve pr\cos\theta}{\hbar}\right)
V(r)r^2\sin\theta\,dr\,d\theta\,d\phi
\end{equation*}

Integrate over $\phi$.
\begin{equation*}
v(\breve{\mathbf p})=2\pi\int_0^\pi\int_0^\infty
\exp\left(\frac{i\breve pr\cos\theta}{\hbar}\right)
V(r)r^2\sin\theta\,dr\,d\theta
\end{equation*}

Convert the exponential to rectangular form.
\begin{equation*}
v(\breve{\mathbf p})=2\pi\int_0^\pi\int_0^\infty
\left(\cos\left(\frac{\breve pr\cos\theta}{\hbar}\right)
+i\sin\left(\frac{\breve pr\cos\theta}{\hbar}\right)\right)
V(r)r^2\sin\theta\,dr\,d\theta
\end{equation*}

Integrate over $\theta$.
\begin{equation*}
v(\breve{\mathbf p})=\frac{4\pi\hbar}{\breve p}\int_0^\infty
\sin\left(\frac{\breve pr}{\hbar}\right)
V(r)r\,dr
\end{equation*}

The above result is due to the integrals
\begin{align*}
\int_0^\pi\cos(A\cos\theta)\sin\theta\,d\theta&=\frac{2\sin A}{A}
\\
\int_0^\pi\sin(A\cos\theta)\sin\theta\,d\theta&=0
\end{align*}

(6-6 cont'd)
Suppose $v(r)$ is the Coulomb potential $-Ze^2/r$.
In this case the integral for $v(\breve p)$ is oscillatory at the upper limit.
But convergence of the integral can be artificially forced by
introducing the factor $e^{-\epsilon r}$ and then taking the limit
of the result as $\epsilon\rightarrow0$.
Following through this calculation,
show that the cross section corresponds to the Rutherford cross section
\begin{equation*}
\frac{d\sigma_\text{Ruth}}{d\Omega}
=\frac{4m^2Z^2e^4}{\breve p^4}
\tag{6.46}
\end{equation*}

From (6.45) and the above hypothesis we have
\begin{equation*}
v(\breve p)=-\frac{4\pi\hbar Ze^2}{\breve p}
\int_0^\infty\sin\left(\frac{\breve pr}{\hbar}\right)\exp(-\epsilon r)\,dr
\end{equation*}

Solve the integral.
\begin{equation*}
v(\breve p)=-\frac{4\pi\hbar Ze^2}{\breve p}\frac{\breve p/\hbar}{(\breve p/\hbar)^2+\epsilon^2}
\end{equation*}

Let $\epsilon\rightarrow0$.
\begin{equation*}
v(\breve p)=-\frac{4\pi\hbar^2Ze^2}{\breve p^2}
\end{equation*}

By equation (6.44)
\begin{equation*}
\frac{d\sigma}{d\Omega}=\left(\frac{m}{2\pi\hbar^2}\right)^2
|v(\breve p)|^2
=\frac{4m^2Z^2e^4}{\breve p^4}
\end{equation*}

\end{document}

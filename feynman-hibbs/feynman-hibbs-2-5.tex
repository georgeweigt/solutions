\documentclass[12pt]{article}
\usepackage{amsmath}

\parindent=0pt

\begin{document}

2-5.
Classically, the energy is defined as
\begin{equation*}
E=\dot xp-L
\tag{2.12}
\end{equation*}

Show that the energy at a final point is
\begin{equation*}
\dot x_b\left(\frac{\partial L}{\partial\dot x}\right)_{x=x_b}
-L(x_b)=-\frac{\partial S_{cl}}{\partial t_b}
\tag{2.13}
\end{equation*}
while the energy at an initial point is
\begin{equation*}
+\frac{\partial S_{cl}}{\partial t_a}
\end{equation*}

{\it Hint:} A change in the time of an end point requires a change in path,
since all paths must be classical paths.

\bigskip
\hrule

\bigskip
We have
\begin{equation*}
\dot x_b=\left(\frac{dx}{dt}\right)_{t=t_b}
\end{equation*}

\end{document}
